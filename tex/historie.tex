\documentclass{report}

\input{../templates/preamble}
\input{../templates/macros}
\input{../templates/letterfonts}

\title{\Huge{marcus}\ historie}
\author{\Huge{Marcus Allen Denslow}}
\date{2026-02-02}

\begin{document}

\maketitle
\newpage% or \cleardoublepage
% \pdfbookmark[<level>]{<title>}{<dest>}
\pdfbookmark[section]{\contentsname}{toc}
\tableofcontents
\pagebreak

\chapter{Kristendommen i Norge på 1800 tallet}
\section{Kirkens makt}
I starten av århundret hadde kirken stor makt over vanlige folk. Prestens ord var lov. Det car ikke lov for andre enn kirkens menn å forskynne Guds Ord

\section{Hans Nilsen Hauge} 
Hans Nilsen Hauge (1771 - 1824) trosset dette. Han reiste rundt og forkynte budskap nåde, omvendelse og frihet. Han starter flere industriforetak - saltkoking - forløper for å starte industrialiseringen i Norge.

\section{Stor bevegelse} 
Det blir dannet en stor bevegelse som ble styrt av lekfolk (Folk som ikke er utdannet prester).

\section{Relgionsfrihet} 
Det var ikke religionsfrihet i Norge. Hauge satt derfor flere år i fengsel for å forkyne til folk uten utdannelse men bevegelsen hans spredde seg over hele landet men også ikke-religiøse bevegelser (politiske) ble startet.

\chapter{Kristendommen i Norge på 1900-tallet}
\section{Kirker} 
mange nye kirker ble bygd mellom 1850 og 1930 -> det can krav om at det skulle romme minst 1/3 av folket i byene så statskirken står sterkt fremdeles. Men det blir også andre kristne kirkesamfunn som baptister, metodister, pinsevenn, frikirker - staten har ikke lenger monopol på forkynnelse - vi har fått religionsfrihet. 

\section{klarere skiller} 
Det blir klarere killer mellom troende og ikke-troende. Det blir en økense sekularisering utover 1900-tallet.

\section{En pluralistisk periode} 
Fra 1970 kommer det mange innvandrere til Norge. De har med seg andre religioner. Islam er den roligionen som blir lagt mest merke til. Det er vedlig få nordmann som går over til Islam, men religionen vokser i Norge grunnet mer innvandring og store fødetall

\section{år 2000 og videre} 
Norge har et stort mangfold av religioner og livssyn. Verdier: toleranse, mangfold, demokrati, men også politisk korrekthet.

something





\end{document}
