\documentclass{report}

\input{../templates/preamble}
\input{../templates/macros}
\input{../templates/letterfonts}

\title{\Huge{Electon Self Energy Corrections}}
\author{\Huge{Marcus Allen Denslow}}
\date{2026-01-09}

\begin{document}

\maketitle
\newpage% or \cleardoublepage
% \pdfbookmark[<level>]{<title>}{<dest>}
\pdfbookmark[section]{\contentsname}{toc}
\tableofcontents
\pagebreak

If one calculates the energy of a point charge using classical electromagnitism, the result is infinate, yet as far as we know, the electron is point charge. One can calculate the energy needed to assemble an electron due, essentially, to the interaction of the electron with its own field. A uniform charge distribution with the classical radius of an electron, we have an energy $\displaystyle m_{e}c^2$. Experiments have probed the electron's charge distribution and found that it is consistent with a point charge down to distances much smallen than the classical radius. Beyond classical calculations, the self energy of the electron calculated in the quantum theory of Dirac is still infinate but the divergences are less severe. 
\\
At this pint we must take the unpleasant position (constant) infinate energy should just be subtracted when we consider the overall zero of energy (as we did for the field energy in the vacuum). Electrons exist and don't carry infinate amount of energy baggage so we just subtract off the infinate constant. Nevertheless, we will find that the electron's self energy may change when it is a bound state and we should account for this change in out energy level calculations. This calculation will also give us the opportunity to understand resonant behaviour in scattering.
\\
We can calculate the lowest order self energy corrections represented by the two Feynman diagrams below.

\begin{figure}[h!]
    \centering
    \includegraphics[width=0.6\textwidth]{/home/bolla/Pictures/latex.png}
\end{figure}

In these, a photon is emitted then reabsorbed. As we now know, both of these amplitutesare in order $\displaystyle e^2$. The first one comes from the $\displaystyle A^2$ term in which the number of photons changes by zero or two and the second comes from the $\displaystyle \vec{A} \cdot \vec{p}$ term in second order time dependent pertubation theory. A calculation of the first diagram will give the same result for a free electron and a bound electron, while the second diagram will give different results because the intermediate states are different if an electron is bound than they are if it is free. We will therefore compute the amplitude from the second diagram.
\begin{align*}
	H_{\text{int}} &= -\frac{e}{mc} \vec{A} \cdot \vec{p} \\
	\vec{A} &= \frac{1}{\sqrt{V} } \sum_{\vec{k}, \alpha} \sqrt{\frac{\hbar c^2}{2\omega}}\hat{\epsilon}^{(\alpha)} \left( a_{\vec{k}, \alpha} e^{i\left( \vec{k} \cdot \vec{x} - \omega t \right) + a^{\dagger}_{\vec{k}, \alpha}} e^{-i \left( \vec{k} \cdot \vec{x} - \omega t \right) } \right) 
.\end{align*}
This contains a term causing absorption of a photon and another term causing emission. We separate the terms for absorption and emission and pull out the time dependence.
\begin{align*}
	\mathbf{H_{\text{int}}} &= \sum_{\vec{k}, \alpha} \left( H^{\text{abs}}_{\vec{k}, \alpha} e^{-i \omega t} + H^{\text{emit}}_{\vec{k}, \alpha} e^{ i \omega t} \right) \\
	\mathbf{H^{\text{abs}}} &= -\sqrt{\frac{\hbar e^2}{2m^2 \omega V}} a_{\vec{k}, \alpha} e^{i \vec{k} \cdot \vec{x}} \vec{p} \cdot \hat{\epsilon}^{(\alpha)} \\
	\mathbf{H^{\text{emit}}} &= -\sqrt{\frac{\hbar e^2}{2m^2 \omega V }} a^{\dagger}_{\vec{k}, \alpha} e^{i \vec{k} \cdot \vec{x}}\vec{p} \cdot \hat{\epsilon}^{(\alpha)}
.\end{align*}
The initial and final state is the same $\displaystyle \left| n \right>$, and second order pertubation theory will involve a sum over intermediate, and second order pertubation theory will involve a sum over intermediate atomic states, $\displaystyle \left| j \right>$ and photon states. We will use the matrix elementsof the interaction Hamiltonian between those states.
\begin{align*}
	\mathbf{H}_{\text{jn}} &= \left<j \left| \mathbf{H}^{\text{emit}}_{\vec{k}, \alpha} \right| n  \right> \\
	\mathbf{H}_{\text{nj}} &= \left<n \left| \mathbf{H}^{\text{abs}}_{\vec{k}, \alpha} \right| j  \right> \\
	\mathbf{H}_{\text{nj}} &= \mathbf{H}^{*}_{\text{jn}}
.\end{align*}
In this case, we want to use the equations for the state we are studying, $\displaystyle \psi_{n}$, and all intermediate states, $\displaystyle \psi_{j}$ plus a photon. Transitions can be made by emitting a photon from $\displaystyle \psi_{n}$ to an intermediate state and transitions can be made back to the state $\displaystyle \psi_{n}$ from any intermediate state. We neglect transitions from one intermediate state to another as they are higher order. (The diagram is emit a photon from $\displaystyle \psi_{n} then reabsorb it.$)
\\
The differential equations for the amplitudes are then.
\begin{align*}
	i\hbar \frac{dc_{j}}{dt} &= \sum_{\vec{k}, \alpha} \mathbf{H}_{jn} e^{i\omega t}c_{n}e^{i \omega_{nj} t} \\
	i\hbar \frac{dc_{n}}{dt} &= \sum_{\vec{k}, \alpha} \sum_{j} \mathbf{H}_{nj}e^{i \omega t} c_{j} e^{i \omega t}
.\end{align*}
In the equations for $\displaystyle c_{n}$, we explicitly account for the fact that an intermediate state can make a transition back to the initial state. Transitions through another intermediate state would be higher order and thus should be neglected. Note that the matrix elements for the transitions to and from the initial state are closely related. We also include the effect that the initial state can become depleted as intermediate states are populated by using $\displaystyle c_{n}$ (instead of 1) in the equation for $\displaystyle c_{j}$. Note also that all the photon states will make nonzero contributions to the sum.
\\
\\
Our task is to solve these coupled equations. Previously, we did this by integration, but needed the assumption that the amplitude to be in the initial state was 1. Since we are attempting to calculate an energy shift, let us make that assumption and plug it into the equations to verify the solution.
\begin{equation}
	c_{n} = e^{\frac{-i \Delta E_{n} t}{\hbar}}
\end{equation}
$\displaystyle \Delta E_{n}$ will be a complex number, the real part of which represents an energy shift, and the imaginary part of which represents the lifetime (and energy width) of the state.
\begin{align*}
	i\hbar \frac{dc_{j}}{dt} &= \sum_{\vec{k}, \alpha} \mathbf{H}_{jn} e^{ i \omega t} c_{n} e^{-i \omega_{nj} t} \\
	c_{n} &= e^{\frac{-i \Delta E_{n} t}{\hbar}} \\
	c_{j}\left( t \right)  &= \frac{1}{i\hbar} \sum_{\vec{k}, \alpha} \int_{0}^{t} dt' \mathbf{H}_{jn} e^{ i \omega t'} e^{\frac{-i \Delta E_{n}t'}{\hbar}} e^{-i \omega_{nj} t '} \\
	c_{j}\left( t \right) &= \frac{1}{i\hbar} \sum_{\vec{k}, \alpha} \int_{0}^{t} dt' \mathbf{H}_{jn} e^{ i \left( - \omega_{nj} - \Delta \omega_{n} + \omega \right)t' } \\
	c_{j}\left( t \right) &= \sum_{\vec{k}, \alpha} \mathbf{H}_{jn} \left[ \frac{e^{i \left( - \omega_{nj} - \Delta \omega_{n} + \omega \right)t' }}{\hbar \left( \omega_{nj} + \Delta \omega_{n} - \omega \right) } \right]^{t}_{0} \\
	c_{j}\left( t \right) &= \sum_{\vec{k}, \alpha} \mathbf{H}_{jn} \frac{e^{i \left( - \omega_{nj} - \Delta \omega_{n} + \omega \right)t -1 }}{hbar \left( \omega_{nj} + \Delta \omega_{n} - \omega \right) }
.\end{align*}
Substitute this back into the differential equation for $\displaystyle c_{n}$ to verify the solution and to find out what $\displaystyle \Delta E_{n}$ is. Note that the double sum over photons reduces to a single sum because we must absorb the same type of photon that was emitted. (We have not explicitly carried along the photon state for economy.)
\begin{align*}
	i\hbar \frac{dc_{n}}{dt} &= \sum_{\vec{k}, \alpha} \sum_{j} \mathbf{H}_{nj} e^{-i \omega t} c_{j} e^{i \omega_{nj} t} \\
	c_{j}\left( t \right)  &= \sum_{\vec{k}, \alpha} \mathbf{H}_{jn} \frac{e^{i \left( - \omega_{nj} - \Delta \omega_{n} + \omega \right) t } - 1}{\hbar \left( \omega_{nj} + \Delta \omega_{n} - \omega \right) } \\
	i\hbar \frac{dc_{n}}{dt} &= \Delta E_{n} e^{-i \Delta E_{n} t / \hbar} = \sum_{\vec{k}, \alpha} \sum_{j} \mathbf{H}_{nj} \mathbf{H}_{jn} e^{-i \omega t} e^{i \omega_{nj} t} \frac{e^{i \left( - \omega_{nj} - \Delta \omega_{n} + \omega \right) t } - 1}{\hbar \left( \omega_{nj} + \Delta \omega_{n} - \omega \right) } \\
	\Delta E_{n} &= \sum_{\vec{k}, \alpha} \sum_{j} \left| \mathbf{H}_{nj} \right|^2 e^{i \left( \omega_{nj} + \Delta \omega_{n} - \omega \right) t } \frac{e^{i \left( - \omega_{nj} - \Delta \omega_{n} + \omega \right)t }-1}{\hbar \left( \omega_{nj} + \Delta \omega_{n} - \omega \right) } \\
	\Delta E_{n} &= \sum_{\vec{k}, \alpha} \sum_{j} \left| \mathbf{H}_{nj} \right|^2 \frac{1 - e^{i \left( \omega_{nj} + \Delta \omega_{n} - \omega \right)t }}{\hbar \left( \omega_{nj} + \Delta \omega_{n} - \omega \right) }
.\end{align*}
Dince this is a calculation to order $\displaystyle e^{2}$ and the interaction Hamiltonian squared contains a factor of $\displaystyle e^2$ we should drop the $\displaystyle \Delta \omega_{n} = \Delta E_{n} / \hbar$ s from the right hand side of this equation.
\begin{equation}
    \Delta E_{n} = \sum_{\vec{k}, \alpha} \sum_{j} \left| \mathbf{H}_{nj} \right|^2 \frac{1 - e^{i \left( \omega_{nj} - \omega \right) t }}{\hbar \left( \omega_{nj} - \omega \right) }
\end{equation}

 

\end{document}
