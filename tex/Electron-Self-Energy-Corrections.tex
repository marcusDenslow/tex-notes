\documentclass{report}

\input{../templates/preamble}
\input{../templates/macros}
\input{../templates/letterfonts}

\title{\Huge{Electon Self Energy Corrections}}
\author{\Huge{Marcus Allen Denslow}}
\date{2026-01-09}

\begin{document}

\maketitle
\newpage% or \cleardoublepage
% \pdfbookmark[<level>]{<title>}{<dest>}
\pdfbookmark[section]{\contentsname}{toc}
\tableofcontents
\pagebreak

\chapter{Electron Self Energy Corrections}
If one calculates the energy of a point charge using classical electromagnitism, the result is infinate, yet as far as we know, the electron is point charge. One can calculate the energy needed to assemble an electron due, essentially, to the interaction of the electron with its own field. A uniform charge distribution with the classical radius of an electron, we have an energy $\displaystyle m_{e}c^2$. Experiments have probed the electron's charge distribution and found that it is consistent with a point charge down to distances much smallen than the classical radius. Beyond classical calculations, the self energy of the electron calculated in the quantum theory of Dirac is still infinate but the divergences are less severe. 
\\
At this pint we must take the unpleasant position (constant) infinate energy should just be subtracted when we consider the overall zero of energy (as we did for the field energy in the vacuum). Electrons exist and don't carry infinate amount of energy baggage so we just subtract off the infinate constant. Nevertheless, we will find that the electron's self energy may change when it is a bound state and we should account for this change in out energy level calculations. This calculation will also give us the opportunity to understand resonant behaviour in scattering.
\\
We can calculate the lowest order self energy corrections represented by the two Feynman diagrams below.

\begin{figure}[h!]
    \centering
    \includegraphics[width=0.6\textwidth]{/home/bolla/Pictures/latex.png}
\end{figure}

In these, a photon is emitted then reabsorbed. As we now know, both of these amplitutesare in order $\displaystyle e^2$. The first one comes from the $\displaystyle A^2$ term in which the number of photons changes by zero or two and the second comes from the $\displaystyle \vec{A} \cdot \vec{p}$ term in second order time dependent pertubation theory. A calculation of the first diagram will give the same result for a free electron and a bound electron, while the second diagram will give different results because the intermediate states are different if an electron is bound than they are if it is free. We will therefore compute the amplitude from the second diagram.
\begin{align*}
	H_{\text{int}} &= -\frac{e}{mc} \vec{A} \cdot \vec{p} \\
	\vec{A} &= \frac{1}{\sqrt{V} } \sum_{\vec{k}, \alpha} \sqrt{\frac{\hbar c^2}{2\omega}}\hat{\epsilon}^{(\alpha)} \left( a_{\vec{k}, \alpha} e^{i\left( \vec{k} \cdot \vec{x} - \omega t \right) + a^{\dagger}_{\vec{k}, \alpha}} e^{-i \left( \vec{k} \cdot \vec{x} - \omega t \right) } \right) 
.\end{align*}
This contains a term causing absorption of a photon and another term causing emission. We separate the terms for absorption and emission and pull out the time dependence.
\begin{align*}
	\mathbf{H_{\text{int}}} &= \sum_{\vec{k}, \alpha} \left( H^{\text{abs}}_{\vec{k}, \alpha} e^{-i \omega t} + H^{\text{emit}}_{\vec{k}, \alpha} e^{ i \omega t} \right) \\
	\mathbf{H^{\text{abs}}} &= -\sqrt{\frac{\hbar e^2}{2m^2 \omega V}} a_{\vec{k}, \alpha} e^{i \vec{k} \cdot \vec{x}} \vec{p} \cdot \hat{\epsilon}^{(\alpha)} \\
	\mathbf{H^{\text{emit}}} &= -\sqrt{\frac{\hbar e^2}{2m^2 \omega V }} a^{\dagger}_{\vec{k}, \alpha} e^{i \vec{k} \cdot \vec{x}}\vec{p} \cdot \hat{\epsilon}^{(\alpha)}
.\end{align*}
The initial and final state is the same $\displaystyle \left| n \right>$, and second order pertubation theory will involve a sum over intermediate, and second order pertubation theory will involve a sum over intermediate atomic states, $\displaystyle \left| j \right>$ and photon states. We will use the matrix elementsof the interaction Hamiltonian between those states.
\begin{align*}
	\mathbf{H}_{\text{jn}} &= \left<j \left| \mathbf{H}^{\text{emit}}_{\vec{k}, \alpha} \right| n  \right> \\
	\mathbf{H}_{\text{nj}} &= \left<n \left| \mathbf{H}^{\text{abs}}_{\vec{k}, \alpha} \right| j  \right> \\
	\mathbf{H}_{\text{nj}} &= \mathbf{H}^{*}_{\text{jn}}
.\end{align*}
In this case, we want to use the equations for the state we are studying, $\displaystyle \psi_{n}$, and all intermediate states, $\displaystyle \psi_{j}$ plus a photon. Transitions can be made by emitting a photon from $\displaystyle \psi_{n}$ to an intermediate state and transitions can be made back to the state $\displaystyle \psi_{n}$ from any intermediate state. We neglect transitions from one intermediate state to another as they are higher order. (The diagram is emit a photon from $\displaystyle \psi_{n} then reabsorb it.$)
\\
The differential equations for the amplitudes are then.
\begin{align*}
	i\hbar \frac{dc_{j}}{dt} &= \sum_{\vec{k}, \alpha} \mathbf{H}_{jn} e^{i\omega t}c_{n}e^{i \omega_{nj} t} \\
	i\hbar \frac{dc_{n}}{dt} &= \sum_{\vec{k}, \alpha} \sum_{j} \mathbf{H}_{nj}e^{i \omega t} c_{j} e^{i \omega t}
.\end{align*}
In the equations for $\displaystyle c_{n}$, we explicitly account for the fact that an intermediate state can make a transition back to the initial state. Transitions through another intermediate state would be higher order and thus should be neglected. Note that the matrix elements for the transitions to and from the initial state are closely related. We also include the effect that the initial state can become depleted as intermediate states are populated by using $\displaystyle c_{n}$ (instead of 1) in the equation for $\displaystyle c_{j}$. Note also that all the photon states will make nonzero contributions to the sum.
\\
\\
Our task is to solve these coupled equations. Previously, we did this by integration, but needed the assumption that the amplitude to be in the initial state was 1. Since we are attempting to calculate an energy shift, let us make that assumption and plug it into the equations to verify the solution.
\begin{equation}
	c_{n} = e^{\frac{-i \Delta E_{n} t}{\hbar}}
\end{equation}
$\displaystyle \Delta E_{n}$ will be a complex number, the real part of which represents an energy shift, and the imaginary part of which represents the lifetime (and energy width) of the state.
\begin{align*}
	i\hbar \frac{dc_{j}}{dt} &= \sum_{\vec{k}, \alpha} \mathbf{H}_{jn} e^{ i \omega t} c_{n} e^{-i \omega_{nj} t} \\
	c_{n} &= e^{\frac{-i \Delta E_{n} t}{\hbar}} \\
	c_{j}\left( t \right)  &= \frac{1}{i\hbar} \sum_{\vec{k}, \alpha} \int_{0}^{t} dt' \mathbf{H}_{jn} e^{ i \omega t'} e^{\frac{-i \Delta E_{n}t'}{\hbar}} e^{-i \omega_{nj} t '} \\
	c_{j}\left( t \right) &= \frac{1}{i\hbar} \sum_{\vec{k}, \alpha} \int_{0}^{t} dt' \mathbf{H}_{jn} e^{ i \left( - \omega_{nj} - \Delta \omega_{n} + \omega \right)t' } \\
	c_{j}\left( t \right) &= \sum_{\vec{k}, \alpha} \mathbf{H}_{jn} \left[ \frac{e^{i \left( - \omega_{nj} - \Delta \omega_{n} + \omega \right)t' }}{\hbar \left( \omega_{nj} + \Delta \omega_{n} - \omega \right) } \right]^{t}_{0} \\
	c_{j}\left( t \right) &= \sum_{\vec{k}, \alpha} \mathbf{H}_{jn} \frac{e^{i \left( - \omega_{nj} - \Delta \omega_{n} + \omega \right)t -1 }}{hbar \left( \omega_{nj} + \Delta \omega_{n} - \omega \right) }
.\end{align*}
Substitute this back into the differential equation for $\displaystyle c_{n}$ to verify the solution and to find out what $\displaystyle \Delta E_{n}$ is. Note that the double sum over photons reduces to a single sum because we must absorb the same type of photon that was emitted. (We have not explicitly carried along the photon state for economy.)
\begin{align*}
	i\hbar \frac{dc_{n}}{dt} &= \sum_{\vec{k}, \alpha} \sum_{j} \mathbf{H}_{nj} e^{-i \omega t} c_{j} e^{i \omega_{nj} t} \\
	c_{j}\left( t \right)  &= \sum_{\vec{k}, \alpha} \mathbf{H}_{jn} \frac{e^{i \left( - \omega_{nj} - \Delta \omega_{n} + \omega \right) t } - 1}{\hbar \left( \omega_{nj} + \Delta \omega_{n} - \omega \right) } \\
	i\hbar \frac{dc_{n}}{dt} &= \Delta E_{n} e^{-i \Delta E_{n} t / \hbar} = \sum_{\vec{k}, \alpha} \sum_{j} \mathbf{H}_{nj} \mathbf{H}_{jn} e^{-i \omega t} e^{i \omega_{nj} t} \frac{e^{i \left( - \omega_{nj} - \Delta \omega_{n} + \omega \right) t } - 1}{\hbar \left( \omega_{nj} + \Delta \omega_{n} - \omega \right) } \\
	\Delta E_{n} &= \sum_{\vec{k}, \alpha} \sum_{j} \left| \mathbf{H}_{nj} \right|^2 e^{i \left( \omega_{nj} + \Delta \omega_{n} - \omega \right) t } \frac{e^{i \left( - \omega_{nj} - \Delta \omega_{n} + \omega \right)t }-1}{\hbar \left( \omega_{nj} + \Delta \omega_{n} - \omega \right) } \\
	\Delta E_{n} &= \sum_{\vec{k}, \alpha} \sum_{j} \left| \mathbf{H}_{nj} \right|^2 \frac{1 - e^{i \left( \omega_{nj} + \Delta \omega_{n} - \omega \right)t }}{\hbar \left( \omega_{nj} + \Delta \omega_{n} - \omega \right) }
.\end{align*}
Dince this is a calculation to order $\displaystyle e^{2}$ and the interaction Hamiltonian squared contains a factor of $\displaystyle e^2$ we should drop the $\displaystyle \Delta \omega_{n} = \Delta E_{n} / \hbar$ s from the right hand side of this equation.
\begin{equation}
    \Delta E_{n} = \sum_{\vec{k}, \alpha} \sum_{j} \left| \mathbf{H}_{nj} \right|^2 \frac{1 - e^{i \left( \omega_{nj} - \omega \right) t }}{\hbar \left( \omega_{nj} - \omega \right) }
\end{equation}
We have a solution to the coupled differential equations to order $\displaystyle e^2$. We should let $t \to \infty$ since the self energy is not a time dependent thing, however, the result oscillates as a function of time. This has been the case for many of out important delta functions, like the dot product of states with definite momentum. Let us analyze this self energy expression for large time.
\\
We have something of the form.
\begin{equation}
    -i \int_{0}^{t} e^{ixt'} \, dt' = \frac{1 - e^{ixt}}{x}
\end{equation}
If we think of $\displaystyle x$ as a complex number, our integral goes along the real axis. In the upper hald plane, just above the real axis, $\displaystyle x \to x + i \epsilon$, the function goes to zero at infinity. In the lower half plane it blows up at infinity and on the axis, it's not well defined. We will calculate our result in the upper half plane and take the limit as we approach the real axis.
\begin{equation}
	\lim_{t \to \infty} \frac{1 - e^{ixt}}{x} = \lim_{\epsilon \to 0+} i \int_{0}^{\infty} e^{ixt'} \, dt' = \lim_{\epsilon \to 0+} \frac{1}{x + i \epsilon} = \lim_{\epsilon \to 0+} \left[ \frac{x}{x^2 + \epsilon^2} - \frac{i \epsilon}{x^2 + \epsilon^2} \right]
\end{equation}
This is well behaved everywhere expect at $\displaystyle x = 0$. The second term goes to $\displaystyle -\infty$ there. A little further analasys could show that the second term is a delta function.
\begin{equation}
    \lim_{t \to \infty} \frac{1 - e^{ixt}}{x} = \frac{1}{x} - i\pi \delta \left( x \right) 
\end{equation}
Recalling that $\displaystyle c_{n}e^{-i E_{n} r / \hbar} = e^{\frac{-i \Delta E_{n}t}{\hbar}} e^{-i E_{n} t / \hbar} = e^{-i \left( E_{n} + \Delta E_{n} \right)t / \hbar }$ , the real part of $\Delta E_{n}$ corresponds to an energy shift in the state $\displaystyle \left|n \right>$ and the imaginary part corresponds to a width.
\begin{align*}
	\Re \left( \Delta E_{n} \right) &= \sum_{\vec{k}, \alpha} \sum_{j} \frac{\left| \mathbf{H}_{nj} \right|^2 }{\hbar \left( \omega_{nj} - \omega \right) } \\
	\Im \left( \Delta E_{n} \right) &= -\pi \sum_{\vec{k}, \alpha} \sum_{j} \frac{\left| \mathbf{H}_{nj} \right|^2 }{\hbar} \delta \left( \omega_{nj} - \omega \right) = -\pi \sum_{\vec{k}, \alpha} \sum_{j}  \left| \mathbf{H}_{nj} \right|^2 \delta \left( E_{n} - E_{j} - \hbar \omega \right) 
.\end{align*}
All photon energies contribute to the real part. Only photons that satisfy the delta function constraint to the imaginary part. Moreover, there will only be an imaginary part if there is a lower energy state into which the state in question can decay. We can relate this width o those we previously calculated.
\begin{equation}
    -\frac{2}{\hbar}\Im \left( \Delta E_{n} \right) = \sum_{\vec{k}, \alpha} \sum_{j} \frac{2\pi \left| \mathbf{H}_{nj} \right|^2 }{\hbar} \delta \left( E_{n} - E_{j} - \hbar \omega \right) 
\end{equation}
The time dependence of the wavefunction for the state $\displaystyle n$ is modified by the self energy correction.
\begin{equation}
    \psi_{n} \left( \vec{x}, t \right) = \psi_{n} \left( \vec{x} \right) e^{-i \left( E_{n} + \Re \left( \Delta E_{n} \right)  \right)t / \hbar  } e^{\frac{- \Gamma_{n} t}{2}}
\end{equation}
This also gives us the exponential decay behaviour that we expect, keeping resonant scattering cross sections from going to infinity. So, the width just goes into the time dependence as expected and we don't have to worry about it anymore. We can now concentrate on the energy shift due to the real part of $\displaystyle \Delta E_{n}$.
\begin{align*}
	\Delta E_{n} \equiv \Re \left( \Delta E_{n} \right) &= \sum_{\vec{k}, \alpha} \sum_{j} \frac{\left| \mathbf{H}_{nj} \right|^2 }{\hbar \left( \omega_{nj} - \omega \right) } \\
	\mathbf{H}_{nj} &= \left<n \left| \mathbf{H}^{\text{abs}}_{\vec{k}, \alpha} \right|j  \right> \\
	\mathbf{H}^{\text{abs}} &= -\sqrt{\frac{\hbar e^2}{2m^2 \omega V}} E^{i \vec{k} \cdot \vec{x}} \vec{p} \cdot \hat{\epsilon}^{(\alpha)} \\
	\Delta E_{n} &= \frac{\hbar e^2}{2m^2 V}\sum_{\vec{k}, \alpha} \sum_{j} \frac{\left| \left<n \left| e^{i \vec{k} \cdot \vec{x}} \vec{p} \cdot \hat{\epsilon}^{(\alpha)} \right|j  \right> \right|^2 }{\hbar \omega \left( \omega_{nj} - \omega \right) } \\
	&= \frac{e^2}{2m^2V} \int \frac{V d^3 k}{\left( 2\pi \right)^3 } \, \sum_{\alpha} \sum_{j}  \frac{\left| \left<n \left| e^{i \vec{k} \cdot \vec{x}} \vec{p} \cdot \hat{\epsilon}^{(\alpha)} \right|j  \right> \right|^2 }{ \omega \left( \omega_{nj}- \omega \right) } \\
	&= \frac{e^2}{\left( 2\pi \right)^3 2m^2 } \sum_{\alpha} \sum_{j}  \int d \Omega \,\frac{k^2 dk}{\omega} \frac{\left| \left<n \left| e^{i \vec{k} \cdot \vec{x}}\vec{p} \cdot \hat{\epsilon}^{(\alpha)} \right|j  \right> \right|^2 }{\left( \omega_{nj} - \omega \right) } \\
	&= \frac{e^2}{\left( 2\pi \right)^3 2m^2 c^3 } \sum_{j} \sum_{\alpha} \int d \Omega \int \frac{\omega \left| \left<n \left| e^{i \vec{k} \cdot \vec{x}}\vec{p} \right|j  \right> \cdot \hat{\epsilon}^{(\alpha)} \right|^2 }{\left( \omega_{nj} - \omega \right) } \, d\omega
.\end{align*}
In our calculation of the total decay rate summed over polarization and integrated over photon direction we computed the cosine of the angle between each polarization vector and the (vector) matrix element. Summing these two and integrating over photon direction we got a factor of $\displaystyle \frac{8\pi}{3}$ and the polarization is eliminated from the matrix element. The same calculation applies here.
\begin{align*}
	\Delta E_{n} &= \frac{e^2}{\left( 2\pi \right)^3 2m^2 c^3 } \sum_{j} \frac{8\pi}{3}\int \frac{\omega \left| \left<n \left| e^{i \vec{k} \cdot \vec{x}} \vec{p} \right|j  \right> \right|^2 }{\left( \omega_{nj} - \omega \right) } \, d\omega \\
	&= \frac{e^2}{6\pi^2m^2 c^3} \sum_{j} \int \frac{\omega \left| \left<n \left| e^{i \vec{k} \cdot \vec{x}} \vec{p} \right|j  \right> \right|^2 }{\left( \omega_{nj} - \omega \right) } \, d\omega \\
	&= \frac{2 \alpha \hbar}{3\pi m^2 c^2} \sum_{j} \int \frac{\omega \left| \left<n \left| e^{i \vec{k} \cdot \vec{x}} \vec{p} \right|j  \right> \right|^2 }{\left( \omega_{nj} - \omega \right) } \, d\omega
.\end{align*}

Note that we wish to use the electric dipole approximation which is not valid for large $k = \frac{\omega}{c}$. It is valid up to about 2000 eV so we wish to cut off the calculation around there. While this calculation clearly diverges, things are less clear here because of the eventually rapid oscillation of the $\displaystyle e^{i\vec{k}\cdot \vec{x}}$ term in the integrand as the E1 approximation fails. Nevertheless, the largest differences in corrections between free electrons and bound electrons occur in the region in which the E1 approximation is valid. For now we will just use it and assume the cut-off is low enough.
\\
It is the difference between the bound electron's self energy and that for a free electron in which we are interested. Therefore, we will start with the free electron with a definite momentum $\displaystyle \vec{p}$. The normalized wave function for the free electron is $\displaystyle \frac{1}{\sqrt{V} } e^{i \vec{p} \cdot \vec{x}} / \hbar$.
\begin{align*}
	\Delta E_{\text{free}} &= \frac{2 \alpha \hbar}{3 \pi m^2 c^2 V^2} \sum_{\vec{p}'} \int \frac{\omega}{\left( \omega_{nj} - \omega \right) } \left| \int e^{-i \vec{p} \cdot \vec{x} / \hbar} \vec{p} e^{i \vec{p}' \cdot \vec{x} / \hbar} d^3  \right|^2 d\omega \\
	&= \frac{2 \alpha \hbar}{3\pi m^2 c^2 V^2} \left| \vec{p} \right|^2 \sum_{\vec{p}} \int  \frac{\omega}{\left( \omega_{nj} - \omega \right) } \left| \int e^{i \left( \vec{p}' / \hbar - \vec{p} \cdot \vec{x} / \hbar \right) } d^3 x \right|^2 d\omega \\
	&= \frac{2 \alpha \hbar}{3\pi m^2 c^2 V^2} \left| \vec{p} \right|^2 \sum_{\vec{p}'} \int \frac{\omega}{\left( \omega_{nj} - \omega \right) } \left| V \delta_{\vec{p}', \vec{p}} \right|^2 d\omega \\
	&= \frac{2 \alpha \hbar}{3\pi m^2 c^2} \left| \vec{p} \right|^2 \int_{0}^{\infty} \frac{\omega}{\left( \omega_{nj} - \omega \right) } \, d\omega \\
	&= \frac{2 \alpha \hbar}{3\pi m^2 c^2} \left| \vec{p} \right|^2 \int_{0}^{\infty} \frac{\omega}{\left( \omega_{nj} - \omega \right) } d\omega \to -\infty
.\end{align*}

It's easy to see that this will go to negative infinity if the limit on the integral is infinite. It is quite reasonable to cut off the integral at some enrgy beyond which the theory we are using is invalid. Since we are still using non-relativistic quantum mechanics, the cut-off should have $\displaystyle \hbar \omega \ll mc^2$. For the E1 approximation, it should be $\displaystyle \hbar \omega \ll 2\pi \hbar c / 1 = 10ke V$. We will approximate $\displaystyle \frac{\omega}{\left( \omega_{nj} - \omega \right) } \approx -1$ since the integral is just giving us a number and we are not interested in high accuracy here. We will be more interested in accuracy in the next section when we compute the difference between free electron and bound electron self energy corrections.
\begin{align*}
	\Delta E_{free} &= \frac{2 \alpha \hbar}{3\pi m^2 c^2} \left| \vec{p} \right|^2 \int_{0}^{E_{cut} - off / \hbar}\frac{\omega}{\left( \omega_{nj} - \omega \right) }  \, d\omega \\
    &= - \frac{2 \alpha \hbar }{3\pi m^2 c^2} \left| \vec{p} \right|^2 \int_{0}^{E_{cut-off} / \hbar} \frac{\omega}{\left( \omega_{nj} - \omega \right) }  \, d\omega \\
	&= - \frac{2 \alpha \hbar }{3\pi m^2 c^2} \left| \vec{p} \right|^2 E_{cut-off} / \hbar \\
	&= - \frac{2 \alpha}{3\pi m^2 c^2} \left| \vec{p} \right|^2 E_{cut-off} \\
	&= -C \left| \vec{p} \right|^2
. \end{align*}
If we were hoping for little dependence on the cut-off we should be disappointed. This self energy calculated is linear in the cut-off.
\\
\\
For a non-relativisitc free electron the energy $\displaystyle \frac{p^2}{2m}$ decreases as the mass of the electron increases, so the negative Yours sincerely,

Marcus Denslow corresponds to a positive shift in the electron's mass, and hence an increase in the real energy of the electron. Later, we will think of this as a renormalization of the electron's mass. The electron starts off with some bare mass. The self-energy due to the interaction of the electron's charge with its own radiation field increases the mass to what is observed.
\\
\\
Note that the correction to the energy is a constant times $\displaystyle p^2$, like the non-relativistic formula for the kinetic energy.
\begin{align*}
    C &= \frac{2 \alpha}{3\pi m^2 c^2} E_{cut-off} \\
	\frac{p^2}{2m_{obs}} &= \frac{p^2}{2m_{bare}} - Cp^2 \\
	\frac{1}{m_{obs}} &= \frac{1}{m_{bare}} - 2C \\
	m_{obs} &= \frac{m_{bare}}{1 - 2C m_{bare}} \approx \left( 1 + 2C m_{bare} \right) m_{bare} \approx \left( 1 + 2Cm \right) m_{bare} \\
			&= \left( 1 + \frac{4 \alpha E_{cut-off}}{3\pi mc^2} \right) m_{bare}
.\end{align*}
If we cut off the integral at $\displaystyle m_{e}c^2$, the correction to the mass is only about 0.3 percent, but if we don't cut off, its infinite. It makes no sense to trust our non-relativistic calculation up to infinite energy, so we must proceed with the cut-off integral.
\\
If we use the Dirac theory, then we will be justified to move the cut-off up to very high energy. It turns out that the relativistic correction diverges logarithmically (instead of linearly) and the difference between bound and free electrons is finite relativistically (while it diverges logarithmically for out non-relativistic calculation).
\\
Note that the self-energy of the free electron depends on the momentum of the electron, so we cannot simply subtract it from out bound state calculation. (What $\displaystyle p^2$ would we choose?) Rather we must account for the mass renormalization. We used the observed electron mass in the calculation of the Hydrogen bound state energies. In doing so, we have already included some of the self energy correction and we must not double correct. This is the subtraction we must make.
\\

It's hard to keep all the minus signs straight in this calculation, particularly if we consider the bound and continuum electron states separately. The free particle correction to the electron mass is positive. Because we ignore the rest energy of the electron in our non-relativistic calculations. This makes a negative energy correction to both the bound  $\displaystyle E = \frac{1}{2n^2} \alpha^2 mc^2$ and the continuum $\displaystyle E \approx \frac{p^2}{2m}$. Bound states and continuum states have the same fractional change in the energy. We need to add back in a positive term in $\displaystyle \Delta E_{n}$ to avoid double counting of the self-energy correction. Since the bound state and continuum state terms have the same fractional, it is convecient to just use $\displaystyle \frac{p^2}{2m}$ for all the corrections.
\begin{align*}
	\frac{p^2}{2m_{obs}} &= \frac{p^2}{2m_{bare}} - C p^2 \\
	\Delta E^{(obs)}_{n} &= \Delta E_{n} + C \left< n \left| p^2 \right|n \right> = \Delta E_{n} + \frac{2 \alpha}{3\pi m^2 c^2} E_{cut-off} \left<n \left| p^2 \right|n  \right>
.\end{align*}

Because we are correcting for the mass used to calculate the base enrgy of the state $\left| n \right>$, our corrections is written in terms of the electron's momentum in that state.


\chapter{The Lamb Shift}
In 1947, Willis E. Lamb and R. C Retherford used microwave techniques to determine the splitting between the $\displaystyle 2S_{\frac{1}{2}}$ and $\displaystyle 2P_{\frac{1}{2}}$ states in Hydrogen to have a frequency of 1.06 GHz, (a wavelenght of about 30 cm). (The shift is now accurately measured to be 1057.864 MHz.) This is abour the same size as the hyperfine splitting of the ground state.
\\
\\
The technique used was quite interesting. They made a beam of Hydrogen atoms in the $\displaystyle 2S_{\frac{1}{2}}$ state, which has a very long lifetime because of selection rules. Microwave radiation with a (fixed) frequency of 2395 MHz was used to cause transitions to the $\displaystyle 2P_{\frac{3}{2}}$ state and a magnetic field was adjusted to shift the enrgy of the states until the rate was largest. The decay of the $\displaystyle 2P_{\frac{3}{2}}$ state to the ground state was observed to determine the transition rate. From this, they were able to deduce the shift between the $\displaystyle 2S_{\frac{1}{2}}$ and $\displaystyle 2P_{\frac{1}{2}}$ states. Hans Bethe used non-relativistic quantum mechanics to calculate the self-energy correction to account for this observation.
\\
\\
We now can compute the correction the same way he did.
\begin{align*}
	\Delta E^{(obs)}_{n} &= \Delta E_{n} + C \left<n \left| p^2 \right|n  \right> = \Delta E_{n} + \frac{2 \alpha}{3\pi m^2 c^2}E_{cut-off} \left<n \left| p^2 \right|n  \right> \\
	\Delta E_{n} &= \frac{2 \alpha \hbar}{3\pi m^2 c^2} \sum_{j}  \int \frac{\omega \left| \left<n \left| e^{i \vec{k} \cdot \vec{x}}\vec{p} \right|j  \right> \right|^2}{\left( \omega_{nj} - \omega \right) }  \, d\omega \\
	\Delta E^{(obs)}_{n} &= \frac{2 \alpha \hbar}{3\pi m^2 c^2} \int_{0}^{\omega_{cut-off}} \left( \frac{\omega}{\left( \omega_{nj} - \omega \right) } \left| \left<n \left| e^{i \vec{k} \cdot \vec{p}} \vec{p} \right|j  \right> \right|^2 + \left<n \left| p^2 \right|n  \right>  \right)   \, d\omega \\
	&= \frac{2 \alpha \hbar}{3\pi m^2 c^2} \int_{0}^{\omega_{cut-off}} \sum_{j} \left( \frac{\omega}{\left( \omega_{nj} - \omega \right) } \left| \left<n \left| e^{i \vec{k} \cdot \vec{x}} \vec{p} \right|j  \right> \right|^2 + \left<n \left| \vec{p} \right|j  \right> \left<j \left| \vec{p} \right| n \right> \right)    \, d\omega \\
	&= \frac{2 \alpha \hbar}{3\pi m^2 c^2} \int_{0}^{\omega_{cut-off}} \sum_{j} \left( \frac{\omega}{\left( \omega_{nj} - \omega \right) } \left| \left<n \left| e^{i \vec{k} \cdot \vec{x}} \vec{p} \right|j  \right> \right|^2 + \left| \left<n \left| \vec{p} \right|j  \right> \right|^2   \right)    \, d\omega
.\end{align*}
It is now necessary to discuss approximate needed to complete this calculation. In particular, the electric dipole approximationwill be of great help, however, it is certainly not warranted for large photon energies. For a good E1 approximation we need $\displaystyle E_{\gamma} << 1973$ eV. On the other hand, we want the cut-off for the calculation to be of order $\displaystyle \omega_{cut-off} \approx mc^2 / \hbar$. We will use the E1 approximation and the high cut-off, as Bethe did, to get the right answer. At the end, the result from a relativistic calculation can be tacked on to show why it turns out to be the right answer. (We aren't aiming for the worlds best calculation anyway.)
\begin{align*}
	\Delta E^{(obs)}_{n} &= \frac{2 \alpha \hbar}{3\pi m^2 c^2} \int_{0}^{\omega_{cut-off}} \sum_{j} \left( \frac{\omega}{\left( \omega_{nj} - \omega \right) } \left| \left<n \left| \vec{p} \right|j  \right> \right|^2 + \left| \left<n \left| \vec{p} \right|j  \right> \right|^2   \right)    \, d\omega \\
	&= \frac{2 \alpha \hbar}{3\pi m^2 c^2} \int_{0}^{\omega_{cut-off}} \sum_{j} \frac{\omega + \left( \omega_{nj} - \omega \right) }{\left( \omega_{nj} - \omega \right) } \left| \left<n \left| \vec{p} \right|j  \right> \right|^2   \, d\omega \\
	&= - \frac{2 \alpha \hbar}{3\pi m^2 c^2} \sum_{j} \int_{0}^{\omega_{cut-off}} \frac{\omega_{nj}}{\omega - \omega_{nj}} \left| \left<n \left| \vec{p} \right|j   \right> \right|^2   \, d\omega \\
	&= - \frac{2 \alpha \hbar}{3\pi m^2 c^2} \sum_{j}  \omega_{nj} \left[ \log \left( \omega - \omega_{nj} \right)  \right]^{\omega_{cut-off}}_{0} \left| \left<n \left| \vec{p} \right|j  \right> \right|^2 \\
	&= \frac{2 \alpha \hbar}{3\pi m^2 c^2} \sum_{j} \omega_{nj} \left[ \log \left( \left| \omega_{nj} \right|  \right) - \log \left( \omega_{cut-off} \right)   \right] \left| \left<n \left| \vec{p} \right|j  \right> \right|^2 \\
	&\approx \frac{2 \alpha \hbar}{3\pi m^2 c^2} \sum_{j}  \omega_{nj} \left[ \log \log \left( \left| \omega_{nj} \right|  \right) - \log \left( \omega_{cut-off} \right)   \right] \left| \left<n \left| \vec{p} \right|j  \right> \right|^2 \\
	&= \frac{2 \alpha \hbar}{3\pi m^2 c^2} \sum_{j}  \omega_{nj} \log \left( \frac{\left| \omega_{nj} \right| }{\omega_{cut-off}} \right)  \left| \left<n \left| \vec{p} \right|j  \right> \right|^2
.\end{align*}

The long term varies more slowly than the rest of the terms in the sum. We can approximate it by an average. Bethe used numerical calculations to determine that the effective average of $\displaystyle \hbar \omega_{nj}$ is $\displaystyle 8.9 \alpha^2 mc^2$. We will do the same and pull the log term out as a constant.
\begin{equation}
    \Delta E^{(obs)}_{n} = \frac{2 \alpha \hbar}{3\pi m^2 c^2} \log \left( \frac{\left| \overline{\omega} \right| }{\omega_{cut-off}} \right) \sum_{j} \omega_{nj} \left| \left<n \left| \vec{p} \right|j  \right> \right|^2
\end{equation}
This sum can now be reduced further to a simple expression proportional to the $\displaystyle \left| \psi_{n} \left( 0 \right)  \right|^2 $ using a typical clever quantum mechanics calculation. The basic Hamitonian for the Hydrogen atom is $\displaystyle \mathbf{H}_{0} = \frac{p^2}{2m} + V \left( r \right) $.
\begin{align*}
	\left[ \vec{p}, \mathbf{H}_{0} \right] &= \left[ \vec{p}, V \right] = \frac{\hbar}{i} \vec{\nabla}V \\
	\left<j \left| \left[ \vec{p}, \mathbf{H}_{0} \right]  \right|n  \right> &= \frac{\hbar}{i} \left<j \left| \vec{\nabla} V \right| n \right> \\
	\sum_{j} \left<n \left| \vec{p} \right|j  \right> \left<j \left| \left[ \vec{p}, \mathbf{H}_{0} \right]  \right| n \right> &= \frac{\hbar}{i} \sum_{j}  \left<n \left| \vec{p} \right|j  \right> \cdot \left<j \left| \vec{\nabla V} \right|n  \right> \\
	\sum_{j}  \left( E_{i} - E_{n} \right) \left<n \left| \vec{p} \right|j  \right> \left<j \left| \vec{p} \right|n  \right> &= \frac{\hbar}{i} \sum_{j}  \left<n \left| \vec{p} \right|j  \right> \left<j \left| \vec{\nabla V} \right|n  \right>
.\end{align*}
This must be a real number so we may use its complex conjugate.
\begin{align*}
	\left( \sum_{j} \left( E_{i} - E_{n} \right) \left<n \left| \vec{p} \right|j  \right> \left<j \left| \vec{p} \right|n  \right>   \right) ^{*} &= \sum_{j} \left( E_{i} - E_{n} \right)  \left<n \left| \vec{p} \right|j  \right> \left<j \left| \vec{p} \right|n  \right> \\
	&= - \frac{\hbar}{i} \sum_{j}  \left<n \left| \vec{\nabla} V \right|j  \right>\left<j \left| \vec{p} \right|n  \right> \\
	\sum_{j}  \left( E_{i} - E_{n} \right) \left<n \left| \vec{p} \right|j  \right> \left<j \left| \vec{p} \right|n  \right> &= \frac{\hbar}{2i} \left[ \sum_{j} \left<n \left| \vec{p} \right|j  \right> \left<j \left| \vec{\nabla}V \right|n  \right> - \left<n \left| \vec{\nabla}V \right|j  \right> \left<j \left| \vec{p} \right|n  \right>  \right]  \\
	&= \frac{\hbar}{2i} \left<n \left| \left[ \vec{p}, \vec{\nabla} V \right]  \right|n  \right> \\
	&= - \frac{\hbar^2}{2} \left<n \left| \nabla^2 V \right|n  \right> \\
	&= - \frac{\hbar^2}{2} \left<n \left| e^2 \delta^3 \left( \vec{x} \right)  \right|n  \right> \\
	&= - \frac{e^2 \hbar^2}{2} \left| \psi_{n} \left( 0 \right)  \right|^2
.\end{align*}
Only the s states will have a non-vanishing probability to be at the origin with $\displaystyle \left| \psi_{n00} \left( 0 \right)  \right|^2 = \frac{1}{\pi n^3 a^3_{0}} $ and $\displaystyle a_0 = \frac{\hbar}{\alpha m c}$. Therefore, only the s states will shift in energy appreciably. The shift will be.
\begin{align*}
	\Delta E^{(obs)}_{n} &= - \frac{2 \alpha \hbar}{3\pi m^2 c^2} \log \left( \frac{\left| \overline{\omega}_{nj} \right| }{\omega_{cut-off}} \right) \frac{e^2 \hbar}{2} \frac{1}{\pi n^3} \left( \frac{\alpha m c}{\hbar} \right)^3 \\
	&= \frac{\alpha^{4} e^2 mc}{3\pi^2 \hbar n^3} \log \left( \frac{\omega_{cut-off}}{\left| \overline{\omega}_{nj} \right| } \right) \\
	&= \frac{4 \alpha^{5} mc^2}{6\pi} \log \left( \frac{mc^2}{8.9 \alpha^2 mc^2} \right) \\
	\Delta E^{(obs)}_{2s} &= \frac{\alpha^{5}mc^2}{6\pi} \log \left( \frac{mc^2}{8.9 \alpha^2 mc^2} \right) \\
	\nu &= \frac{\Delta E^{(obs)}_{2s}}{2\pi \hbar} = \frac{\alpha^{5} mc^2 c}{12\pi^2 \hbar c} \log \left( \frac{1}{8.9 \alpha^2} \right)  = 1.041 GHz
.\end{align*}
This agrees far too well with the measurement, considering the approximations made and the dependence on the cut-off. There is, however, justification in the relativistic calculation. Typically, the full calculation was made by using this non-relativistic approach up to some energy of the order of $\displaystyle \alpha m c^2$, and using the relativistic calculation above that. The relativistic free electron self-energy correction diverges only logarithmically and a very high cutoff can be used without a problem. The mass of the electron is renormalized as above. The Lamb shift does not depend on the cutoff and hence it is well calculated. We only need the non-relativistic part of the calculation up to photon energies for which the E1 approximations is OK. The relativistic part of the calculation down to $\displaystyle \omega_{min}$ yields.
\begin{equation}
    \Delta E_{n} = \frac{4 \alpha^{5}}{3\pi n^3} \left( \log \left( \frac{mc^2}{2 \hbar \omega_{min}} \right) + \frac{11}{24} - \frac{1}{5}  \right) mc^2
\end{equation}

The non-relativistic calculation gave.
\begin{equation}
    \Delta E_{n} = \frac{4 \alpha^{5}}{3\pi n^3} \log \left( \frac{\omega_{min}}{\left| \overline{\omega}_{nj} \right| } \right) mc^2
\end{equation}
So the sum of the two gives.
\begin{equation}
    \Delta E^{(obs)}_{n} = \frac{4 \alpha^{5}}{3\pi n^3} \left( \log \left( \frac{mc^2}{2 \hbar \overline{\omega}_{nj}} \right) + \frac{11}{24} - \frac{1}{5}  \right) mc^2
\end{equation}



\end{document}
