\documentclass{report}

\input{../templates/preamble}
\input{../templates/macros}
\input{../templates/letterfonts}

\title{\Huge{Quantum Theory of Radiation}}
\author{\Huge{Marcus Allen Denslow}}
\date{2026-01-18}

\begin{document}

\maketitle
\newpage% or \cleardoublepage
% \pdfbookmark[<level>]{<title>}{<dest>}
\pdfbookmark[section]{\contentsname}{toc}
\tableofcontents
\pagebreak

\chapter{Quantum Theory of Radiation}
\section{Transverse and Longitudinal Fields}
In non-relativistic Quantum Mechanics, the static Electric field is represented by a scalar potential, magnetic fields by the vector potential, and the radiation field also through the vector potential. It will be convenient to keep this separation between the large static atomic Electric field and the radiation fields, however, the equations we have contain the four-vector $\displaystyle A_{\mu}$ with all the fields mixed. When the quantize the field, all E and B fields as well as electromagnetic waves will be made up of photons. It is useful to be able to separate the E fields due to fixed charges from the EM radiation from moving charges. This separation is not Lorentz invariant, but it is still useful. Enrico Fermi showed, in 1930, that $\displaystyle A_{\|}$ together with $\displaystyle A_0$ give rise to Coulomb interactions between particles, whereas $\displaystyle A_{\bot}$ gives rise to the EM radiation from moving charges. With this separation, we can maintain the form of our non-relativistic Hamiltonian.
\\
\\
\begin{equation}
	\boxed{\mathbf{H} = \sum_{j} \frac{1}{2m_{j}} \left( \vec{p} - \frac{e}{c} \vec{A}_{\bot} \left( \vec{x}_{j} \right)  \right)^2 + \sum_{i > j} \frac{e_{i} e_{j}}{4\pi \left|  \right|\vec{x}_{i} - \vec{x}_{j}} + \mathbf{H_{rad}} }
\end{equation}
\\
\\
Where $\displaystyle \mathbf{H_{rad}}$ is purely the Hamiltonian of the radiation (containing only $\displaystyle \vec{A}_{\bot}$) and $\displaystyle \vec{A}_{\bot}$ is the part of the vector potential which satisfies $\displaystyle \nabla \cdot \vec{A}_{\bot} = 0$. Note that $\displaystyle \vec{A}\| \text{ and } A_4$ appear nowhere in the Hamiltonian. Instead, we have the Coulomb potential. This separation allows us to continue with our standard Hydrogen solution and just add radiation. We will not derive this result.
\\
\\
In a region in which there are no source terms,
\\
\\
\begin{equation}
    j_{\mu} = 0
\end{equation}
\\
\\
we can make a gauge transformation which eliminates $\displaystyle A_0$ by choosing $\displaystyle \Lambda$ such that
\\
\\
\begin{equation}
	\frac{1}{c} \frac{ \partial \Lambda }{ \partial t }  = A_0
\end{equation}
\\
\\
Since the fourth component of $\displaystyle A_{\mu}$ is now eliminated, the Lorentz condition now implies that
\\
\\
\begin{equation}
    \vec{\nabla} \cdot \vec{A} = 0
\end{equation}
\\
\\
Again, making one component of a 4-vector zero is not a Lorentz invariant way of working. We have to redo the gauge transformation if we move to another frame.
\\
\\
If $\displaystyle j_{\mu} \neq 0$, then we cannot eliminate $\displaystyle A_0$, since $\displaystyle \boxed A_0 = \frac{j_0}{c}$ and we are only allowed to make gauge transformation for which $\displaystyle \boxed \Lambda = 0$. In this case we must separate the vector potential into the transverse and longitudinal parts, with
\\
\\
\begin{align*}
	\vec{A} &= \vec{A}_{\bot} + \vec{A}_{\|}\\
	\vec{\nabla} \cdot \vec{A}_{\bot} &= 0 \\
	\vec{\nabla} \times \vec{A}_{\|} &= 0
.\end{align*}
\\
\\
We will now study the radiation field in a region with no sources so that $\displaystyle \vec{\nabla} \cdot \vec{A} = 0 $. We will use the equations
\\
\\
\begin{align*}
	\vec{B} &= \vec{\nabla} \times \vec{A} \\
	\vec{E} &= -\frac{1}{c} \frac{ \partial \vec{A} }{ \partial t } \\
	\nabla^2 \vec{A} - \frac{1}{c^2} \frac{ \partial \vec{A} }{ \partial t } &= 0
.\end{align*}


\section{Fourier Decomposition of Radiation Oscillators}
Our goal is to write the Hamiltonian for the radiation field in terms of a sum of harmonic oscillators Hamiltonians. The first step is to write the radiation field in a simple way as possible, as a sum of harmonic components. We will work in a cubic volume $\displaystyle V = L^3$ and apply periodic boundary conditions on our electromagnetic waves. We also assume for now that there are no sources inside the region so that we can make a gauge transformation to make $\displaystyle A_0 = 0$ and hence $\displaystyle \vec{\nabla} \cdot \vec{A} = 0$ . We decompose the field into its Fourier components at $\displaystyle t = 0$
\\
\\
\begin{equation}
    \vec{A} \left( \vec{x}, t = 0 \right)  = \frac{1}{ \sqrt{V} } \sum_{k} \sum_{\alpha = 1}^{2} \hat{\epsilon}^{\left( \alpha \right) } \left( c_{k, \alpha} \left( t = 0 \right) e^{i \vec{k} \cdot \vec{x}} + c^{*}_{k, \alpha} \left( t = 0 \right) e^{-i \vec{k} \cdot \vec{x}} \right) 
\end{equation}
\\
\\
where $\displaystyle \hat{\epsilon}^{\left( \alpha \right) }$ are real unit vectors, and $\displaystyle c_{k, \alpha}$ is the coefficient of the wave with wave vector $\displaystyle \vec{k}$ and polarization vector $\displaystyle \hat{\epsilon}^{\left( \alpha \right) }$. Once the wave vector is chose, the two polarization vectors must be picked so that $\displaystyle \hat{\epsilon}^{\left( 1 \right) }$, $\displaystyle \hat{\epsilon}^{\left( 2 \right) }$, and $\displaystyle \vec{k}$ form a right handed orthogonal system. The components of the wave vector must satisfy
\\
\\
\begin{equation}
    k_{i} = \frac{2 \pi n_{i}}{L}
\end{equation}
\\
\\
due to the periodic boundary conditions. The factor out front is set to normalize the states nicely since
\\
\\
\begin{equation}
    \frac{1}{V} \int  \, d^3 x e^{i \vec{k} \cdot \vec{x}} e^{-i \vec{k}' \vec{x}} = \delta_{\vec{k} \vec{k}'}
\end{equation}
\\
\\
and
\\
\\
\begin{equation}
    \hat{\epsilon}^{\left( \alpha \right) } \cdot \hat{\epsilon}^{\left( \alpha' \right) } = \delta_{\alpha \alpha'}
\end{equation}
\\
\\
We know the time dependence of the waves from Maxwell's equation,
\\
\\
\begin{equation}
    c_{k, \alpha} \left( t \right) = c_{k, \alpha} \left( 0 \right) e^{-i \omega t}
\end{equation}
\\
\\
where $\displaystyle \omega = kc$. We can now write the vector potential as a function of position and time.
\\
\\
\begin{equation}
    \vec{A} \left( \vec{x}, \vec{t} \right) = \frac{1}{\sqrt{V} } \sum_{k} \sum_{\alpha = 1}^{2} \hat{\epsilon}^{\left( \alpha \right) } \left( c_{k, \alpha} \left( t \right) e^{i \vec{k} \cdot \vec{x}} + c^{*}_{k, \alpha} \left( t \right) e^{-i \vec{k} \cdot \vec{x}}  \right) 
\end{equation}
We may need to write this solution in several different ways, and use the best one for the calcualtion being performed. One nice way to write this is in terms 4-vector $\displaystyle k_{\mu}$, the wave number,
\\
\\
\begin{equation}
    k_{\mu} = \frac{p_{\mu}}{\hbar} = \left( k_{x}, k_{y}, k_{z}, ik \right)  = \left( k_{x}, k_{y}, k_{z}, i \frac{\omega}{c} \right) 
\end{equation}
\\
\\
so that
\\
\\
\begin{equation}
    k_{p} x_{p} = k \cdot x = \vec{k} \cdot \vec{x} - \omega t
\end{equation}
\\
\\
We can then write the radiation field in a more covariant way
\\
\\
\begin{equation}
    \vec{A} \left( \vec{x}, t \right) = \frac{1}{\sqrt{V} } \sum_{k} \sum_{\alpha = 1}^{2} \hat{\epsilon}^{\left( \alpha \right) } \left( c_{k, \alpha} \left( 0 \right) e^{ik_{p} x_{p}} + c^{*}_{k, \alpha} \left( 0 \right) e^{-i k_{p} x_{p}} \right) 
\end{equation}
\\
\\
A convenient shorthand for calculations is possible by noticing that the second term is just the complex conjugate of the first.
\\
\\
\begin{align*}
	\vec{A} \left( \vec{x}, t \right) &= \frac{1}{\sqrt{V} } \sum_{k}  \sum_{\alpha = 1}^{2} \hat{\epsilon}^{\left( \alpha \right) } \left( c_{k, \alpha} \left( 0 \right) e^{ik_{p} x_{p}}  + c.c. \right)  \\
	\vec{A} \left( \vec{x}, t \right) &= \frac{1}{\sqrt{V} } \sum_{k}  \sum_{\alpha=1}^{2} \hat{\epsilon}^{\left( \alpha \right) }  c_{k, \alpha} \left( 0 \right) e^{ik_{p} x_{p}} + c.c.
\end{align*}
\\
\\
Note again that we have made this a transverse field by construction. The unit vectors $\displaystyle \hat{\epsilon}^{\left( \alpha \right) }$ are transverse to the deirection of propagation. Also note that we are working in a guage with $\displaystyle A_4 = 0$, so this can also represent the 4-vector form of the potential. The Fourier Decomposition of the radiation field can be be be written very simply.
\\
\\
\begin{equation}
	\boxed{A_{\mu} = \frac{1}{\sqrt{V}} \sum_{k} \sum_{\alpha=1}^{2} \epsilon^{\left( \alpha \right) }_{\mu} c_{k, \alpha} \left( 0 \right) e^{ik_{p}x_{p}} + c.c.  }
\end{equation}
\\
\\
This choice of guage makes switching between 4-vector and 3-vector expressions for the potential trivial. Let's verify that this decomposition of the radiation field satisfies the Maxwell equation, just for some practice. It's most convenient to use the covariant form of the equation and field.
\\
\\
\begin{align*}
	\Box A_{\mu} &= 0 \\
	\Box \left( \frac{1}{\sqrt{V} } \sum_{k} \sum_{\alpha=1}^{2} \epsilon^{\left( \alpha \right) }_{\mu} c_{k, \alpha} \left( 0 \right) e^{ik_{p}x_{p}} + c.c. \right) &= \frac{1}{\sqrt{V} } \sum_{k} \sum_{\alpha=1}^{2} \epsilon^{\left( \alpha \right) }_{\mu} c_{k, \alpha} \left( 0 \right)  \Box e^{ik_{p}x_{p}} + c.c. \\
	&= \frac{1}{\sqrt{V} } \sum_{k}  \sum_{\alpha=1}^{2} \epsilon^{\left( \alpha \right) }_{\mu} c_{k, \alpha} \left( 0 \right)  \left( -k_{\nu} k_{\nu} \right)  e^{ik_{p}x_{p}} + c.c. = 0
\end{align*}
\\
\\
The result is zero since $\displaystyle k_{\nu} k_{\nu} = k^2 -k^2 = 0$.
\\
\\
Let's also verify that $\displaystyle \vec{\nabla} \cdot \vec{A} = 0$
\\
\\
\begin{align*}
	\vec{\nabla} \cdot \left( \frac{1}{\sqrt{V} } \sum_{k} \sum_{\alpha = 1}^{2} \hat{\epsilon}^{\left( \alpha \right) } c_{k, \alpha} \left( t \right) e^{i\vec{k} \cdot \vec{x} } + c.c.  \right) &= \frac{1}{\sqrt{V} } \sum_{k} \sum_{\alpha = 1}^{2} c_{k, \alpha} \left( t \right) \hat{\epsilon}^{\left( \alpha \right) } \cdot \vec{\nabla}  e^{i \vec{k} \cdot \vec{x}} + c.c. \\ 
	&= \frac{1}{\sqrt{V} } \sum_{k} \sum_{\alpha=1}^{2} c_{k, \alpha} \left( t \right) \hat{\epsilon}^{\left( \alpha \right) }  \vec{k} e^{i \vec{k} \cdot \vec{x}} + c.c.  = 0
.\end{align*}
\\
\\
The result here is zero because $\displaystyle \hat{\epsilon}^{\left( \alpha \right) } \cdot \vec{k} = 0$


\section{The Hamiltonian for the Radiation Field}

We now wish to compute the Hamiltonian in terms of the coefficients $\displaystyle c_{k, \alpha} \left( t \right) $. This is an important calculation because we will use the Hamiltonian formalism to do the quantization of the field. We will do the calculation using the covariant notaion (while Sakurai outlines an alternate  calculation using 3-vectors). We have already calculated th Hamiltonian density of a classical EM field.
\\
\\
\begin{equation}
	\boxed{\mathcal{H} = F_{\mu 4} \frac{ \partial A_{\mu} }{ \partial x_4 } + \frac{1}{4} F_{\mu \nu} F_{\mu \nu}}
\end{equation}
\\
\\
\begin{align*}
	\mathcal{H} &= \left( \frac{ \partial A_{4} }{ \partial x_{\mu} } - \frac{ \partial A_4 }{ \partial x_4 }    \right)  \frac{ \partial A_{\mu} }{ \partial x_4 } + \frac{1}{4} \left( \frac{ \partial A_{\nu} }{ \partial x_{\mu} }  - \frac{ \partial A_{\mu} }{ \partial x_{\nu} }  \right)  \left( \frac{ \partial A_{\nu} }{ \partial x_{\mu} } - \frac{ \partial  A_{\mu} }{ \partial x_{\nu} }   \right)  \\  
	\mathcal{H} &= - \frac{ \partial A_{4} }{ \partial x_{\mu} }  \frac{ \partial A_{\mu} }{ \partial x_4 } + \frac{1}{2} \left( \frac{ \partial A_{\nu} }{ \partial x_4 } \frac{ \partial A_{\nu} }{ \partial x_4 } - \frac{ \partial  A_{\nu} }{ \partial x_{\nu} } \frac{ \partial A_{\mu} }{ \partial x_{\nu}  }    \right) 
.\end{align*}
\\
\\
Now lets compute the basic element of the above formula for our decomposed radiation field.
\\
\\
\begin{align*}
	A_{\mu} &= \frac{1}{\sqrt{V} } \sum_{k} \sum_{\alpha= 1}^{2} \epsilon^{\left( \alpha \right) }_{\mu} \left(  c_{k, \alpha}  \left( 0 \right)  e^{ik_{p} x_{p}} + c^{*}_{k, \alpha} \left( 0 \right) e^{- ik_{p} x_{p}} \right)  \\ 
	\frac{ \partial A_{\mu} }{ \partial x_{\nu} } &= \frac{1}{\sqrt{V} } \sum_{k} \sum_{\alpha=1}^{2} \epsilon^{\left( \alpha \right) }_{\mu} \left( c_{k, \alpha} \left( 0 \right) \left( ik_{\nu} \right) e^{i k_{p} x_{p}} + c^{*}_{k, \alpha} \left( 0 \right) \left( -ik_{\nu} \right) e^{-i k_{p} x_{p}}   \right) \\
	\frac{ \partial A_{\mu} }{ \partial x_{\nu} } &= i \frac{1}{\sqrt{V} } \sum_{k} \sum_{\alpha=1}^{2} \epsilon^{\left( \alpha \right) }_{\mu} \frac{\omega}{c} \left( c_{k, \alpha} \left( 0 \right) e^{ik_{p} x_{p}} - c^{*}_{k, \alpha} \left( 0 \right) e^{ik_{p}x_{p}} \right) \\
	\frac{ \partial A_{\mu} }{ \partial x_4 } &= - \frac{1}{\sqrt{V} } \sum_{k} \sum_{\alpha=1}^{2} \epsilon^{\left( \alpha \right) }_{\mu}  \frac{\omega}{c} \left( c_{k, \alpha} \left( 0 \right) e^{ik_{p} x_{p}} - c^{*}_{k, \alpha} \left( 0 \right) e^{ik_{p}x_{p}} \right) 
.\end{align*}
\\
\\
We have all the elements to finish the calculation of the Hamiltonian. Before pulling this all together in a brute force way, it's good to realize that almost all the terms will give zero. We see that the derivative of $\displaystyle A_{\mu}$ is proportional to a 4-vector, say $\displaystyle k_{\nu}$ and to a polarization vector, say $\displaystyle \epsilon^{\left( \alpha \right) }_{\mu}$. The dot products of the 4-vectors, either $\displaystyle k$ with itself, or $\displaystyle k$ with $\epsilon$ are zero. Going back to out expression for the Hamiltonian density, we can eliminate some terms. 
\\
\\
\begin{align*}
	\mathcal{H} &= - \frac{ \partial A_{\mu} }{ \partial x_4  } \frac{ \partial  A_{\mu} }{ \partial x_4 }  + \frac{1}{2} \left( \frac{ \partial A_{\nu} }{ \partial x_{\mu} }  \frac{ \partial A_{\nu} }{ \partial x_{\mu} } - \frac{ \partial A_{\nu} }{ \partial x_{\mu} } \frac{ \partial A_{\mu} }{ \partial x_{\nu} }    \right)  \\ 
	\mathcal{H} &= - \frac{ \partial A_{\nu} }{ \partial x_4 } \frac{ \partial A_{\mu} }{ \partial x_4 }  + \frac{1}{2} \left( 0 - 0  \right) \\
	\mathcal{H} &= - \frac{ \partial A_{\mu} }{ \partial x_4 } \frac{ \partial A_{\mu} }{ \partial x_4 } 
.\end{align*}
\\
\\
The remaining term has a dot product between polarization vectors which will be nonzero if the polarization vectors are the same. (Note that this simplification is possible because we have assumed no sources in the region.) The total Hamiltonian we are aiming at, is the integral of the Hamiltonian density.
\\
\\
\begin{equation}
	H = \int  \, d^3x \mathcal{H}
\end{equation}
\\
\\
When we integrate over the volume only products like $\displaystyle e^{ik_{p}x_{p}} e^{-ik_{p}x_{p}}$ will give a nonzero result. So when we multiply one sum over $k$ by another, only the terms with the same  $k$ will contribute to the integral, basically because the waves with different wave number are orthogonal.
\\
\\
\begin{equation}
    \frac{1}{V} \int  \, d^3x e^{ik_{p} x_{p}} e^{-ik'_{p}x_{p}} = \delta_{k k'}
\end{equation}
\\
\\
\begin{align*}
	H &= \int  \, d^3x \mathcal{H} \\ 
	\mathcal{H} &= - \frac{ \partial A_{\mu} }{ \partial x_4 } \frac{ \partial A_{\mu} }{ \partial x_4 } \\ 
	\frac{ \partial A_{\mu} }{ \partial x_4 } &= - \frac{1}{\sqrt{V} } \sum_{k} \sum_{\alpha=1}^{2} \epsilon^{\left( \alpha \right) }_{\mu}  \left( c_{k, \alpha} \left( 0 \right) \frac{\omega}{c} e^{ik_{p}x_{p}} - c^{*}_{k, \alpha} \left( 0 \right) \frac{\omega}{c} e^{-ik_{p}x_{p}}   \right) \\
	H &= - \int  \, d^3x \frac{ \partial A_{\mu} }{ \partial x_4 } \frac{ \partial A_{\mu} }{ \partial x_4 } \\
	H &= - \int  \, d^3x \frac{1}{V} \sum_{k} \sum_{\alpha}^{2} \left( c_{k, \alpha} \left( 0 \right) \frac{\omega}{c} e^{ik_{p}x_{p}} - c^{*}_{k, \alpha} \left( 0 \right) e^{-ik_{p}x_{p}} \right)  \\
	H &= - \sum_{k} \sum_{\alpha=1}^{2} \left( \frac{\omega}{c} \right)^2 \left[ - c_{k, \alpha} \left( t \right) c^{*}_{k, \alpha}  \left( t \right) - c^{*}_{k, \alpha} \left( t \right) c_{k, \alpha}  \left( t \right)  \right] \\
	H &= \sum_{k} \sum_{\alpha}^{2} \left( \frac{\omega}{c} \right)^2 \left[ c_{k, \alpha} \left( t \right) c^{*}_{k, \alpha}  \left( t \right) + c^{*}_{k, \alpha} \left( t \right) c_{k, \alpha} \left( t \right)  \right] \\
	H &= \sum_{k, \alpha} \left( \frac{\omega}{c} \right)^2 \left[ c_{k, \alpha} \left( t \right) c^{*}_{k, \alpha} \left( t \right) + c^{*}_{k, \alpha} \left( t \right) c_{k, \alpha} \left( t \right)  \right] 
.\end{align*}
\\
\\
This is the result we will use to quantize the field. We have been careful not to commute $C$ and  $C*$ here in anticipation of the fact that they do not commute.
\\
\\
It should not be a surprise that the terms that made up the Lagrangian gave a zero contribution because  $\displaystyle \mathcal{L} = \frac{1}{2} \left( E^2 - B^2 \right) $ and we know that E and B have the same magnitude in radiation field. (There is one wrinkle we have glossed over; terms with $\displaystyle \vec{k}' = -\vec{k}$.)


\section{Canonical Coordinates and Momenta}

We now have the Hamiltonian for the radiation field
\\
\\
\begin{equation}
	\boxed{H = \sum_{k, \alpha} \left( \frac{\omega}{c} \right)^2 \left[ c_{k, \alpha} \left( t \right) c^{*}_{k, \alpha} \left( t \right) + c^{*}_{k, \alpha} \left( t \right) c_{k, \alpha} \left( t \right)  \right]   }
\end{equation} 
\\
\\
It was with the Hamiltonian that we first quantized the non-relativistic motion of particles. The position and momentum became operators which did not commute. Lets define $\displaystyle c_{k, \alpha}$ to be the time dependent Fourier coefficient.
\\
\\
\begin{equation}
    \ddot{c}_{k, \alpha} = -\omega^2 c_{k, \alpha}
\end{equation}
\\
\\
We can then simplify our notation a bit
\\
\\
\begin{equation}
	H = \sum_{k,\alpha} \left( \frac{\omega}{c} \right)^2 \left[ c_{k, \alpha} c^{*}_{k, \alpha} + c^{*}_{k, \alpha} c_{k, \alpha} \right]
\end{equation} 
\\
\\
This now clearly looks like the Hamiltonian for a collection of uncoupled oscillators; one oscillator for each wave vector and polariation.
\\
\\
We wish th write the Hamiltonian in terms of a coordinate for each oscillator and the conjugate momenta. The coordinate should be real so it can be represented by a Hermitian operator and have a physical meaning. The simplest choice for a read coordinates is $\displaystyle c + c^{*}$. With a little effort we can identify the coordinate
\\
\\
\begin{equation}
	Q_{k, \alpha} = \frac{1}{c} \left( c_{k, \alpha} + c^{*}_{k, \alpha} \right) 
\end{equation}
\\
\\
and its conjugate momentum for each oscillator, 
\\
\\
\begin{equation}
    P_{k, \alpha} = - \frac{i \omega}{c} \left( c_{k, \alpha} - c^{*}_{k, \alpha} \right) 
\end{equation}
\\
\\
The Hamiltonian can be written in terms of these
\\
\\
\begin{align*}
	H &= \frac{1}{2} \sum_{k, \alpha} \left[ P^2_{k, \alpha} + \omega^2 Q^2_{k, \alpha} \right] \\
	  &= \frac{1}{2} \sum_{k, \alpha} \left[ - \left( \frac{\omega}{c} \right)^2 \left( c_{k, \alpha} - c^{*}_{k, \alpha} \right)^2 + \left( \frac{\omega}{c} \right)^2 \left( c_{k, \alpha} + c^{*}_{k, \alpha} \right)^2     \right] \\
	  &= \frac{1}{2} \sum_{k ,\alpha} \left( \frac{\omega}{c} \right)^2 \left[ - \left( c_{k, \alpha} - c^{*}_{k, \alpha}  \right)^2 + \left( c_{k, \alpha} + c^{*}_{k, \alpha} \right)^2   \right] \\
	  &= \frac{1}{2} \sum_{k, \alpha} \left( \frac{\omega}{c} \right)^2 2 \left[ c_{k, \alpha} c^{*}_{k, \alpha} + c^{*}_{k, \alpha} c_{k, \alpha} \right] \\
	  &= \sum_{k, \alpha}  \left( \frac{\omega}{c} \right)^2 \left[ c_{k, \alpha} c^{*}_{k, \alpha} + c^{*}_{k, \alpha} c_{k, \alpha} \right]
.\end{align*}
\\
\\
This verifies that this choice gives the right Hamiltonian. We should also check that this choice of coordinates and momenta satisfy Hamiltonian's equations to identify them as the canonical coordinates. The first equation is 
\\
\\
\begin{align*}
	\frac{ \partial H }{ \partial Q_{k, \alpha} }  &= - \dot{P}_{k, \alpha} \\
	\omega^2 Q_{k, \alpha} &= \frac{i \omega}{c} \left( \dot{c}_{k, \alpha} - \dot{c}^{*}_{k, \alpha} \right) \\
	\frac{\omega^2}{c} \left( c_{k, \alpha} + c^{*}_{k, \alpha} \right) &= \frac{i \omega}{c} \left( - i \omega c_{k, \alpha} - i \omega c^{*}_{k, \alpha} \right) \\
	\frac{\omega^2}{c} \left( c_{k, \alpha} + c^{*}_{k, \alpha} \right) &= \frac{\omega^2}{c} \left( c_{k, \alpha} + c^{*}_{k, \alpha} \right) 
.\end{align*}
\\
\\
This one checks out OK.
\\
\\
The other equation of Hamiltonian is
\\
\\
\begin{align*}
	\frac{ \partial H }{ \partial P_{k, \alpha} }  &= Q_{k, \alpha} \\
	P_{k, \alpha} &= \frac{1}{c} \left( \dot{c}_{k, \alpha} + \dot{c}^{*}_{k, \alpha} \right) \\
	- \frac{-i \omega}{c} \left( c_{k, \alpha} - c^{*}_{k, \alpha} \right) &= \frac{1}{c} \left( -i \omega c_{k, \alpha} + i \omega c^{*}_{k, \alpha} \right) \\
	\frac{- i \omega}{c} \left( c_{k, \alpha} - c^{*}_{k, \alpha} \right)  &= - \frac{i \omega}{c} \left( c_{k, \alpha} - c^{*}_{k, \alpha} \right) 
.\end{align*}
\\
\\
This also checks out, so we have identified the canonical coordinates and momenta of our oscillators.
\\
\\
We have a collection of uncoupled oscillators with identified canonical coordinate and mementum. The next step is to quantize the oscillators.



\section{Quantization of the Oscillators} 
To summarize the result of the calculatons of the last section we have the Hamiltonian for the radiation field.
\\
\\
\begin{equation}
	\boxed{
		\begin{gathered}
			\boxed{\makebox[0.75\textwidth]{$\displaystyle \mathbf{H} = \sum_{k, \alpha} \left( \frac{\omega}{c} \right)^2 \left[ c_{k,\alpha} c^{*}_{k, \alpha} + c^{*}_{k, \alpha} c_{k, \alpha} \right]$}} \\[1em] \boxed{\makebox[0.75\textwidth]{$\displaystyle Q_{k, \alpha} = \frac{1}{c} \left( c_{k, \alpha} + c^{*}_{k, \alpha} \right)$}} \\[1em] \boxed{\makebox[0.75\textwidth]{$\displaystyle P_{k, \alpha} = - \frac{i \omega}{c} \left( c_{k, \alpha} - c^{*}_{k, \alpha} \right)$}} \\[1em] \boxed{\makebox[0.75\textwidth]{$\displaystyle H = \frac{1}{2} \sum_{k, \alpha} \left[ P^2_{k, \alpha} + \omega^2 Q^2_{k, \alpha} \right]$}} \end{gathered} }
\end{equation}
\\
\\
Soon after the development of non-relativistic quantum mechanics, Dirac proposed that the canonical variables of the radiation oscillators be reated like $p$ and $x$  in the quantum mechanics we know. The place to start is with the commutators. The coordinate and its corresponding momentum do not commute. For example $\displaystyle \left[ p_{x}, x \right] = \frac{\hbar}{i}$. Coordinates and momenta that do not correspond, do not commute. For example $\displaystyle \left[ p_{y}, x \right] = 0 $. Different coordinates vommute with each other as do different momenta. We will impose the same rules here.
\\
\\
\begin{align*}
	\left[Q_{k, \alpha}, P_{k', \alpha'}  \right] &= i \hbar \delta_{k k'} \delta_{\alpha \alpha'} \\
	\left[ Q_{k, \alpha} , Q_{k', \alpha'} \right] &= 0 \\
	\left[ P_{k, \alpha}, P_{k', \alpha'} \right]  &= 0
.\end{align*}
\\
\\
By now we know that if the $Q$ and  $P$ do not commute, neither do the  $\displaystyle c$ and $\displaystyle c^{*}$ so we should continue to avoid commuting them.
\\
\\
Since we are dealing with harmonic oscillators, we want to find the analog of the raising and lowering operators. We developed the raising and lowering operators by trying to write the Hamiltonian as $\displaystyle H = A^{\dagger} A \hbar \omega$. Follwing the same idea, we get
\\
\\
\begin{align*}
	a_{k, \alpha} &= \frac{1}{\sqrt{2 \hbar \omega} } \left( \omega Q_{k, \alpha} + iP_{k, \alpha} \right) \\
	a^{\dagger}_{k, \alpha} &= \frac{1}{\sqrt{2\hbar \omega} } \left( \omega Q_{k, \alpha} - i P_{k, \alpha} \right) \\
	a^{\dagger}_{k, \alpha} a_{k, \alpha} &= \frac{1}{2 \hbar \omega} \left( \omega Q_{k, \alpha} - i P_{k, \alpha}  \right)  \left( \omega Q_{k, \alpha} + iP_{k, \alpha} \right) \\
	&= \frac{1}{2\hbar \omega} \left( \omega^2 Q^2_{k, \alpha} + P^2_{k, \alpha} + i \omega Q_{k, \alpha} P_{k, \alpha} - i \omega P_{k, \alpha} Q_{k, \alpha} \right) \\
	 &= \frac{1}{2\hbar \omega} \left( \omega^2 Q^2_{k, \alpha} + P^2_{k, \alpha}  + i \omega Q_{k, \alpha} P_{k, \alpha} - i \omega \left( Q_{k, \alpha} P_{k, \alpha} + \frac{\hbar}{i} \right)   \right) \\
	 &= \frac{1}{2 \hbar \omega} \left( \omega^2 Q^2_{k, \alpha} + P^2_{k, \alpha} - \hbar \omega \right) \\
	a^{\dagger}_{k, \alpha} a_{k, \alpha} + \frac{1}{2} &= \frac{1}{2 \hbar \omega} \left( \omega^2 Q^2_{k, \alpha} + P^2_{k, \alpha} \right) \\
	\left( a^{\dagger}_{k, \alpha} a_{k, \alpha} + \frac{1}{2} \right) \hbar \omega &= \frac{1}{2} \left( \omega^2 Q^2_{k, \alpha} + P^2_{k, \alpha}  \right)  = \mathbf{H}
.\end{align*}
\\
\\
\begin{equation}
    \boxed{ \mathbf{H} = \left( a^{\dagger}_{k, \alpha} a_{k, \alpha} + \frac{1}{2} \right) \hbar \omega} 
\end{equation}
\\
\\
This is just the same as the Hamiltonian that we had for the one dimensional harmonic oscillator. We therefore have the raising and lowering operators, as long as $\displaystyle \left[ a_{k, \alpha}, a^{\dagger}_{k, \alpha} \right] = 1$, as we had for the 1D harmonic oscillator.
\\
\\
\begin{align*}
	\left[ a_{k, \alpha}, a^{\dagger}_{k, \alpha} \right] &= \left[ \frac{1}{\sqrt{2 \hbar \omega} } \left( \omega Q_{k, \alpha} + iP_{k, \alpha} \right), \frac{1}{\sqrt{2 \hbar \omega} } \left( \omega Q_{k, \alpha} - iP_{k, \alpha} \right)   \right] \\
	&= \frac{1}{2\hbar \omega} \left[ \omega Q_{k, \alpha} + iP_{k, \alpha}, \omega Q_{k, \alpha} - iP_{k, \alpha} \right] \\
	&= \frac{1}{2 \hbar \omega} \left( - i \omega \left[ Q_{k, \alpha}, P_{k, \alpha} \right] + i \omega \left[ P_{k, \alpha}, Q_{k, \alpha} \right] \right) \\
	&= \frac{1}{2\hbar \omega} \left( \hbar \omega + \hbar \omega \right) \\
	&= 1
.\end{align*}
\\
\\
So these are definitely the raising and loweing operators. Of course the commutator would be zero if the operators were not for the same oscillator.
\\
\\
\begin{equation}
	\left[ a_{k, \alpha}, a^{\dagger}_{k', \alpha'} \right]  = \delta_{k k'} \delta_{\alpha \alpha'}
\end{equation}
\\
\\
(Note that all of our commutators are assumed to be taken at equal time.) The Hamiltonian is written in terms $a \text{ and } a^{\dagger}$ in the same way as for the 1D harmonic oscillator. Therefore, everything we know about the raising and lowering operators applies here, including the commutator with the Hamiltonian, the raising and lowering of energy eigenstates, and even the constants.
\\
\\
\begin{align*}
	a_{k, \alpha} \left| n_{k, \alpha} \right> &= \sqrt{n_{k, \alpha}} \left| n_{k, \alpha} - 1 \right> \\
	a^{\dagger}_{k, \alpha} \left| n_{k, \alpha} \right> &= \sqrt{n_{k, \alpha} + 1} \left| n_{k, \alpha} \right>
.\end{align*}
\\
\\
The $n_{k, \alpha}$ can only take on integer values as with the harmonic oscillator we knw.
\\
\\
As with the 1D harmonic oscillator, we also casn define the number operator.
\\
\\
\begin{align*}
	\mathbf{H} &= \left( a^{\dagger}_{k, \alpha} a_{k, \alpha} + \frac{1}{2} \right) \hbar \omega = \left( N_{k, \alpha} + \frac{1}{2} \right) \hbar \omega
.\end{align*}
\\
\\
The last step is to compute the raising and lowering operators in terms of the original coefficients.
\\
\\
\begin{align*}
	a_{k, \alpha} &= \frac{1}{\sqrt{2 \hbar \omega} } \left( \omega Q_{k, \alpha} + iP_{k, \alpha} \right) \\
	Q_{k, \alpha} &= \frac{1}{c} \left( c_{k, \alpha} + c^{*}_{k, \alpha} \right) \\
	P_{k, \alpha} &= - \frac{i \omega}{c} \left( c_{k, \alpha} - c^{*}_{k, \alpha} \right) \\
	&= \frac{1}{\sqrt{2 \hbar \omega} } \frac{\omega}{c} \left( \left( c_{k, \alpha} + c^{*}_{k, \alpha} \right) + \left( c_{k, \alpha} - c^{*}_{k, \alpha} \right)   \right) \\
	&= \frac{1}{\sqrt{2 \hbar \omega} }\frac{\omega}{c} \left( c_{k, \alpha} + c^{*}_{k, \alpha} + c_{k, \alpha} - c^{*}_{k, \alpha} \right) \\
	&= \sqrt{\frac{\omega}{2 \hbar c^2}}  \left( 2c_{k, \alpha} \right) \\
	&= \sqrt{\frac{2\omega}{\hbar c^2}  c_{k, \alpha}}
.\end{align*}
\\
\\
\begin{equation}
	\boxed{c_{k, \alpha} = \sqrt{\frac{\hbar c^2}{2 \omega}} a_{k, \alpha} }   
\end{equation}
\\
\\
Similarly we can compute that
\\
\\
\begin{equation}
    \boxed{c^{*}_{k, \alpha} = \sqrt{\frac{\hbar c^2}{2\omega}} a^{\dagger}_{k, \alpha}} 
\end{equation}
\\
\\
Since we now have the coefficients in our decomposition of the field equal to a constant times the raising or lowering operators, it is clear that these coefficients have themselves operators.

\section{Photon States} 
It is now obvious that the integer $\displaystyle n_{k, \alpha}$ is the number of photons in the volume with wave number $\displaystyle \vec{k}$ and polarization $\displaystyle \hat{\epsilon}^{\left( \alpha \right) }$. It is called the occupation number for the state designated by the wave number $\displaystyle \vec{k}$ and polarization $\displaystyle \hat{\epsilon}^{\left( \alpha \right) }$. We can represent the state of the entire volume by giving the number of photons of each type (and some phases). The state vector for the volume is given by the direct product of the states for each type of photon.
\\
\\
\begin{equation}
	\left| n_{k_1, \alpha_1}, n_{k_2, \alpha_2}, \dots , n_{k_{i}, \alpha_{i}}, \dots \right> = \left| n_{k_1, \alpha_1} \right> \left|n_{k_2, \alpha_2}\right> \dots \left|n_{k_{i}, \alpha_{i}}\right> \dots
\end{equation}

The ground state for a particular oscillator cannot be lowered. The state in which all the oscilltors are in the ground state is called the vacuum state and can be written simply as $\displaystyle \left| 0 \right>$

\begin{equation}
    \left| n_{k_1, \alpha_1}, n_{k_2, \alpha_2}, \dots, n_{k_{i}, \alpha_{i}}, \dots \right> = \prod_{i} \frac{\left( a^{\dagger}_{k_{i}, \alpha_{i}} \right)^{n_{k_{i}, \alpha_{i}}} }{\sqrt{n_{k_{i}, \alpha_{i}}!} }
\end{equation}
\\
\\
The factorial on the bottom cancels all the $\displaystyle \sqrt{n + 1} $ we get from the raising operators.
\\
\\
Any multi-photon state we construct is automatically symmetric under the interchange of pairs of photons. For example if we want to raise two photons out of the vacuum, we apply two raising operators. Since $\displaystyle \left[ a^{\dagger}_{k, \alpha}, a^{\dagger}_{k', \alpha'} \right] = 0$, interchanging the photons gives the same state.
\\
\\
\begin{equation}
    a^{\dagger}_{k, \alpha}, a^{\dagger}_{k', \alpha'} \left| 0 \right>  = a^{\dagger}_{k', \alpha'}, a^{\dagger}_{k, \alpha} \left| 0 \right>
\end{equation}
\\
\\
So the fact that the creation operators commute dictates that photon states are symmetricunder interchange.


\section{Fermion Operators} 
At this point, we can hypothesize that the operators that create fermion states do not commute. In fact, if we assume the operators fermion states anti-commute (as do the Pauli matricies), then we can show that fermion states are antisymmetric under interchange. Assume $\displaystyle b^{\dagger}_{r}$ and $\displaystyle b_{r}$ are the creation and annihilation operators for fermions and that they anti-commute.
\\
\\
\begin{equation}
	\boxed{\left\{ b^{\dagger}_{r}, b^{\dagger}_{r'} \right\} = 0} 
\end{equation}
\\
\\
The states are then antisymmetric under interchange of pairs of fermions.
 \\
 \\
 \begin{equation}
     b^{\dagger}_{r} b^{\dagger}_{r'} \left| 0 \right> = -b^{\dagger}_{r'} b^{\dagger}_{r} \left| 0 \right>
 \end{equation}
 \\
 \\
 It's not hard to show that the occupation number for fermion states is either zero or one.

\section{Quantized Radiation Field}  
The Fourier coefficients of the expansion of the classical radiation field should now be replaced by operators.
\\
\\
\begin{align*}
	c_{k, \alpha} &\to \sqrt{\frac{\hbar c^2}{2 \omega}} a_{k, \alpha} \\
	c^{*}_{k, \alpha} &\to \sqrt{\frac{\hbar c^2}{2 \omega}}  a^{\dagger}_{k, \alpha} \\
	A_{\mu} &= \frac{1}{\sqrt{V} } \sum_{k, \alpha} \sqrt{\frac{\hbar c^2}{2 \omega}} \epsilon^{\left( \alpha \right) }_{\mu} \left( a_{k, \alpha} \left( t \right) e^{i\vec{k} \cdot \vec{x}} + a^{\dagger}_{k, \alpha} \left( t \right) e^{-i \vec{k} \cdot \vec{x}} \right) 
.\end{align*}
\\
\\
$A$ is now an operator that acts on state vectors in occupation number space. The operator is parameterized in terms of  $\vec{x}$ and $\displaystyle t$. This type of operator is called a field operator or a quantized field. The Hamiltonian operator can also be written in terms of the creation and annihilation operators.
\\
\\
\begin{align*}
	\mathbf{H} &= \sum_{k, \alpha} \left( \frac{\omega}{c} \right)^2 \left[ c_{k, \alpha}c^{*}_{k, \alpha} + c^{*}_{k, \alpha} c_{k, \alpha} \right] \\
			   &= \sum_{k, \alpha} \left( \frac{\omega}{c} \right)^2 \frac{\hbar c^2}{2 \omega} \left[ a_{k, \alpha} a^{\dagger}_{k, \alpha} + a^{\dagger}_{k, \alpha} a_{k, \alpha} \right] \\
			   &= \frac{1}{2} \sum_{k, \alpha} \hbar \omega \left[ a_{k, \alpha} a^{\dagger}_{k, \alpha} + a^{\dagger}_{k, \alpha} a_{k, \alpha} \right]
.\end{align*}
\\
\\
\begin{equation}
    \boxed{\mathbf{H} = \sum_{k, \alpha} \hbar \omega \left( N_{k, \alpha} + \frac{1}{2} \right) } 
\end{equation}
\\
\\
For our purposes, we may remove the (infinite) constant energy due to the ground state enrgy of all the oscillators. It is simply the energy of the vacuum which we may define as zero. Note that the field fluctuations that cause this energy density, also cause the spontaneous decay of states of atoms. One thing that must be done is to cut off the sum at some maximum value of $k$. We so not expect electricity and magnetism to be completely valid up to infinite energy. Certainly by the gravitational or grand unified enrgy scale there must be important corrections to our formulas. The energy density of the vacuum is hard to define but plays an important role in cosmology. At this time, physicists have difficulty explaining how small the energy density in the vacuum is. Until recent experiments showed otherwise, most physicists thought it was actually zero due to some unknown symmetry. In any case we are not ready to consider this problem.
\\
\\
 \begin{equation}
    \mathbf{H} = \sum_{k, \alpha} \hbar \omega N_{k, \alpha}
\end{equation}
\\
\\
With this subtraction, the enrgy of the vacuum state has been defined to be zero.
\\
\\
\begin{equation}
    \mathbf{H} \left| 0 \right> = 0 
\end{equation}
\\
\\
This time the $\frac{1}{2}$ can really be dropped since the sum is over positive and negative $\vec{k}$, so it sums to zero.
\\
\\
\begin{equation}
    \vec{P} = \sum_{k, \alpha}  \hbar \vec{k} N_{k, \alpha}
\end{equation}
\\
\\
The total momentum in the (transverse) radiation field can also be computed (from the classical formula for the Poynting vector)
\\
\\
\begin{equation}
    \vec{P} = \frac{1}{c} \int \vec{E} \times \vec{B} d^3x = \sum_{k, \alpha} \hbar \vec{k} \left( N_{k, \alpha} + \frac{1}{2} \right) 
\end{equation}
\\
\\
We can compute the energy and momentum of a single photon state by operating on the state with the Hamiltonian and with the total momentum operator. The state for a single photon with a given momentum and polarization can be written as $\displaystyle a^{\dagger}_{k, \alpha} \left| 0 \right>$
\\
\\
\begin{equation}
	\mathbf{H} a^{\dagger}_{k, \alpha} \left| 0 \right> = \left( a^{\dagger}_{k, \alpha} \mathbf{H} + \left[ \mathbf{H}, a^{\dagger}_{k, \alpha} \right]  \right) \left| 0 \right> = 0 + \hbar \omega a^{\dagger}_{k, \alpha} \left| 0 \right> = \hbar \omega a^{\dagger}_{k, \alpha} \left| 0 \right>
\end{equation}
\\
\\
The energy of single photon state is $\hbar \omega$
\\
\\
\begin{equation}
	Pa^{\dagger}_{k, \alpha} \left| 0 \right> = \left( a^{\dagger}_{k, \alpha} P + \left[ P, a^{\dagger}_{k, \alpha} \right]  \right) \left| 0 \right> = 0 + \hbar \vec{k} a^{\dagger}_{k, \alpha} \left| 0 \right> = \hbar \vec{k} a^{\dagger}_{k, \alpha} \left| 0 \right>
\end{equation}
\\
\\
The momentum of the single photon state is $\hbar \vec{k}$. The mass of the photon can be computed.
\\
\\
\begin{align*}
	E^2 &= p^2 c^2 + \left( mc^2 \right)^2 \\
	mc^2 &= \sqrt{ \left( \hbar \omega \right)^2 - \left( \hbar k \right)^2 c^2  }  = \hbar \sqrt{\omega^2 - \omega^2}  = 0
.\end{align*}
\\
\\



\noindent
\begin{minipage}[t]{0.48\textwidth}
The polarization $\displaystyle \hat{\epsilon}^{\left( \pm \right) }$ is associated with the $\displaystyle m = \pm 1$ component of the photon's spin. These are the transverse mode of the photon, $\vec{k} \cdot \hat{\epsilon}^{\left( \pm \right) } = 0$. We have separated the field into transverse and longitudinal parts. The longitudinal part is partially responsible for static E and B fields, while the transverse part makes up radiation. The $\displaystyle m=0$ component of the photon is not present in radiation but is important in understanding static fields.
\end{minipage}
\hfill
\begin{minipage}[t]{0.48\textwidth}
	By assuming the canonical coordinates and momenta in the Hamiltonian have commutators like those of the position and momentum of a particle, led to an understanding that radiation is made up of spin-1 particles with mass zero. All fields correspond to a particle of definite mass and spin. We now have a pretty good idea how to quantize the field for any particle.
\end{minipage}


\section{The Time Development of Field Operators} 
The creation and annihilation operators are related to the time dependent coefficients in our Fourier expansion of the radiation field.
\\
\\
\begin{align*}
	c_{k, \alpha} \left( t \right) &= \sqrt{\frac{\hbar c^2}{2 \omega}} a_{k, \alpha} \\
	c^{*}_{k, \alpha} \left( t \right) &= \sqrt{\frac{\hbar c^2}{2 \omega}} a^{\dagger}_{k, \alpha}
.\end{align*}
\\
\\
This means that the creation, annihilation, and other operators are time dependent operators as we have studied the Heisenberg representation. In particular, we derived the canonical equation for the time dependence of an operator.
\\
\\
\begin{align*}
	\frac{d}{dt}B \left( t \right) &= \frac{i}{\hbar} \left[ H, B \left( t \right)  \right] \\
	\dot{a}_{k, \alpha} &= \frac{i}{\hbar} \left[ H, a_{k, \alpha} \left( t \right)  \right] = \frac{i}{\hbar} \left( - \hbar \omega \right) a_{k, \alpha} \left( t \right) = - i \omega a_{k, \alpha} \left( t \right) \\
	\dot{a}^{\dagger}_{k, \alpha} &= \frac{i}{\hbar} \left[ H, a^{\dagger}_{,k, \alpha} \left( t \right)  \right] = i \omega a^{\dagger}_{k, \alpha} \left( t \right) 
.\end{align*}
\\
\\
So the operators have the same time dependence as did the coefficients in the Fourier expansion.
\\
\\
\begin{align*}
	a_{k, \alpha} &= a_{k, \alpha} \left( 0 \right) e^{-i \omega t} \\
	a^{\dagger}_{k, \alpha} \left( t \right) &= a^{\dagger}_{k, \alpha} \left( 0 \right) e^{i \omega t}
.\end{align*}
\\
\\
We can now write the quantized radiation field in terms of the operators at $\displaystyle t = 0$.
\\
\\
\begin{equation}
    \boxed{A_{\mu} = \frac{1}{\sqrt{V} } \sum_{k, \alpha}  \sqrt{\frac{\hbar c^2}{2 \omega}} \epsilon^{\left( \alpha \right) }_{\mu} \left( a_{k, \alpha} \left( 0 \right) e^{ik_{p}x_{p}} + a^{\dagger}_{k, \alpha} \left( 0 \right) e^{-ik_{p}x_{p}} \right)    } 
\end{equation}
\\
\\
Again, the 4-vector $\displaystyle x_{p}$ is a parameter of this field, not the location of a photon. The field operator is Hermitial and the field itself is real.


\section{Uncertainty relations and RMS Field Fluctuations} 

since the fields are sum of creation and annihilation operators, they do not commute with the occupation number operators
\begin{equation}
	N_{k, \alpha} = a^{\dagger}_{k, \alpha} a_{k, \alpha}
\end{equation}


\end{document}
