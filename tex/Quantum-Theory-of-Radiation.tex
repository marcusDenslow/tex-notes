\documentclass{report}

\input{../templates/preamble}
\input{../templates/macros}
\input{../templates/letterfonts}

\title{\Huge{Quantum Theory of Radiation}}
\author{\Huge{Marcus Allen Denslow}}
\date{2026-01-18}

\begin{document}

\maketitle
\newpage% or \cleardoublepage
% \pdfbookmark[<level>]{<title>}{<dest>}
\pdfbookmark[section]{\contentsname}{toc}
\tableofcontents
\pagebreak

\chapter{Quantum Theory of Radiation}
\section{Transverse and Longitudinal Fields}
In non-relativistic Quantum Mechanics, the static Electric field is represented by a scalar potential, magnetic fields by the vector potential, and the radiation field also through the vector potential. It will be convenient to keep this separation between the large static atomic Electric field and the radiation fields, however, the equations we have contain the four-vector $\displaystyle A_{\mu}$ with all the fields mixed. When the quantize the field, all E and B fields as well as electromagnetic waves will be made up of photons. It is useful to be able to separate the E fields due to fixed charges from the EM radiation from moving charges. This separation is not Lorentz invariant, but it is still useful. Enrico Fermi showed, in 1930, that $\displaystyle A_{\|}$ together with $\displaystyle A_0$ give rise to Coulomb interactions between particles, whereas $\displaystyle A_{\bot}$ gives rise to the EM radiation from moving charges. With this separation, we can maintain the form of our non-relativistic Hamiltonian.
\\
\\
\begin{equation}
	\boxed{\mathbf{H} = \sum_{j} \frac{1}{2m_{j}} \left( \vec{p} - \frac{e}{c} \vec{A}_{\bot} \left( \vec{x}_{j} \right)  \right)^2 + \sum_{i > j} \frac{e_{i} e_{j}}{4\pi \left|  \right|\vec{x}_{i} - \vec{x}_{j}} + \mathbf{H_{rad}} }
\end{equation}
\\
\\
Where $\displaystyle \mathbf{H_{rad}}$ is purely the Hamiltonian of the radiation (containing only $\displaystyle \vec{A}_{\bot}$) and $\displaystyle \vec{A}_{\bot}$ is the part of the vector potential which satisfies $\displaystyle \nabla \cdot \vec{A}_{\bot} = 0$. Note that $\displaystyle \vec{A}\| \text{ and } A_4$ appear nowhere in the Hamiltonian. Instead, we have the Coulomb potential. This separation allows us to continue with our standard Hydrogen solution and just add radiation. We will not derive this result.
\\
\\
In a region in which there are no source terms,
\\
\\
\begin{equation}
    j_{\mu} = 0
\end{equation}
\\
\\
we can make a gauge transformation which eliminates $\displaystyle A_0$ by choosing $\displaystyle \Lambda$ such that
\\
\\
\begin{equation}
	\frac{1}{c} \frac{ \partial \Lambda }{ \partial t }  = A_0
\end{equation}
\\
\\
Since the fourth component of $\displaystyle A_{\mu}$ is now eliminated, the Lorentz condition now implies that
\\
\\
\begin{equation}
    \vec{\nabla} \cdot \vec{A} = 0
\end{equation}
\\
\\
Again, making one component of a 4-vector zero is not a Lorentz invariant way of working. We have to redo the gauge transformation if we move to another frame.
\\
\\
If $\displaystyle j_{\mu} \neq 0$, then we cannot eliminate $\displaystyle A_0$, since $\displaystyle \boxed A_0 = \frac{j_0}{c}$ and we are only allowed to make gauge transformation for which $\displaystyle \boxed \Lambda = 0$. In this case we must separate the vector potential into the transverse and longitudinal parts, with
\\
\\
\begin{align*}
	\vec{A} &= \vec{A}_{\bot} + \vec{A}_{\|}\\
	\vec{\nabla} \cdot \vec{A}_{\bot} &= 0 \\
	\vec{\nabla} \times \vec{A}_{\|} &= 0
.\end{align*}
\\
\\
We will now study the radiation field in a region with no sources so that $\displaystyle \vec{\nabla} \cdot \vec{A} = 0 $. We will use the equations
\\
\\
\begin{align*}
	\vec{B} &= \vec{\nabla} \times \vec{A} \\
	\vec{E} &= -\frac{1}{c} \frac{ \partial \vec{A} }{ \partial t } \\
	\nabla^2 \vec{A} - \frac{1}{c^2} \frac{ \partial \vec{A} }{ \partial t } &= 0
.\end{align*}


\section{Fourier Decomposition of Radiation Oscillators}
Our goal is to write the Hamiltonian for the radiation field in terms of a sum of harmonic oscillators Hamiltonians. The first step is to write the radiation field in a simple way as possible, as a sum of harmonic components. We will work in a cubic volume $\displaystyle V = L^3$ and apply periodic boundary conditions on our electromagnetic waves. We also assume for now that there are no sources inside the region so that we can make a gauge transformation to make $\displaystyle A_0 = 0$ and hence $\displaystyle \vec{\nabla} \cdot \vec{A} = 0$ . We decompose the field into its Fourier components at $\displaystyle t = 0$
\\
\\
\begin{equation}
    \vec{A} \left( \vec{x}, t = 0 \right)  = \frac{1}{ \sqrt{V} } \sum_{k} \sum_{\alpha = 1}^{2} \hat{\epsilon}^{\left( \alpha \right) } \left( c_{k, \alpha} \left( t = 0 \right) e^{i \vec{k} \cdot \vec{x}} + c^{*}_{k, \alpha} \left( t = 0 \right) e^{-i \vec{k} \cdot \vec{x}} \right) 
\end{equation}
\\
\\
where $\displaystyle \hat{\epsilon}^{\left( \alpha \right) }$ are real unit vectors, and $\displaystyle c_{k, \alpha}$ is the coefficient of the wave with wave vector $\displaystyle \vec{k}$ and polarization vector $\displaystyle \hat{\epsilon}^{\left( \alpha \right) }$. Once the wave vector is chose, the two polarization vectors must be picked so that $\displaystyle \hat{\epsilon}^{\left( 1 \right) }$, $\displaystyle \hat{\epsilon}^{\left( 2 \right) }$, and $\displaystyle \vec{k}$ form a right handed orthogonal system. The components of the wave vector must satisfy
\\
\\
\begin{equation}
    k_{i} = \frac{2 \pi n_{i}}{L}
\end{equation}
\\
\\
due to the periodic boundary conditions. The factor out front is set to normalize the states nicely since
\\
\\
\begin{equation}
    \frac{1}{V} \int  \, d^3 x e^{i \vec{k} \cdot \vec{x}} e^{-i \vec{k}' \vec{x}} = \delta_{\vec{k} \vec{k}'}
\end{equation}
\\
\\
and
\\
\\
\begin{equation}
    \hat{\epsilon}^{\left( \alpha \right) } \cdot \hat{\epsilon}^{\left( \alpha' \right) } = \delta_{\alpha \alpha'}
\end{equation}
\\
\\
We know the time dependence of the waves from Maxwell's equation,
\\
\\
\begin{equation}
    c_{k, \alpha} \left( t \right) = c_{k, \alpha} \left( 0 \right) e^{-i \omega t}
\end{equation}
\\
\\
where $\displaystyle \omega = kc$. We can now write the vector potential as a function of position and time.
\\
\\
\begin{equation}
    \vec{A} \left( \vec{x}, \vec{t} \right) = \frac{1}{\sqrt{V} } \sum_{k} \sum_{\alpha = 1}^{2} \hat{\epsilon}^{\left( \alpha \right) } \left( c_{k, \alpha} \left( t \right) e^{i \vec{k} \cdot \vec{x}} + c^{*}_{k, \alpha} \left( t \right) e^{-i \vec{k} \cdot \vec{x}}  \right) 
\end{equation}
We may need to write this solution in several different ways, and use the best one for the calcualtion being performed. One nice way to write this is in terms 4-vector $\displaystyle k_{\mu}$, the wave number,
\\
\\
\begin{equation}
    k_{\mu} = \frac{p_{\mu}}{\hbar} = \left( k_{x}, k_{y}, k_{z}, ik \right)  = \left( k_{x}, k_{y}, k_{z}, i \frac{\omega}{c} \right) 
\end{equation}
\\
\\
so that
\\
\\
\begin{equation}
    k_{p} x_{p} = k \cdot x = \vec{k} \cdot \vec{x} - \omega t
\end{equation}
\\
\\
We can then write the radiation field in a more covariant way
\\
\\
\begin{equation}
    \vec{A} \left( \vec{x}, t \right) = \frac{1}{\sqrt{V} } \sum_{k} \sum_{\alpha = 1}^{2} \hat{\epsilon}^{\left( \alpha \right) } \left( c_{k, \alpha} \left( 0 \right) e^{ik_{p} x_{p}} + c^{*}_{k, \alpha} \left( 0 \right) e^{-i k_{p} x_{p}} \right) 
\end{equation}
\\
\\
A convenient shorthand for calculations is possible by noticing that the second term is just the complex conjugate of the first.
\\
\\
\begin{align*}
	\vec{A} \left( \vec{x}, t \right) &= \frac{1}{\sqrt{V} } \sum_{k}  \sum_{\alpha = 1}^{2} \hat{\epsilon}^{\left( \alpha \right) } \left( c_{k, \alpha} \left( 0 \right) e^{ik_{p} x_{p}}  + c.c. \right)  \\
	\vec{A} \left( \vec{x}, t \right) &= \frac{1}{\sqrt{V} } \sum_{k}  \sum_{\alpha=1}^{2} \hat{\epsilon}^{\left( \alpha \right) }  c_{k, \alpha} \left( 0 \right) e^{ik_{p} x_{p}} + c.c.
\end{align*}
\\
\\
Note again that we have made this a transverse field by construction. The unit vectors $\displaystyle \hat{\epsilon}^{\left( \alpha \right) }$ are transverse to the deirection of propagation. Also note that we are working in a guage with $\displaystyle A_4 = 0$, so this can also represent the 4-vector form of the potential. The Fourier Decomposition of the radiation field can be be be written very simply.
\\
\\
\begin{equation}
	\boxed{A_{\mu} = \frac{1}{\sqrt{V}} \sum_{k} \sum_{\alpha=1}^{2} \epsilon^{\left( \alpha \right) }_{\mu}   }
\end{equation}





\end{document}
