\documentclass{report}

\input{../templates/preamble}
\input{../templates/macros}
\input{../templates/letterfonts}

\title{\Huge{Quantum Theory of Radiation}}
\author{\Huge{Marcus Allen Denslow}}
\date{2026-01-18}

\begin{document}

\maketitle
\newpage% or \cleardoublepage
% \pdfbookmark[<level>]{<title>}{<dest>}
\pdfbookmark[section]{\contentsname}{toc}
\tableofcontents
\pagebreak

\chapter{Quantum Theory of Radiation}
\section{Transverse and Longitudinal Fields}
In non-relativistic Quantum Mechanics, the static Electric field is represented by a scalar potential, magnetic fields by the vector potential, and the radiation field also through the vector potential. It will be convenient to keep this separation between the large static atomic Electric field and the radiation fields, however, the equations we have contain the four-vector $\displaystyle A_{\mu}$ with all the fields mixed. When the quantize the field, all E and B fields as well as electromagnetic waves will be made up of photons. It is useful to be able to separate the E fields due to fixed charges from the EM radiation from moving charges. This separation is not Lorentz invariant, but it is still useful. Enrico Fermi showed, in 1930, that $\displaystyle A_{\|}$ together with $\displaystyle A_0$ give rise to Coulomb interactions between particles, whereas $\displaystyle A_{\bot}$ gives rise to the EM radiation from moving charges. With this separation, we can maintain the form of our non-relativistic Hamiltonian.
\\
\\
\begin{equation}
	\boxed{\mathbf{H} = \sum_{j} \frac{1}{2m_{j}} \left( \vec{p} - \frac{e}{c} \vec{A}_{\bot} \left( \vec{x}_{j} \right)  \right)^2 + \sum_{i > j} \frac{e_{i} e_{j}}{4\pi \left|  \right|\vec{x}_{i} - \vec{x}_{j}} + \mathbf{H_{rad}} }
\end{equation}
\\
\\
Where $\displaystyle \mathbf{H_{rad}}$ is purely the Hamiltonian of the radiation (containing only $\displaystyle \vec{A}_{\bot}$) and $\displaystyle \vec{A}_{\bot}$ is the part of the vector potential which satisfies $\displaystyle \nabla \cdot \vec{A}_{\bot} = 0$. Note that $\displaystyle \vec{A}\| \text{ and } A_4$ appear nowhere in the Hamiltonian. Instead, we have the Coulomb potential. This separation allows us to continue with our standard Hydrogen solution and just add radiation. We will not derive this result.
\\
\\
In a region in which there are no source terms,
\\
\\
\begin{equation}
    j_{\mu} = 0
\end{equation}
\\
\\
we can make a gauge transformation which eliminates $\displaystyle A_0$ by choosing $\displaystyle \Lambda$ such that
\\
\\
\begin{equation}
	\frac{1}{c} \frac{ \partial \Lambda }{ \partial t }  = A_0
\end{equation}
\\
\\
Since the fourth component of $\displaystyle A_{\mu}$ is now eliminated, the Lorentz condition now implies that
\\
\\
\begin{equation}
    \vec{\nabla} \cdot \vec{A} = 0
\end{equation}
\\
\\
Again, making one component of a 4-vector zero is not a Lorentz invariant way of working. We have to redo the gauge transformation if we move to another frame.
\\
\\
If $\displaystyle j_{\mu} \neq 0$, then we cannot eliminate $\displaystyle A_0$, since $\displaystyle \boxed A_0 = \frac{j_0}{c}$ and we are only allowed to make gauge transformation for which $\displaystyle \boxed \Lambda = 0$. In this case we must separate the vector potential into the transverse and longitudinal parts, with
\\
\\
\begin{align*}
	\vec{A} &= \vec{A}_{\|} \ \vec{A}_{\bot}
.\end{align*}

\end{document}
