\documentclass{report}

\input{../templates/preamble}
\input{../templates/macros}
\input{../templates/letterfonts}

\title{\Huge{Scattering of Photons}}
\author{\Huge{Marcus Allen Denslow}}
\date{2026-01-14}

\begin{document}

\maketitle
\newpage% or \cleardoublepage
% \pdfbookmark[<level>]{<title>}{<dest>}
\pdfbookmark[section]{\contentsname}{toc}
\tableofcontents
\pagebreak

\chapter{Scattering of Photons}
\section{Scattering of Photons}

In the scattering of photons, for example from an atom, an initial state photon with wave-number $\vec{k}$ and polarization $\displaystyle \hat{\epsilon}$ is absorbed by the atom and a final state photon with wave-number $\vec{k}'$ and polarization $\displaystyle \hat{\epsilon}'$ is emitted. The atom may remain in the same state (elastic scattering) or it may change to another state (inelastic). Any calculation we will do will use the matrix element of the interaction Hamiltonian between initial and final states.
\begin{align*}
	\mathbf{H}_{ni} &= \left<n; \vec{k}' \hat{\epsilon}^{\left( \alpha' \right) }\left| \mathbf{H}_{int} \right|i; \vec{k} \hat{\epsilon}^{\left( \alpha \right) }  \right> \\
	\mathbf{H}_{int} &= - \frac{e}{mc} \vec{A} \left( x \right)  \cdot \vec{p} + \frac{e^2}{2mc^2} \vec{A} \cdot \vec{A}
.\end{align*}
The scattering process clearly requires terms in $\mathbf{H}_{int}$ that annihilate one photon and create another. The order does not matter. The $\displaystyle \frac{e^2}{2mc^2} \vec{A} \cdot \vec{A}$ is the square of the Fourier decomposition of the radiation field so it contains terms like $\displaystyle a^{\dagger}_{k', \alpha'} a_{k, \alpha}$



\end{document}
