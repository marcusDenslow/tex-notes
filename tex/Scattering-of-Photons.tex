\documentclass{report}

\input{../templates/preamble}
\input{../templates/macros}
\input{../templates/letterfonts}

\title{\Huge{Scattering of Photons}}
\author{\Huge{Marcus Allen Denslow}}
\date{2026-01-14}

\begin{document}

\maketitle
\newpage% or \cleardoublepage
% \pdfbookmark[<level>]{<title>}{<dest>}
\pdfbookmark[section]{\contentsname}{toc}
\tableofcontents
\pagebreak

\chapter{Scattering of Photons}
\section{Scattering of Photons}

In the scattering of photons, for example from an atom, an initial state photon with wave-number $\vec{k}$ and polarization $\displaystyle \hat{\epsilon}$ is absorbed by the atom and a final state photon with wave-number $\vec{k}'$ and polarization $\displaystyle \hat{\epsilon}'$ is emitted. The atom may remain in the same state (elastic scattering) or it may change to another state (inelastic). Any calculation we will do will use the matrix element of the interaction Hamiltonian between initial and final states.
\begin{align*}
	\mathbf{H}_{ni} &= \left<n; \vec{k}' \hat{\epsilon}^{\left( \alpha' \right) }\left| \mathbf{H}_{int} \right|i; \vec{k} \hat{\epsilon}^{\left( \alpha \right) }  \right> \\
	\mathbf{H}_{int} &= - \frac{e}{mc} \vec{A} \left( x \right)  \cdot \vec{p} + \frac{e^2}{2mc^2} \vec{A} \cdot \vec{A}
.\end{align*}
The scattering process clearly requires terms in $\mathbf{H}_{int}$ that annihilate one photon and create another. The order does not matter. The $\displaystyle \frac{e^2}{2mc^2} \vec{A} \cdot \vec{A}$ is the square of the Fourier decomposition of the radiation field so it contains terms like $\displaystyle a^{\dagger}_{k', \alpha'} a_{k, \alpha}$, which are just what we want. The $\displaystyle - \frac{e}{mc} \vec{A} \cdot \vec{p}$ term has both creation and annihilation operators in it but not profucts of them. It changes the number of photons by plus or minus one, not by zero as required for the scattering process. Nevertheless this part of the interaction could contribute in second order perturbation theory, by absorbing one photon in a transition from the initial atomic state to an intermediate state, then emitting another photon and making a transition to the final atomic state. While this is higher order in perturbation theory, it is the same order in the electromagnetic coupling constant $e$, which is what really counts when expanding in powers of  $\displaystyle \alpha$. Therefore, we will need to consider the $\displaystyle \frac{e^2}{2mc^2} \vec{A} \cdot \vec{A}$ term in first order and the $\displaystyle -\frac{e}{mc} \vec{A} \cdot \vec{p}$ term in the second order perturbation theory to get an order $\displaystyle \alpha$ calculation of the matrix element. Start with the first order perturbation theory term. All the terms in the sum that do not annihilate the initial state photon and create the final state photon give zero. We will assume that the wavelength of the photon's is long compared to the size of the atom so that $\displaystyle e^{i \vec{k} \cdot \vec{r}} \approx 1$

\begin{align*}
	A_{\mu}\left( x \right) &= \frac{1}{\sqrt{V} } \sum_{k \alpha} \sqrt{\frac{\hbar c^2}{2 \omega}} \epsilon^{\left( \alpha \right) }_{\mu} \left( a_{k, \alpha} \left( 0 \right) e^{i k_{p} x_{p}} + a^{\dagger}_{k, \alpha} \left( 0 \right) e^{-ik_{p}x_{p}}  \right) 
.\end{align*}



\end{document}
