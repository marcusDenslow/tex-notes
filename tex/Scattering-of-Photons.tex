\documentclass{report}

\input{../templates/preamble}
\input{../templates/macros}
\input{../templates/letterfonts}

\usepackage{ amssymb }

\title{\Huge{Scattering of Photons}}
\author{\Huge{Marcus Allen Denslow}}
\date{2026-01-14}

\begin{document}

\maketitle
\newpage% or \cleardoublepage
% \pdfbookmark[<level>]{<title>}{<dest>}
\pdfbookmark[section]{\contentsname}{toc}
\tableofcontents
\pagebreak

\chapter{Scattering of Photons}
\section{Scattering of Photons}

In the scattering of photons, for example from an atom, an initial state photon with wave-number $\vec{k}$ and polarization $\displaystyle \hat{\epsilon}$ is absorbed by the atom and a final state photon with wave-number $\vec{k}'$ and polarization $\displaystyle \hat{\epsilon}'$ is emitted. The atom may remain in the same state (elastic scattering) or it may change to another state (inelastic). Any calculation we will do will use the matrix element of the interaction Hamiltonian between initial and final states.
\begin{align*}
	\mathbf{H}_{ni} &= \left<n; \vec{k}' \hat{\epsilon}^{\left( \alpha' \right) }\left| \mathbf{H}_{int} \right|i; \vec{k} \hat{\epsilon}^{\left( \alpha \right) }  \right> \\
	\mathbf{H}_{int} &= - \frac{e}{mc} \vec{A} \left( x \right)  \cdot \vec{p} + \frac{e^2}{2mc^2} \vec{A} \cdot \vec{A}
.\end{align*}
The scattering process clearly requires terms in $\mathbf{H}_{int}$ that annihilate one photon and create another. The order does not matter. The $\displaystyle \frac{e^2}{2mc^2} \vec{A} \cdot \vec{A}$ is the square of the Fourier decomposition of the radiation field so it contains terms like $\displaystyle a^{\dagger}_{k', \alpha'} a_{k, \alpha}$, which are just what we want. The $\displaystyle - \frac{e}{mc} \vec{A} \cdot \vec{p}$ term has both creation and annihilation operators in it but not profucts of them. It changes the number of photons by plus or minus one, not by zero as required for the scattering process. Nevertheless this part of the interaction could contribute in second order perturbation theory, by absorbing one photon in a transition from the initial atomic state to an intermediate state, then emitting another photon and making a transition to the final atomic state. While this is higher order in perturbation theory, it is the same order in the electromagnetic coupling constant $e$, which is what really counts when expanding in powers of  $\displaystyle \alpha$. Therefore, we will need to consider the $\displaystyle \frac{e^2}{2mc^2} \vec{A} \cdot \vec{A}$ term in first order and the $\displaystyle -\frac{e}{mc} \vec{A} \cdot \vec{p}$ term in the second order perturbation theory to get an order $\displaystyle \alpha$ calculation of the matrix element. Start with the first order perturbation theory term. All the terms in the sum that do not annihilate the initial state photon and create the final state photon give zero. We will assume that the wavelength of the photon's is long compared to the size of the atom so that $\displaystyle e^{i \vec{k} \cdot \vec{r}} \approx 1$

\begin{align*}
	A_{\mu}\left( x \right) &= \frac{1}{\sqrt{V} } \sum_{k \alpha} \sqrt{\frac{\hbar c^2}{2 \omega}} \epsilon^{\left( \alpha \right) }_{\mu} \left( a_{k, \alpha} \left( 0 \right) e^{i k_{p} x_{p}} + a^{\dagger}_{k, \alpha} \left( 0 \right) e^{-ik_{p}x_{p}}  \right)  \\
	\frac{e^2}{2mc^2} \left<n; \vec{k}' \hat{\epsilon}^{\left( \alpha' \right) } \left| \vec{A} \cdot \vec{A} \right|i; \vec{k} \hat{\epsilon}^{\left( \alpha \right) }  \right> &= \frac{e^2}{2mc^2} \frac{1}{V} \frac{\hbar c^2}{2 \sqrt{\omega' \omega} } \epsilon^{\left( \alpha \right) }_{\mu} \epsilon^{\left( \alpha' \right) }_{\mu} \left<n; \vec{k}'\hat{\epsilon}^{\left( \alpha' \right) } \left| \left( a_{k, \alpha} a^{\dagger}_{k', \alpha'} + a^{\dagger}_{k' \alpha'} a_{k, \alpha} \right) e^{i \left( k_{p} - k_{p}' \right)x_{p} }  \right| i; \vec{k} \hat{\epsilon}^{\left( \alpha \right) }   \right> \\ 
	&= \frac{e^2}{2mc^2} \frac{1}{V} \frac{\hbar c^2}{2 \sqrt{\omega' \omega} } \epsilon^{\left( \alpha \right) }_{\mu} \epsilon^{\left( \alpha' \right) }_{\mu} e^{-i \left( \omega - \omega' \right)t } \left<n; \vec{k}' \hat{\epsilon}^{\left( \alpha' \right) } \left| 2 \right| i; \vec{k}' \hat{\epsilon}^{\left( \alpha' \right) }  \right> \\
	&= \frac{e^2}{2mc^2} \frac{1}{V} \frac{\hbar c^2}{2 \sqrt{\omega' \omega} } \epsilon^{\left( \alpha \right) }_{\mu} \epsilon^{\left( \alpha' \right) }_{\mu} e^{-i \left( \omega - \omega' \right)t } 2 \left< n |i \right> \\
	&= \frac{e^2}{2mc^2} \frac{1}{V} \frac{\hbar c^2}{\sqrt{\omega' \omega} }\epsilon^{\left( \alpha \right) }_{\mu} \epsilon^{\left( \alpha' \right) }_{\mu} e^{-i \left( \omega - \omega' \right)t } \delta_{ni}
.\end{align*}
This is the matrix element $\displaystyle \mathbf{H}_{ni}\left( t \right) $. The amplitude to be in the final state $\displaystyle \left| n; \vec{k}' \hat{\epsilon}^{\left( \alpha' \right) } \right> $ is given by first order time dependent perturbation theory.
\begin{align*}
	c^{\left( 1 \right) }_{n} \left( t \right) &= \frac{1}{i \hbar} \int_{0}^{t} e^{i \omega_{ni} t'} \mathbf{H}_{ni}\left( t' \right)  \, dt' \\
	c^{\left( 1 \right) }_{n; \vec{k}' \hat{\epsilon}^{\left( \alpha' \right) }} \left( t \right) &= \frac{1}{i \hbar} \frac{e^2}{2mc^2} \frac{1}{V} \frac{\hbar c^2}{\sqrt{\omega' \omega} } \epsilon^{\left( \alpha \right) }_{\mu} \epsilon^{\left( \alpha' \right) }_{\mu} \delta_{ni} \int_{0}^{t} e^{i \omega_{ni} t'}e^{-i \left( \omega - \omega' \right)t } \, dt' \\
	&= \frac{e^2}{2imV \sqrt{\omega' \omega} } \hat{\epsilon}^{\left( \alpha \right) } \cdot \hat{\epsilon}^{\left( \alpha' \right) }\delta_{ni} \int_{0}^{t} e^{i \left( \omega_{ni} + \omega' \omega \right)t' } \, dt'
.\end{align*}
Recall that the absoulte square of the time integral will turn into $\displaystyle 2\pi t \delta \left( \omega_{ni} + \omega' - \omega \right) $. We will carry along the integral for now, since we are not yet ready to square it.
\\
\\
Now we very carefully put the interaction term into the formula for second order time dependent perturbation theory, again using $\displaystyle e^{i \vec{k} \cdot \vec{x}} \approx 1$. Our notation is that the intermediate state of atom and field is called $\displaystyle \left| I \right> = \left| j, n_{\vec{k}, \alpha}, n_{\vec{k}', \alpha'} \right>  $ where $\displaystyle \mathbf{j}$ represents the state of the atom and we may have zero or two photons, as indicated in the diagram.
\begin{align*}
	\mathcal{V} &= -\frac{e}{mc} \vec{A} \cdot \vec{p} \equiv -\frac{e}{mc} \frac{1}{\sqrt{V} } \sum_{\vec{k} \alpha} \sqrt{\frac{\hbar c^2}{2\omega}} \hat{\epsilon}^{\left( \alpha \right) } \cdot \vec{p} \left( a_{k, \alpha} e^{-i \omega t} + a^{\dagger}_{k, \alpha} e^{ i \omega t} \right) \\
	c^{\left( 2 \right) }_{n} \left( t \right) &= -\frac{1}{\hbar^2} \sum_{j, \vec{k}, \alpha}  \int_{0}^{t} dt_2 \mathcal{V}_{nI} \left( t_2 \right) e^{ i \omega_{nj} t_2} \int_{0}^{t_2} dt_1 e^{i \omega_{ji}t_1}\mathcal{V}_{Ii} \left( t_1 \right) \\
	c^{\left( 2 \right) }_{n; \vec{k}' \hat{\epsilon}^{\left( \alpha' \right) }} \left( t \right) &= -\frac{e^2}{m^2 c^2 \hbar^2} \sum_{I}  \frac{1}{V} \frac{\hbar c^2}{2 \sqrt{\omega' \omega} }\int_{0}^{t} dt_2 \left<n; \vec{k}' \hat{\epsilon}^{\left( \alpha' \right) } \left| \left( \hat{\epsilon}^{\left( \alpha \right) }a_{k, \alpha} e^{-i \omega t_2} + \hat{\epsilon}^{\left( \alpha' \right) } a^{\dagger}_{k', \alpha'} e^{i \omega' t_2} \right) \cdot \vec{p} \right| I  \right> e^{i \omega_{nj}t_2} \\
	&\times \int_{0}^{t_2} dt_1 e^{i \omega_{ji} t_1}\left<I \left| \hat{\epsilon}^{\left( \alpha \right) } a_{k, \alpha} e^{-i ome t_1} + \hat{\epsilon}^{\left( \alpha' \right) }a^{\dagger}_{k', \alpha'} e^{i \omega' t} \cdot \vec{p} \right| i; \vec{k} \hat{\epsilon}^{\left( \alpha \right) }  \right>
.\end{align*}
We can understand this formula as a second order transition from state $\displaystyle \left| i \right>$ to state $\displaystyle \left| n \right>$ through all possible intermediate states. The transition from the initial state to the intermediate state takes place at time $t_1$. The transition from the intermediate state to the final state takes place at time $t_2$.
\\
\\
The space-time diagram below shows the three terms in $\displaystyle c_{n} \left( t \right) $ Time is assumed to run updwards in the diagrams.
\\
\\

% TODO! add diagram here

\noindent
\begin{minipage}[t]{0.48\textwidth}
Diagram (c) represents the $\displaystyle A^2$ term in which one photon is absorbed and one emitted at the same piont. Diagrams (a) and (b) represent two second order terms. In diagram (a) the initial state photon is absorbed at time $\displaystyle t_1$, leaving the atom in an intermediate state which may or may not be the same as the initial (or final) atomic state. This intermediate state has no photons in the field. In digram (b), the atom emits the final state photon at time $\displaystyle t_1$, leaving the atom in some intermediate state. The intermediate state $\displaystyle \left| I \right>$ includes two photons in the field for this diagram. At time $\displaystyle t_2$ the atom absorbs the initial state photon.
\end{minipage}
\hfill
\begin{minipage}[t]{0.48\textwidth}
Looking again at the formula for the second order scattering amplitude, note that we integrate over the times $\displaystyle t_1$ and $\displaystyle t_2$ and that $\displaystyle t_1 < t_2$. For diagram (a), the annihilation operator $\displaystyle a_{k, \alpha}$ is active at time $t_1$ and the creation operator is active at time $\displaystyle t_2$. For diagram (b) it's just the opposite. The second order formula above contains four terms as written. The $\displaystyle a^{\dagger} a$ and $\displaystyle a a^{\dagger}$ terms are the ones described by the diagram. The $\displaystyle aa$ and $\displaystyle a^{\dagger} a^{\dagger}$ terms will clearly give zero. Note that we are just picking the terms that will survive the calculation, not changing any formulas. Now, reduce to the two nonzero terms. The operators give a factor of $\displaystyle 1$ and make the photon states work out. If $\displaystyle \left| j \right>$ is the intermediate atomic state, the second order term reduces to.
\end{minipage}

\begin{align*}
	c^{\left( 2 \right) }_{n; \vec{k}' \hat{\epsilon}^{\left( \alpha' \right) }} &= -\frac{e^2}{2V m^2 \hbar \sqrt{\omega' \omega} } \sum_{j} \int_{0}^{t}  \, dt_2 \int_{0}^{t_2}  \, dt_1 \left[ e^{i \left( \omega' + \omega_{nj} \right)t_2 } \left<n \left| \hat{\epsilon}^{\left( \alpha' \right) } \cdot \vec{p} \right|i  \right> e^{i \left( \omega_{ji} - \omega \right)t_1 } \right. \\
	&\left. + e^{i \left( \omega_{nj} - \omega \right)t_2 } \left<n \left| \hat{\epsilon}^{\left( \alpha \right) } \cdot \vec{p} \right|j  \right> e^{i \left( \omega' + \omega_{ji} \right)t_1 } \right ] \\
	c^{\left( 2 \right) }_{n; \vec{k}' \hat{\epsilon}'} \left( t \right) &= -\frac{e^2}{2V m^2 \hbar \sqrt{\omega' \omega} } \sum_{j}  \int_{0}^{t}  \, dt_2 \left[ e^{i \left( \omega' + \omega_{nj} \right)t_2 } \left<n \left| \hat{\epsilon}' \cdot \vec{p} \right|j  \right> \left<j \left| \hat{\epsilon} \cdot \vec{p} \right|i  \right> \left[ \frac{e^{i \left( \omega_{nj} - \omega \right)t_1 }}{i \left( \omega_{ji} - \omega \right) } \right]^{t_2}_{0}  \right. \\
	&\left.  + e^{i \left( \omega_{nj} - \omega \right)t_2 } \left<n \left| \hat{\epsilon} \cdot \vec{p}  \right|j  \right> \left<j \left| \hat{ \epsilon'} \cdot \vec{p} \right|i  \right> \left[ \frac{e^{i \left( \omega' + \omega_{ji} \right)t_1 }}{i \left( \omega' + \omega_{ji} \right) } \right]^{t_2}_{0}  \right] \\
	c^{\left( 2 \right) }_{n; \vec{k}' \hat{\epsilon}'} \left( t \right)  &= -\frac{e^2}{2V m^2 \hbar \sqrt{\omega' \omega} } \sum_{j} \int_{0}^{t}  \, dt_2 \left[ e^{i \left( \omega' + \omega_{nj} \right)t_2 } \left<n \left| \hat{\epsilon}' \cdot \vec{p} \right|j  \right> \left<j \left| \hat{\epsilon} \cdot \vec{p} \right|j  \right> \left[ \frac{e^{i \left( \omega_{ji} - \omega \right)t_2 } - 1}{i \left( \omega_{ji} - \omega \right) } \right]  \right. \\
	&\left. e^{i \left( \omega_{nj} - \omega \right)t_2 } \left<n \left| \hat{\epsilon} \cdot \vec{p} \right|j  \right> \left<j \left| \hat{\epsilon}' \cdot \vec{p} \right|i  \right> \left[\frac{e^{i \left( \omega' + \omega_{ji} \right)t_2 }-1}{i \left( \omega' + \omega_{ji} \right) }   \right] \right]
.\end{align*}
The $\displaystyle -1$ terms coming from the integration over $\displaystyle t_1$ can be dropped. We can anticipate that the integral over $\displaystyle t_2$ will eventually give us a delta function of energy conservation, going to infinity when energy is conserved and going to zero when it is not. Those $\displaystyle -1$ terms can never go to infinity and can therefore be neglected. When the energy conservation is satisfied, those terms are negligible and when it is not, the whole thing goes to zero.
\begin{align*}
	c^{\left( 2 \right) }_{n; \vec{k}' \hat{\epsilon}'} \left( t \right)  &= -\frac{e^2}{2V m^2 \hbar \sqrt{\omega' \omega} } \sum_{j}  \int_{0}^{t}  \, dt_2 \left[ e^{i \left( \omega_{ni} + \omega' - \omega \right)t_2 } \left<n \left| \hat{\epsilon}' \cdot \vec{p} \right|j  \right> \left<j \left| \hat{\epsilon} \cdot \vec{p} \right|i  \right> \left[ \frac{1}{i \left( \omega_{ji} - \omega \right) } \right] \right. \\
&\left. + e^{i \left( \omega_{ni} + \omega' - \omega \right)t_2 } \left<n \left| \hat{ \epsilon} \cdot \vec{p} \right|j  \right> \left<j \left| \hat{\epsilon'} \cdot \vec{p} \right|i  \right> \left[ \frac{}{} \right] ]   \right]
.\end{align*}






\end{document}
