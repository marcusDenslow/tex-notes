\documentclass{report}

\input{../templates/preamble}
\input{../templates/macros}
\input{../templates/letterfonts}

\title{\Huge{kuleflater}\ undertext}
\author{\Huge{Marcus Allen Denslow}}
\date{}

\begin{document}

\maketitle
\newpage% or \cleardoublepage
% \pdfbookmark[<level>]{<title>}{<dest>}
\pdfbookmark[section]{\contentsname}{toc}
\tableofcontents
\pagebreak

\chapter{Chapter Title}
\section{Section Title}

en kuleflate består av alle punter med en viss lenge, $r$, radius Fra et gitt punk  $s$, sentrum
 \begin{align*}
	 \sqrt{\left( x - x_0 \right)^2 + \left( y - y_0 \right)^2 + \left( z -z_0 \right)^2 } &= r \\
	 \left( x - x_0 \right)^2 + \left( y - y_0 \right)^2 + \left( z - z_0 \right)^2 = r^2
.\end{align*}

oppgave (a)
\begin{align*}
	\left( x-2 \right)^2 + \left( y + 4 \right)^2 + \left( z -1 \right)^2 &= 3^2 \\
	x^2 -4x +4 + y^2 + 2y + 16 + z^2 -2z +1 &= 9 \\
	x^2 -4x +y^2 + 8y + z^2 -2z &= -12
.\end{align*}

oppgave (b)
\begin{equation}
    \left( -2 \right)^2 + 1^2 + 2^2 = 4 + 1 + 4 = 9 \implies \left( 0,-3,3 \right)  \text{ligger på kuleflata}
.\end{equation}

\begin{equation}
    0^2 + 6^2 + 1^2 = 37 > 9 \implies \left( 2,2,2 \right) \text{ligger utenfor kula}
\end{equation}


eksempel 36
\begin{align*}
	x^2 -2x +1 + y^2 -y + 9 + z^2 + 4z + 4 &= 11 + 1 + 9 + 4 \\
	\left( x -1 \right)^2 + \left( y - 3 \right)^2 + \left( z-2 \right)^2 &= 5^2 \\
	S &= \left( 1, 3, -2 \right) \\
	r &= 5
.\end{align*}


eksempel 37

\begin{align*}
	R &= 3 \text{ radius til kula } \\
	A \left( 2,-1,4 \right) &= \text{ sentrum til kule} \\
	r &= \text{ radius til sirkel} \\
	B &= \text{ sentrum til sirkel} \\
	D &= \left| \vec{AB} \right|
.\end{align*}

vi finner $D$ ved å bruke avstand fra punkt til plan

\begin{align*}
	D &= \frac{\left| ax + by cz + d \right| }{\sqrt{a^2 + b^2 + c^2} } \\
	&= \frac{\left| 1 \cdot 2 + 2 \cdot \left( -1 \right) + 2 \cdot 4 - 14 \right| }{\sqrt{1^2 + 2^2 + 2^2} } \\
	&= \frac{6}{\sqrt{9} } = 2
.\end{align*}

finner $r$

\begin{align*}
	r^2 &= R^2 - D^2 \\
	r^2 &= \sqrt{9-4} = \sqrt{5} 
.\end{align*}

Vi bruker linja gjennom $A$ med retningsvektor lik normalvektor til planet og finner deretter skjæringspunkt mellom linja og planet.

\begin{align*}
	l &: \begin{cases} x = 2+t \\ y = -1 + 2t \\ z = 4 + 2t \end{cases}
.\end{align*}
finner skjæringspunktet

\begin{align*}
	x + 2y + 2z &= 14 \\
	2+t - 2 + 4t + 8 + 4t &= 14 \\
	9t &= 6 \\
	t = \frac{2}{3}
.\end{align*}

\begin{align*}
	B &= \left( \frac{8}{3}, \frac{1}{3}, \frac{16}{3} \right) \\
	r &= \sqrt{5} 
.\end{align*}

eksempel 38

 \begin{align*}
	 \vec{n} = \vec{SP} &= \left[ 2, 2, 1 \right] \\
	 \alpha : \begin{cases}
                       2\left( x -4 \right) + 2\left( y - 1 \right) + \left( z-5 \right) = 0 \\ 
		       2x-8 + 2y -2 + z-5 = 0 \\
		       2x + 2y +2 -15 = 0
	 \end{cases}
 .\end{align*}



 \chapter{oppgaver}

 \section{5.109} 

 \qs{Finn likningen til kuleflaten med }{sentrum i origo og radius 2}
 \sol
 \begin{equation}
     r^2 = x^2 + y^2 + z^2
 \end{equation}
 med sentrum i origo $\left( 0,0,0 \right) $ og radius $r = 2$ 
  \begin{equation}
     x^2 + y^2 + z^2 = 4
 \end{equation}

 \qs{b}{sentrum i $\left( 0,4,-3 \right) $ og radius $3$}

\end{document}
