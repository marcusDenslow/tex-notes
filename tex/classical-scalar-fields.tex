\documentclass{report}

\input{../templates/preamble}
\input{../templates/macros}
\input{../templates/letterfonts}

\title{\Huge{Classical Maxwell Fields}}
\author{\Huge{Marcus Allen Denslow}}
\date{2026-02-06}

\begin{document}

\maketitle
\newpage% or \cleardoublepage
% \pdfbookmark[<level>]{<title>}{<dest>}
\pdfbookmark[section]{\contentsname}{toc}
\tableofcontents
\pagebreak

\chapter{Classical Maxwell Fields}
\section{Rationalized Heaviside-Lorentz Units}

The SI units are based on a unit of length of the order of human size originally related to the size of the earth, a unit of time approximately equal to the time between heartbeats, and a unit of mass related to the length unit and the mass of water. None of these depend on any even nearly fundamental physical quantities. Therfore many important physical equations end up with extra (needless) constants in them like $c$. Even with the three basic units defined, we could have chosen the unit of charge correctly to make  $\epsilon_0$ and $\mu_0$ unnecessary but instead a very arbitrary choice was made $\mu_0 = 4\pi \times 10^{-7}$ and the Ampere is defined by the current in parallel wires at one meter distance from each other that gives a force of $2 \times 10^{-7}$ Newtons per meter. The Coulomb is set so that the Ampere isone Coulomb per second. With these choices SI units make Maxwell's equations and our field theory look very messy. 



\end{document}
