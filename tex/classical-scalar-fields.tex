\documentclass{report}

\input{../templates/preamble}
\input{../templates/macros}
\input{../templates/letterfonts}

\title{\Huge{Classical Scalar Fields}}
\author{\Huge{Marcus Allen Denslow}}
\date{2026-02-07}

\begin{document}

\maketitle
\newpage% or \cleardoublepage
% \pdfbookmark[<level>]{<title>}{<dest>}
\pdfbookmark[section]{\contentsname}{toc}
\tableofcontents
\pagebreak

\chapter{Classical Scalar Fields}
\section{Classical Scalar Fields}
The non-relativistic quantum mechanics that we have studied so far developed largely between 1923 and 1925, based on the hypothesis of Plank from the late 19th century. It assumes that a particle has a probability that integrates to one over all space and that the particles are not created or destroyed. The theory neither dels with the quantized electromagnetic field nor with the relativistic energy equation.
\\
\\
It was not long after the non-relativistic theory was completed that Dirac introduced a relativistic theory for electrons. By about 1928, relativistic theories, in which the electromagnetic field was quantized and the creation and absorption of particles was possible, had been developed by Dirac.
\\
\\
Quantum Mechanics became a quantum theory of fields, with the fields for bosons and fermions treated in a symmetric way, yet behaving quite differently. In 1940, Pauli proved the spin-statistics theorem whyich showed why spin one-half particles should behave like fermions and spin zero or spin one particles should have the properties of bosons.
\\
\\
Quantum Field Theory (QFT) was quite successful in describing all detailed experiments in electromagnetic interactions and many aspects of the weak interactions. Nevertheless, by the 1960s, when our textbook was written, most particle theorists were doubtful that QFT was suitable for describing the strong interactions and some aspects of the weak interactions. This all changed dramatically around 1970 when very successful Guage Theories of the strong and weak interactions were introduced. By now, the physics of the electromagnetic, weak, and strong interactions are well described by Quantum Field (Guage) Theories that together form the standard model.
\\
\\
Dirac's relativistic theory of electrons introduced many new ideas such as antiparticles and four component spinors. As we quantize the EM field, we must treat the propagation of photons relativistically. Hence we will work toward understanding relativistic QFT.
\\
\\
In this chapter, we will review classical theory, learn to write our equations in a covariant way in four dimensions, and recall aspects of Lagrangian and Hamiltonian formalisms for use in field theory. The emphasis will be on learning how all these things work and on getting practice with calculations, not on mathematical rigor. While we already have a good deal of knowledge about classical electromagnetism, we will start with simple field theories to get some practice.
\\
\\
\section{Simple Machanical Systems and Fields} 
This section is a review of mechanical systems largely from the point of veiw of Lagrangian dynamics. In particular, we review the equations of a string as an example of a field theory in one dimension.
\\
\\
We start with the Lagrangian of a discrete system like a single particle
\\
\\
\begin{equation}
    L \left( q, \dot{q} \right) = T- V
\end{equation}
\\
\\
Lagrange's equations are
\\
\\
\begin{equation}
    \frac{d}{dt} \left( \frac{ \partial L }{ \partial \dot{q}_{i} }  \right) - \frac{ \partial L }{ \partial q_{i} } = 0
\end{equation}
\\
\\
where $q_{i}$are the coordinates of the particle. This equatino is derivable from the principle of least action.
\\
\\
\begin{equation}
    \delta \int_{t_1}^{t_2} L \left( q_{i}, \dot{q}_{i} \right)  \, dx = 0
\end{equation}
\\
\\
Similarly, we can define the Hamiltonian
\\
\\
\begin{equation}
    \mathbf{H} \left( q_{i}, p_{i} \right) = \sum_{i} p_{i} \dot{p}_{i} - L
\end{equation}
\\
\\
Where $p_{i}$ are momenta conjugate to the coordinates $q_{i}$.
\\
\\
\begin{equation}
    p_{i} = \frac{ \partial L }{ \partial \dot{q}_{i} } 
\end{equation}
\\
\\
Foe a continius system, like a string, the Lagrangian is an integral of a Lagrangian density function.
\\
\\
\begin{equation}
    L = \int \mathcal{L} \, dx
\end{equation}
\\
\\
For example, for a string.
\\
\\
\begin{equation}
    \mathcal{L} = \frac{1}{2} \left[ \mu \dot{\upeta}^2 - Y \left( \frac{ \partial \upeta  }{ \partial x }  \right)^2  \right] 
\end{equation}
\\
\\
Where $Y$ is Young's modulus for the material of the string and  $\mu$ is the mass density. The Euler-Lagrange Equation for a continius system is also derivable from the principle of least action states above. For the string, this would be.
\\
\\
\begin{equation}
    \frac{ \partial  }{ \partial x } \left( \frac{ \partial \mathcal{L } }{ \partial \left( \partial \upeta / \partial x \right)  }  \right) + \frac{ \partial  }{ \partial t } \left( \frac{ \partial \mathcal{L } }{ \partial \left( \partial \upeta / \partial x \right)  }  \right) - \frac{ \partial \mathcal{L } }{ \partial \upeta } = 0
\end{equation}
\\
\\
Recall that the Lagrangian is a function of $\upeta$ and its space and time derivatives.
\\
\\
In this example of a string,  $\upeta \left( x,t \right) $is a simple scalar field. The string has a displacement at each point along it which varies as a function of time.
\\
\\
If we apply the Euler-Lagrange equation, we get a differential equation taht the string's displacement will satisfy.
\\
\\
\begin{align*}
    \mathcal{L} &= \frac{1}{2} \left[ \mu \dot{\upeta}^2 - Y \left( \frac{ \partial \upeta  }{ \partial x }  \right)^2  \right] \\
    \frac{ \partial  }{ \partial x } \left( \frac{ \partial \mathcal{L } }{ \partial \left( \partial \upeta / \partial x \right)  }  \right) + \frac{ \partial  }{ \partial t } \left( \frac{ \partial \mathcal{L } }{ \partial \left( \partial \upeta / \partial t \right)  }  \right) - \frac{ \partial \mathcal{L } }{ \partial \upeta } &= 0 \\
    \frac{ \partial \mathcal{L } }{ \partial \left( \partial \upeta / \partial x \right)  } &= - Y \frac{ \partial \upeta  }{ \partial x } \\
    \frac{ \partial \mathcal{L } }{ \partial \left( \partial \upeta / \partial t \right)  } &= \mu \dot{\upeta} \\
    -Y \frac{ \partial^2 \upeta }{ \partial x^2 } + \mu \ddot{\upeta} + 0 &= 0 \\
    \ddot{\upeta} &= \frac{Y}{\mu} \frac{ \partial^2 \upeta }{ \partial x^2 } 
\end{align*}
\\
\\
This is the wave equation for the string. There are easier ways to get to this wave equation, but, as we move away from simple mechanical systems, a formal way of proceeding will be very helpful.


\section{Classical Scalar Field in Four Dimensions} 
Assume we have a field defined everywhere in space and time. For simplicity we will start with a scalar field (instead of the vector\ldots fields of E&M). 
\\
\\
\begin{equation}
    \phi \left( \vec{r}, t \right) 
\end{equation}
\\
\\
The property that makes this a true scalar field is that it is invariant under rotations and Lorentz boosts. 
\\
\\
\begin{equation}
    \phi \left( \vec{r}, t \right)  = \phi' \left( \vec{r}', t' \right) 
\end{equation}
\\
\\
The Euler-Lagrange equation derived from the principle of least action is 
\\
\\
\begin{equation}
    \sum_{k}  \frac{ \partial  }{ \partial x_{k} } \left( \frac{ \partial \mathcal{L} }{ \partial \left( \partial \phi / \partial x_{k} \right)  }  \right) + \frac{ \partial  }{ \partial t }  \left( \frac{ \partial \mathcal{L } }{ \partial \left( \partial \phi / \partial t \right)  }  \right)  - \frac{ \partial \mathcal{L} }{ \partial \phi } = 0
\end{equation}
\\
\\
\noindent
\begin{minipage}[t]{0.48 \textwidth}
    Note that since there is only one field, there is only one equation. 
    Since we are aiming for a description of relativistic quantum mechanics, it will benefit us to write our equations in a covariant way. I think this also simplifies the equations. We will follow the notation of Sakurai.
\end{minipage}
\hfill
\begin{minipage}[t]{0.48 \textwidth}
    (The convention does not really matter and one should not get hung up on it.) As usual the Latin indices like $i, j, k, \dots$ will run from 1 to 3 and represent the space coordinates. The Greek indices like $\mu, \nu, \sigma, \lambda, \dots$ will run from 1 to 4. Sakurai would give the spacetime coordinate vector either as
\end{minipage}
\\
\\
\begin{equation}
    \left( x_1, x_2, x_3, x_4 \right)  = \left( x, y, z, ict \right)
\end{equation}
\\
\\
or as 
\\
\\
\begin{equation}
    \left( x_0, x_1, x_2, x_3 \right)  = \left( t, x, y, z \right) 
\end{equation}
\\
\\
And use the former to do real computations. 
\\
\\
We will not use the so called covariant and contravariant indices. Instead we will pu an $i$ on the fourth component of a vector which give that component a  $-$ sign in a dot product. 
\\
\\
 \begin{equation}
    x_{\mu} x_{\mu} = x^2 + y^2 + z^2 - c^2 t^2
\end{equation}
\\
\\
Note we can have all lower indices. As Sakurai points out, there is no need for the complication of a metric tensor to raise and lower indices unless general relativity comes into play and the geometry of space-time is not flat. We can assume the $i$ in the fourth component is a calculational convecience, not an indication of the need for complex numbers in our coordinate systems. So while we may have said "farewell to ict" some time in the part, we will use it here because the notation is less complicated. The  $i$ here should never really be used to multiply an  $i$ in the complex wave function, but everything will work out so that doesn't happen unless we make an algebra mistake. 
\\
\\
The spacetime coordinate  $x_{\mu}$ is a Lorentz vector transforming under rotations and boosts as follows. 
\\
\\
\begin{equation}
    x_{\mu} = a_{\mu \nu}x_{\nu}
\end{equation}
\\
\\
(Note that we will always sum over repeated indices, Latin or Greek). The Lorentz transformation is done with a 4 by 4 matrix with the property that the inverse is the transpose of the matrix. 
\\
\\
\begin{equation}
    a^{-1}_{\mu \nu} = a_{\nu \mu}
\end{equation}
\\
\\
The $a_{ij}$ and $a_{44}$ are real while the $a_{4j}$ and $a_{j_4}$are imaginary in our convention. THus we may compute the coordinate using the inverse transformation. 
\\
\\
\begin{equation}
    x_{\mu} = a_{\nu \mu}x_{\nu}'
\end{equation}
\\
\\
Vecotrs transform as we change our reference system by rotating or boosting. Higher rank tensors also transform with one Lrentz transformation matrix per index on the tensor. 
\\
\\
The Lorentz transformation matrix to a coordinate system boosted along the x direction is. 
\\
\\
\begin{equation}
    a_{\mu \nu} = \begin{pmatrix}
                \gamma & 0 & 0 & i \beta \gamma \\
                0 & 1 & 0 & 0 \\
                0 & 0 & 1 & 0 \\
                i \beta \gamma & 0 & 0 & \gamma
            \end{pmatrix} 
\end{equation}
\\
\\
The $i$ shows up on space-time elements. To deal with the $i$ we have put on the time components of 4-vectors. It is interesting to note the similarity between Lorentz boosts and rotations. A rotation in the  $xy$ plane through an angle  $\theta$ is implemented with the transformation. 
\\
\\
\begin{equation}
    a_{\mu \nu} = \begin{pmatrix} \cos \theta & \sin \theta & 0 & 0 \\
                -\sin \theta & \cos \theta & 0 & 0 \\ 
                0 & 0 & 1 & 0 \\
                0 & 0 & 0 & 1 
            \end{pmatrix} 
\end{equation}
\\
\\
A boost along the $x$ direction is like a rotation in the  $xt$ through and angle of $\theta$ where $tanh \theta = \beta$. Dinxe we are in Minkowski space where we need a minus sign on the time component of dot products, we need to add an $i$ in this rotation too. \\
\\
 \begin{equation}
     a_{\mu \nu} = \begin{pmatrix} \cos i \theta & 0 & 0 & \sin i \theta \\ 
                 0 & 1 & 0 & 0 \\ 
                 0 & 0 & 1 & 0 \\ 
             -\sin i \theta & 0 & 0 & \cos i \theta 
         \end{pmatrix} 
         = 
         \begin{pmatrix} \cosh \theta & 0 & 0 & i \sinh \theta \\
                 0 & 1 & 0 & 0 \\ 
                 0 & 0 & 1 & 0 \\ 
                 -i \sinh \theta & 0 & 0 & \cosh \theta 
         \end{pmatrix} 
         = 
         \begin{pmatrix} \gamma & 0 & 0 & i \beta \gamma \\ 
           0 & 1 & 0 & 0 \\ 
           0 & 0 & 1 & 0 \\ 
       -i \beta \gamma & 0 & 0 & \gamma 
   \end{pmatrix} 
 \end{equation}

\\
\\
Effectively, a Lorentz boost is a rotation in which $tan i \theta = \beta$. We will make essentially no use of Lorentz transformations because we will write our theories in terms of Lorentz scalars whenever possible. For example, our Lagrangian density should be invariant. 
\\
\\
\begin{equation}
    \mathcal{L}' \left( x' \right)  = \mathcal{L} \left( x \right) 
\end{equation}
\\
\\
The Lagrangians we have seen so far have derivatives with respect to the coordinates. THe 4-vector way of writing this will be $\frac{ \partial  }{ \partial x_{\mu} } $. We need to know what the transformation properties of this are. We can compute this form the transformatinos and the chain rule. 
\\
\\
\begin{align*}
    x_{\nu} &= a_{\mu \nu} x_{\mu}' \\
    \frac{ \partial  }{ \partial x_{\mu}' }  &= \frac{ \partial x_{\nu } }{ \partial x_{\mu}' } \frac{ \partial  }{ \partial x_{\nu} } = a_{\mu \nu} \frac{ \partial  }{ \partial x_{\nu} } 
.\end{align*}
\\
\\
This means that it transforms like a vector. Compare it to our original transformation formula for $x_{\mu}$. 
\\
\\
\begin{equation}
    x_{\mu}' = a_{\mu \nu} x_{\nu}
\end{equation}
\\
\\
We may safely assume that all our derivatives with one index transform as a vector. 
\\
\\
With this, lets work on the Euler-Lagrange equation to get it into covariant shape. Remember that the fiel $\phi$ is a Lorentz scalar. 
\\
\\
\begin{align*}
    \sum_{k} \frac{ \partial  }{ \partial x_{k} } \left( \frac{ \partial \mathcal{L } }{ \partial \left( \partial \phi / \partial x_{k} \right)  }  \right)  + \frac{ \partial  }{ \partial t }  \left( \frac{ \partial \mathcal{L } }{ \partial \left( \partial \phi / \partial t \right)  }  \right)  - \frac{ \partial \mathcal{L } }{ \partial \phi } &= 0 \\
    \sum_{k} \frac{ \partial  }{ \partial x_{k} }  \left( \frac{ \partial \mathcal{L } }{ \partial \left( \partial \phi / \phi x_{k} \right)  }  \right) + \frac{ \partial  }{ \partial \left( ict \right)  }  \left( \frac{ \partial \mathcal{L } }{ \partial \left( \partial \phi / \partial \left( ict \right)  \right)  }  \right) - \frac{ \partial \mathcal{L } }{ \partial \phi }  &= 0 \\
    \frac{ \partial  }{ \partial x_{\mu} } \left( \frac{ \partial \mathcal{L } }{ \partial \left( \partial \phi / \partial x_{\mu} \right)  }  \right)  - \frac{ \partial \mathcal{L } }{ \partial \phi } &= 0
.\end{align*}




\end{document}

