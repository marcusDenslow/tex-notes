\documentclass{report}

\input{../templates/preamble}
\input{../templates/macros}
\input{../templates/letterfonts}

\title{\Huge{Classical Scalar Fields}}
\author{\Huge{Marcus Allen Denslow}}
\date{2026-02-07}

\begin{document}

\maketitle
\newpage% or \cleardoublepage
% \pdfbookmark[<level>]{<title>}{<dest>}
\pdfbookmark[section]{\contentsname}{toc}
\tableofcontents
\pagebreak

\chapter{Classical Scalar Fields}
\section{Classical Scalar Fields}
The non-relativistic quantum mechanics that we have studied so far developed largely between 1923 and 1925, based on the hypothesis of Plank from the late 19th century. It assumes that a particle has a probability that integrates to one over all space and that the particles are not created or destroyed. The theory neither dels with the quantized electromagnetic field nor with the relativistic energy equation



\end{document}
