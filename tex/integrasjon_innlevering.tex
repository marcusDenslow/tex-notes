\documentclass{report}

\input{../templates/preamble_no}
\input{../templates/macros_no}
\input{../templates/letterfonts}

\title{\Huge{Matematikk R2}\\Innlevering kap 2\\Integrasjon}
\author{\huge{Marcus Denslow}}
\date{29.10.25}

\begin{document}

\maketitle
\pagebreak

\chapter{Innlevering kap 2 - Integrasjon}


\section{Oppgave 1}

\thm{Delvis integrasjon}{
La $u$ og $v$ være deriverbare funksjoner av $x$. Da gjelder
$$\int u \, dv = uv - \int v \, du$$
eller ekvivalent
$$\int u(x)v'(x) \, dx = u(x)v(x) - \int u'(x)v(x) \, dx$$
}

\ex{Delvis integrasjon}{
Vi skal beregne $\int x e^x \, dx$.

La $u = x$ og $dv = e^x \, dx$. Da får vi $du = dx$ og $v = e^x$.

Ved delvis integrasjon:
\begin{align*}
\int x e^x \, dx &= x e^x - \int e^x \, dx \\
&= x e^x - e^x + C \\
&= e^x(x-1) + C
\end{align*}
}

\qs{1a}{Finn et integral som man kan løse ved metoden delvis integrasjon, men som ikke kan løses ved variabelskifte eller delbrøksoppspalting. Forklar hvorfor bare den ene metoden fungerer og løs integralet.}

\sol

Integralet som skal løses er
$$\int x \sin(x) \, dx$$

\textbf{Hvorfor virker bare delvis integrasjon?}

\begin{itemize}
\item \textbf{Variabelskifte fungerer ikke:} Det finnes ingen naturlig substitusjon som forenkler integralet. Med $u = x$ forsvinner ikke $\sin(x)$, og med $u = \sin(x)$ får man $du = \cos(x) \, dx$, men integranden inneholder $x$, ikke $\cos(x)$.

\item \textbf{Delbrøksoppspalting fungerer ikke:} Metoden brukes bare for rasjonale funksjoner (brøker av polynomer). Siden integralet inneholder $\sin(x)$, som ikke er en rasjonell funksjon, kan ikke metoden brukes her.

\item \textbf{Delvis integrasjon fungerer:} Integralet er et produkt av to ulike funksjonstyper ($x$ og $\sin(x)$), noe som passer perfekt for delvis integrasjon.
\end{itemize}

\textbf{Løsning med delvis integrasjon:}

Med formelen $\int u \, dv = uv - \int v \, du$ setter man $u = x$ og $dv = \sin(x) \, dx$.

Dette gir:
\begin{align*}
du &= dx \\
v &= -\cos(x)
\end{align*}

Ved delvis integrasjon:
\begin{align*}
\int x \sin(x) \, dx &= x \cdot (-\cos(x)) - \int (-\cos(x)) \, dx \\
&= -x \cos(x) + \int \cos(x) \, dx \\
&= -x \cos(x) + \sin(x) + C
\end{align*}

\textbf{Svar:} $\displaystyle \int x \sin(x) \, dx = -x \cos(x) + \sin(x) + C$

\vspace{2cm}

\nt{
\textbf{Delbrøksoppspalting} brukes når vi har en rasjonell funksjon (brøk av polynomer) som kan deles opp i enklere brøker. For eksempel kan $\frac{1}{(x-1)(x+2)}$ skrives som summen av to enklere brøker: $\frac{A}{x-1} + \frac{B}{x+2}$.
}

\qs{1b}{Finn et integral som man kan løse ved metoden delbrøksoppspalting, men som ikke kan løses ved variabelskifte eller delvis integrasjon. Forklar hvorfor bare den ene metoden fungerer og løs integralet.}

\sol
$$\int \frac{1}{x^2-1} \, dx$$

\textbf{Hvorfor virker bare delbrøksoppspalting?}

\begin{itemize}
	\item \textbf{Variabelskifte fungerer ikke:} Det finnes ingen naturlig substitusjon som forenkler integralet. Nevneren $x^2-1$ er et polynom som ikke blir enklere ved substitusjon.

	\item \textbf{Delvis integrasjon fungerer ikke:} Delvis integrasjon krever et produkt av to funksjoner der man kan derivere den ene og integrere den andre. Her har man en rasjonell funksjon (brøk), ikke et produkt.

	\item \textbf{Delbrøksoppspalting fungerer:} Dette er en ekte rasjonell funksjon der tellerens grad er mindre enn nevnerens grad, og nevneren kan faktoriseres. Perfekt for delbrøksoppspalting.
\end{itemize}

\textbf{Løsning med delbrøksoppspalting:}

Nevneren faktoriseres:
$$x^2 - 1 = (x-1)(x+1)$$

Brøken spaltes opp:
$$\frac{1}{(x-1)(x+1)} = \frac{A}{x-1} + \frac{B}{x+1}$$

Multipliserer begge sider med $(x-1)(x+1)$:
$$1 = A(x+1) + B(x-1)$$

For å finne $A$ og $B$ settes det inn strategiske verdier av $x$:

For $x = 1$:
$$1 = A(2) + B(0) \quad \Rightarrow \quad A = \frac{1}{2}$$

For $x = -1$:
$$1 = A(0) + B(-2) \quad \Rightarrow \quad B = -\frac{1}{2}$$

Altså:
$$\frac{1}{x^2-1} = \frac{1/2}{x-1} - \frac{1/2}{x+1}$$

Integralet blir da:
\begin{align*}
	\int \frac{1}{x^2-1} \, dx &= \int \left( \frac{1/2}{x-1} - \frac{1/2}{x+1} \right) dx \\
	&= \frac{1}{2} \int \frac{1}{x-1} \, dx - \frac{1}{2} \int \frac{1}{x+1} \, dx \\
	&= \frac{1}{2} \ln|x-1| - \frac{1}{2} \ln|x+1| + C \\
	&= \frac{1}{2} \ln\left|\frac{x-1}{x+1}\right| + C
\end{align*}

\textbf{Svar:} $\displaystyle \int \frac{1}{x^2-1} \, dx = \frac{1}{2} \ln\left|\frac{x-1}{x+1}\right| + C$

\nt{
Integranden er ikke definert for $x = \pm 1$, så resultatet gjelder på hvert sammenhengende intervall der $x \neq \pm 1$, det vil si for $x \in (-\infty, -1)$, $x \in (-1, 1)$ eller $x \in (1, \infty)$.
}

\pagebreak

\section{Oppgave 2}

\thm{Variabelskiftesetningen}{
La $g$ være en deriverbar funksjon og $f$ være kontinuerlig. Da gjelder
$$\int f(g(x)) \cdot g'(x) \, dx = \int f(u) \, du$$
der $u = g(x)$ og $du = g'(x) \, dx$.

For bestemte integraler:
$$\int_a^b f(g(x)) \cdot g'(x) \, dx = \int_{g(a)}^{g(b)} f(u) \, du$$
}

\ex{Variabelskifte}{
Vi skal beregne $\int 2x(x^2+1)^3 \, dx$.

La $u = x^2 + 1$. Da er $du = 2x \, dx$.

Ved variabelskifte:
\begin{align*}
\int 2x(x^2+1)^3 \, dx &= \int u^3 \, du \\
&= \frac{u^4}{4} + C \\
&= \frac{(x^2+1)^4}{4} + C
\end{align*}
}

\qs{2a}{Bruk derivasjon til å vise at $\displaystyle \int\frac{1}{(1-x)^2} dx  = \frac{x}{1-x} + C_1$}

\sol
For å vise at $\displaystyle \int \frac{1}{(1-x)^2} \, dx = \frac{x}{1-x} + C_1$ deriveres høyre side. Hvis resultatet blir integranden, stemmer likningen.

Deriverer $\displaystyle \frac{x}{1-x}$ med kvotientregelen: $\displaystyle \left( \frac{u}{v} \right)' = \frac{u'v - uv'}{v^2}$

Med $u = x$ og $v = 1-x$ får man $u' = 1$ og $v' = -1$.

\begin{align*}
	\frac{d}{dx} \left( \frac{x}{1-x} \right) &= \frac{1 \cdot (1-x) - x \cdot (-1)}{(1-x)^2}\\
	&= \frac{1-x + x}{(1-x)^2} \\
	&= \frac{1}{(1-x)^2}
\end{align*}

Dette er akkurat integranden.

Siden derivasjon og integrasjon er motsatte operasjoner, følger det at:
$$\int \frac{1}{(1-x)^2} \, dx = \frac{x}{1-x} + C_1$$

$\square$


\vspace{2cm}

\qs{2b}{Bruk variabelskiftet $u = 1-x$ til å vise at $\displaystyle\int\frac{1}{(1-x)^2}dx  = \frac{1}{1-x} + C_2$}

\sol

Med variabelskiftet $u = 1-x$ får man:
\begin{align*}
\frac{du}{dx} &= -1 \\
du &= -dx \\
dx &= -du
\end{align*}

Substituerer inn i integralet:
\begin{align*}
\int\frac{1}{(1-x)^2}dx &= \int\frac{1}{u^2} \cdot (-du) \\
&= -\int u^{-2} \, du
\end{align*}

Integrerer:
\begin{align*}
-\int u^{-2} \, du &= -\left( \frac{u^{-1}}{-1} \right) + C_2 \\
&= -\left( -\frac{1}{u} \right) + C_2 \\
&= \frac{1}{u} + C_2
\end{align*}

Tilbakesubstitusjon med $u = 1-x$ gir:
$$\int\frac{1}{(1-x)^2}dx = \frac{1}{1-x} + C_2$$

$\square$

\vspace{2cm}

\qs{2c}{Forklar hvordan begge disse formlene kan være samme samtidig.}

\sol

To forskjellige uttrykk for det samme integralet:

\begin{align*}
\text{Fra 2a:} \quad &\int\frac{1}{(1-x)^2} dx = \frac{x}{1-x} + C_1 \\
\text{Fra 2b:} \quad &\int\frac{1}{(1-x)^2} dx = \frac{1}{1-x} + C_2
\end{align*}

Disse ser forskjellige ut, men de representerer faktisk samme familie av funksjoner. For å se sammenhengen, omskrives $\displaystyle\frac{x}{1-x}$.

Siden $x = (x-1) + 1$ kan man skrive:

\begin{align*}
\frac{x}{1-x} &= \frac{(x-1) + 1}{1-x} \\
&= \frac{x-1}{1-x} + \frac{1}{1-x} \\
&= \frac{-(1-x)}{1-x} + \frac{1}{1-x} \\
&= -1 + \frac{1}{1-x} \\
&= \frac{1}{1-x} - 1
\end{align*}

Dermed:
$$\frac{x}{1-x} + C_1 = \frac{1}{1-x} - 1 + C_1 = \frac{1}{1-x} + (C_1 - 1)$$

Med $C_2 = C_1 - 1$ får man:
$$\frac{x}{1-x} + C_1 = \frac{1}{1-x} + C_2$$

Begge formlene er altså riktige og representerer samme familie av antideriverte. Forskjellen ligger bare i integrasjonskonstanten, der $C_2 = C_1 - 1$. Dette viser at integralet til en funksjon ikke er unikt bestemt, men representerer en hel familie av funksjoner som kun skiller seg med en konstant.

\nt{
Integranden er ikke definert for $x = 1$, så uttrykkene gjelder for intervaller der $x \neq 1$.
}


\pagebreak

\section{Oppgave 3}

\dfn{Bestemt integral (Riemann-sum)}{
Det bestemte integralet $\int_a^b f(x) \, dx$ kan tilnærmes ved å dele intervallet $[a,b]$ inn i $n$ like store delintervaller av bredde $\Delta x = \frac{b-a}{n}$.

Riemann-summen er da:
$$\int_a^b f(x) \, dx \approx \sum_{i=1}^{n} f(x_i) \cdot \Delta x$$
der $x_i$ er et punkt i det $i$-te delintervallet. Når $n \to \infty$ konvergerer summen mot det eksakte integralverdien.
}

Gitt funksjonen $f(x) = \sqrt{1-x^2}$.

\qs{3a}{Lag et program som finner en tilnærming på arealet som er avgrenset av $x$-aksen og $f(x)$.}

\sol

For å finne arealet under $f(x) = \sqrt{1-x^2}$ på intervallet $[-1, 1]$ brukes Riemann-summer. Intervallet deles inn i $n$ like store delintervaller, og arealene av rektanglene summeres.

\textbf{Python-program:}

\begin{lstlisting}[language=Python, basicstyle=\small\ttfamily, keywordstyle=\color{blue}, commentstyle=\color{green!60!black}, stringstyle=\color{red}, numbers=left, numberstyle=\tiny, frame=single]
import numpy as np

def f(x):
    """Funksjonen f(x) = sqrt(1 - x^2)"""
    return np.sqrt(1 - x**2)

def riemann_sum(a, b, n):
    """
    Beregner Riemann-sum for f(x) pa intervallet [a, b]
    med n rektangler. Bruker hoyre endepunkt.
    """
    delta_x = (b - a) / n
    total = 0

    for i in range(1, n + 1):
        x_i = a + i * delta_x
        total += f(x_i) * delta_x

    return total

# Beregn tilnaerming med ulike verdier av n
a, b = -1, 1
n_values = [10, 100, 1000, 10000]

print("Tilnaerming av arealet under f(x) = sqrt(1-x^2):")
for n in n_values:
    areal = riemann_sum(a, b, n)
    print(f"n = {n:5d}: Areal ~ {areal:.8f}")
\end{lstlisting}

\textbf{Resultat:}

\begin{verbatim}
Tilnærming av arealet under f(x) = sqrt(1-x^2):
n =    10: Areal ~ 1.51852441
n =   100: Areal ~ 1.56913426
n =  1000: Areal ~ 1.57074374
n = 10000: Areal ~ 1.57079466
\end{verbatim}

Når antall rektangler øker, konvergerer tilnærmingen mot en verdi rundt 1.5708. Dette gir mening siden $f(x) = \sqrt{1-x^2}$ representerer øvre halvdel av en sirkel med radius 1, og arealet av en halvsirkel er $\frac{\pi}{2} \approx 1.5708$.

\vspace{2cm}

\qs{3b}{Hvor stort er det eksakte arealet $\displaystyle\int_{-1}^{1} f(x)dx$? Finn ut hvor mange rektangler man må dele flatestykket opp i for at tilnærmingsverdien fra oppgave a) skal få 8 riktige siffer fra eksakteverdien}

\sol

\textbf{Del 1: Eksakt areal}

Funksjonen $f(x) = \sqrt{1-x^2}$ representerer øvre halvdel av sirkelen $x^2 + y^2 = 1$ med radius 1.

Arealet av en halvsirkel med radius $r$ er $\frac{\pi r^2}{2}$. Med $r = 1$:

$$\int_{-1}^{1} \sqrt{1-x^2} \, dx = \frac{\pi \cdot 1^2}{2} = \frac{\pi}{2} \approx 1.57079633$$

\textbf{Del 2: Antall rektangler for 8 riktige siffer}

For å få 8 riktige siffer må feilen være mindre enn $0.5 \times 10^{-8}$. Binær søk brukes for å finne det minste $n$ som gir tilstrekkelig nøyaktighet.

\textbf{Python-program:}

\begin{lstlisting}[language=Python, basicstyle=\small\ttfamily, keywordstyle=\color{blue}, commentstyle=\color{green!60!black}, numbers=left, numberstyle=\tiny, frame=single]
import numpy as np

def f(x):
    return np.sqrt(1 - x**2)

def riemann_sum(a, b, n):
    delta_x = (b - a) / n
    total = 0
    for i in range(1, n + 1):
        x_i = a + i * delta_x
        total += f(x_i) * delta_x
    return total

eksakt_areal = np.pi / 2
target_accuracy = 0.5e-8

# Binaer sok mellom n = 100000 og n = 1000000
left, right = 100000, 1000000

while right - left > 1000:
    mid = (left + right) // 2
    approx = riemann_sum(-1, 1, mid)
    error = abs(eksakt_areal - approx)

    if error < target_accuracy:
        right = mid  # Funker, prov lavere n
    else:
        left = mid   # For stor feil, treng hoyere n

# Rund opp til naermeste 1000
n_final = ((right + 999) // 1000) * 1000
approx_final = riemann_sum(-1, 1, n_final)
error_final = abs(eksakt_areal - approx_final)

print(f"Minste n: {n_final}")
print(f"Feil: {error_final:.2e}")
\end{lstlisting}

\textbf{Resultat:}

\begin{verbatim}
Eksakt areal: 1.5707963268

Bruker binær sok:
Tester n =  550000: Feil = 4.08e-09 -> Funker!
Tester n =  325000: Feil = 8.98e-09 -> For stor feil
Tester n =  437500: Feil = 5.75e-09 -> For stor feil
Tester n =  493750: Feil = 4.79e-09 -> Funker!
...
Tester n =  480565: Feil = 4.99e-09 -> Funker!

Minste n (rundet til nærmeste 1000): 481000
Feil: 4.99e-09

Dette gir 8 riktige siffer!
\end{verbatim}

Det minste antallet rektangler som trengs er \textbf{481 000}.

\vspace{2cm}

\qs{3c}{Finn ved programmering øvre grense, $b$, slik at $\displaystyle\int_{-1}^{b} f(x)dx = 1$}

\sol

For å finne $b$ brukes binær søk (intervallhalveringsmetoden). Integralet er 0 når $b = -1$ og $\frac{\pi}{2} \approx 1.571$ når $b = 1$. Siden vi søker etter integral = 1, må $b$ ligge mellom $-1$ og $1$.

\textbf{Python-program:}

\begin{lstlisting}[language=Python, basicstyle=\small\ttfamily, keywordstyle=\color{blue}, commentstyle=\color{green!60!black}, numbers=left, numberstyle=\tiny, frame=single]
import numpy as np

def f(x):
    return np.sqrt(1 - x**2)

def riemann_sum(a, b, n):
    delta_x = (b - a) / n
    total = 0
    for i in range(1, n + 1):
        x_i = a + i * delta_x
        total += f(x_i) * delta_x
    return total

# Finn b slik at integral fra -1 til b er 1
target_value = 1.0
n = 10000  # Bruk mange rektangler for god presisjon

# Binaer sok
left, right = -1, 1
tolerance = 1e-6

iterations = 0
while right - left > tolerance:
    iterations += 1
    b_mid = (left + right) / 2
    integral = riemann_sum(-1, b_mid, n)

    if integral < target_value:
        left = b_mid
    else:
        right = b_mid

b_final = (left + right) / 2
integral_final = riemann_sum(-1, b_final, n)

print(f"Resultat: b ~ {b_final:.8f}")
print(f"Integral fra -1 til {b_final:.8f} = {integral_final:.8f}")
\end{lstlisting}

\textbf{Resultat:}

\begin{verbatim}
Resultat: b ~ 0.21624041
Integral fra -1 til 0.21624041 = 1.00000030
\end{verbatim}

Den øvre grensen er $b \approx 0.216$.

Dette kan verifiseres geometrisk: Med $b \approx 0.216$ får man et areal på 1, som er litt under $\frac{2}{3}$ av det totale arealet $\frac{\pi}{2} \approx 1.571$. Dette gir mening siden det søkes etter omtrent $\frac{1}{1.571} \approx 0.637$ av halvsirkelen.

\pagebreak

\section{Oppgave 4}

\thm{Volum ved rotasjon (Skivemetoden)}{
Når et område mellom kurven $y = f(x)$ og $x$-aksen fra $x = a$ til $x = b$ roteres $360^\circ$ om $x$-aksen, får vi et rotasjonslegeme med volum
$$V = \pi \int_a^b [f(x)]^2 \, dx$$

Hvis vi roterer området mellom to kurver $y = f(x)$ og $y = g(x)$ (der $f(x) \geq g(x)$) om $x$-aksen, blir volumet
$$V = \pi \int_a^b \left([f(x)]^2 - [g(x)]^2\right) \, dx$$
}

\nt{
Ved rotasjon om $x$-aksen tenker vi oss at vi lager tynne skiver vinkelrett på $x$-aksen. Hver skive har tykkelse $dx$ og radius $r = f(x)$, så arealet av tverrsnitt er $\pi r^2 = \pi [f(x)]^2$. Volumet blir da summen (integralet) av alle disse skivene.
}

Funksjonene $f$ og $g$ er gitt ved
\begin{align*}
f(x) &= x \\
g(x) &= -\frac{1}{4}x^2 + 2x
\end{align*}

Et flatestykke $F$ er avgrenset av de to grafene.

\qs{4a}{Finn arealet av flatestykket $F$.}

\sol

For å finne arealet av flatestykket $F$ må skjæringspunktene mellom de to grafene finnes først, deretter integreres differansen mellom funksjonene.

\textbf{Skjæringspunktene:}

Setter $f(x) = g(x)$:
\begin{align*}
	 x &= - \frac{1}{4}x^2 + 2x \\
	 \frac{1}{4}x^2 + x -2x &= 0 \\
	 \frac{1}{4}x^2 - x &= 0 \\
	 x \left( \frac{1}{4}x - 1 \right) &= 0
\end{align*}

Dette gir $x = 0$ eller $\displaystyle \frac{1}{4}x = 1 \quad \Rightarrow \quad x = 4$

Skjæringspunktene er $x = 0$ og $x = 4$.

\textbf{Hvilken funksjon er øverst:}

Tester med et punkt mellom skjæringspunktene, f.eks. $x = 2$:
\begin{align*}
	 f(2) &= 2 \\
	 g(2) &= - \frac{1}{4}(2)^2 + 2(2) = -1 + 4 = 3
\end{align*}

Siden $g(2) > f(2)$ ligger $g(x)$ over $f(x)$ på intervallet $[0, 4]$.

\textbf{Arealet:}

\begin{align*}
	A &= \int_{0}^{4} \left[ g(x) - f(x) \right]  \, dx \\
	  &= \int_{0}^{4} \left[ - \frac{1}{4}x^2 + 2x - x \right]  \, dx  \\
	  &= \int_{0}^{4} \left[ - \frac{1}{4}x^2 + x \right]  \, dx  \\
	  &= \left[ - \frac{1}{12}x^3 + \frac{1}{2}x^2 \right]_{0}^{4} \\
	  &= \left( - \frac{1}{12}\cdot 64 + \frac{1}{2} \cdot 16 \right) - 0 \\
	  &= -\frac{64}{12} + 8 \\
	  &= - \frac{16}{3} + \frac{24}{3} \\
	  &= \frac{8}{3}
\end{align*}

\textbf{Svar:} Arealet av flatestykket $F$ er  $\displaystyle \frac{8}{3}$ kvadratenheter.

\vspace{2cm}

\qs{4b}{Finn volumet av den gjenstående kroppen vi får når vi dreier flatestykket $360^\circ$ om $x$-aksen}

\sol

For å finne volumet når flatestykket $F$ roteres om $x$-aksen brukes formelen for rotasjonsvolum mellom to kurver.

Når området mellom to kurver $y = g(x)$ og $y = f(x)$ (der $g(x) \geq f(x)$) roteres om $x$-aksen, blir volumet:
\[
	V = \pi \int_{a}^{b} \left( [g(x)]^2 - [f(x)]^2 \right)  \, dx
\]

Fra oppgave 4a er $g(x) \geq f(x)$ på intervallet $[0, 4]$.

\begin{align*}
	 V &= \pi \int_{0}^{4} \left( [g(x)]^2 - [f(x)]^2 \right)  \, dx \\
	   &= \pi \int_{0}^{4} \left[ \left( - \frac{1}{4}x^2 + 2x \right)^2 - (x)^2  \right]  \, dx
\end{align*}

Utvider $\left( - \frac{1}{4}x^2 + 2x \right)^2 $:

\begin{align*}
	\left( - \frac{1}{4}x^2 + 2x \right)^2 &= \left( - \frac{1}{4}x^2 \right)^2 + 2 \cdot \left( - \frac{1}{4}x^2 \right) \cdot (2x) + (2x)^2 \\
	&= \frac{1}{16}x^{4} - x^3 + 4x^2
\end{align*}

Setter inn:
\begin{align*}
	V &= \pi \int_{0}^{4} \left[ \frac{1}{16}x^{4} - x^3 + 4x^2 - x^2 \right]  \, dx \\
	  &=  \pi \int_{0}^{4} \left[ \frac{1}{16}x^{4} - x^3 + 3x^2 \right] \, dx  \\
	  &= \pi \left[ \frac{1}{80}x^{5} - \frac{1}{4}x^{4} + x^3 \right]_{0}^{4} \\
	  &= \pi \left[ \frac{1}{80} \cdot 1024 - \frac{1}{4} \cdot 256 + 64 \right] - 0 \\
	  &= \pi \left[ \frac{1024}{80}-64+64 \right]  \\
	  &= \pi \cdot \frac{1024}{80} \\
	  &= \pi \cdot \frac{64}{5} \\
	  &= \frac{64\pi}{5}
\end{align*}

\textbf{Svar:} Volumet av rotasjonslegemet er $\displaystyle \frac{64\pi}{5}$ kubikkenheter.

\vspace{2cm}

\qs{4c}{Et flatestykke $G$ er avgrenset av de to grafene, linja $x = a$ og linja $x = b$, der $a$ og $b$ er to tall mellom 0 og 4, og der $a < b$.}

Bruk CAS til å bestemme tallene $a$ og $b$ slik at $G$ får arealet $\displaystyle\frac{11}{6}$, og at volumet blir $\displaystyle\frac{361\pi}{40}$ når vi dreier $G$ $360^\circ$ om $x$-aksen.

\sol

Flatestykket $G$ er avgrenset av grafene til $f(x) = x$ og $g(x) = -\frac{1}{4}x^2 + 2x$, samt linjene $x = a$ og $x = b$.

$a$ og $b$ må finnes slik at:
\begin{itemize}
    \item Arealet: $\displaystyle \int_a^b \left[-\frac{1}{4}x^2 + x\right] dx = \frac{11}{6}$
    \item Volumet: $\displaystyle \pi \int_a^b \left[\frac{1}{16}x^4 - x^3 + 3x^2\right] dx = \frac{361\pi}{40}$
\end{itemize}

\textbf{Løsning med GeoGebra CAS:}

Volumlikningen deles med $\pi$ på begge sider for å forenkle:
$$\int_a^b \left[\frac{1}{16}x^4 - x^3 + 3x^2\right] dx = \frac{361}{40}$$

I GeoGebra CAS skriver vi følgende kommandoer:

\begin{verbatim}
f(x) := x
g(x) := -1/4 * x^2 + 2*x
areal := Integral(g(x) - f(x), x, a, b) = 11/6
volum := Integral((g(x))^2 - (f(x))^2, x, a, b) = 361/40
Solve({areal, volum}, {a, b})
\end{verbatim}

\textbf{Resultat fra CAS:}

GeoGebra gir flere løsninger:
\begin{verbatim}
{{a = 1, b = 3}, {a = 6.12, b = -1.53}, {a = -0.67, b = 2.79},
 {a = 1.12, b = 4.76}, ...}
\end{verbatim}

Løsningen må oppfylle betingelsene:
\begin{itemize}
    \item $0 < a < b < 4$ (begge tall skal være mellom 0 og 4)
    \item $a < b$
\end{itemize}

Den eneste løsningen som oppfyller dette er: $\boxed{a = 1 \text{ og } b = 3}$

\textbf{Verifisering:}

Areal:
\begin{align*}
A &= \int_1^3 \left[-\frac{1}{4}x^2 + x\right] dx = \left[-\frac{1}{12}x^3 + \frac{1}{2}x^2\right]_1^3 \\
&= \left(-\frac{27}{12} + \frac{9}{2}\right) - \left(-\frac{1}{12} + \frac{1}{2}\right) = \frac{11}{6} \quad \checkmark
\end{align*}

Volum (kan verifiseres tilsvarende):
$$V = \pi \int_1^3 \left[\frac{1}{16}x^4 - x^3 + 3x^2\right] dx = \frac{361\pi}{40} \quad \checkmark$$

\textbf{Svar:} $a = 1$ og $b = 3$

\pagebreak

\section{Oppgave 5}

\thm{Geometrisk rekke}{
En uendelig geometrisk rekke på formen
$$a + ar + ar^2 + ar^3 + \ldots = \sum_{n=0}^{\infty} ar^n$$
konvergerer hvis og bare hvis $|r| < 1$, og har da summen
$$S = \frac{a}{1-r}$$
der $a$ er første ledd og $r$ er kvotienten mellom påfølgende ledd.
}

\dfn{Konvergensområde for en potensrekke}{
For en potensrekke $\sum_{n=0}^{\infty} a_n x^n$ er konvergensområdet mengden av alle $x$-verdier der rekka konvergerer.

For en geometrisk rekke $\sum_{n=0}^{\infty} r^n$ er konvergensområdet $|r| < 1$, altså intervallet $(-1, 1)$.
}

\thm{Leddvis integrasjon av potensrekker}{
La $f(x) = \sum_{n=0}^{\infty} a_n x^n$ være en potensrekke med konvergensområde $|x| < R$.

Da kan vi integrere ledd for ledd:
$$\int f(x) \, dx = \sum_{n=0}^{\infty} \int a_n x^n \, dx = \sum_{n=0}^{\infty} \frac{a_n}{n+1} x^{n+1} + C$$

Den integrerte rekka har samme konvergensområde $|x| < R$.
}

\nt{
Når vi integrerer en potensrekke ledd for ledd, beholder den nye rekka samme konvergensområde som den opprinnelige. Dette gir oss en kraftig metode for å finne potensrekkeutviklinger av funksjoner.
}

Vi ser på den uendelige rekka
$$1 - x + x^2 - x^3 + \ldots$$

\qs{5a}{Finn konvergensområdet til rekka.}

\sol

Rekka er:
\[
1-x + x^2 - x^3 + \dots
\]

Dette er en geometrisk rekke på formen:
\[
 \sum_{n=0}^{\infty} ar^{n}
\]

der $a$ er første ledd og $r$ er kvotienten.

Første ledd: $a = 1$

For å finne kvotienten $r$ ser man på forholdet mellom påfølgende ledd:

\begin{align*}
	 \frac{\text{andre ledd}}{\text{første ledd}} &= \frac{-x}{1} = -x \\
	 \frac{\text{tredje ledd}}{\text{andre ledd}} &= \frac{x^2}{-x} = -x \\
	 \frac{\text{fjerde ledd}}{\text{tredje ledd}} &= \frac{-x^3}{x^2} = -x
\end{align*}

Altså er kvotienten $r = -x$.

Rekka kan også skrives som:
\[
\sum_{n=0}^{\infty} (-x)^{n} = \sum_{n=0}^{\infty} (-1)^{n} x^{n}
\]

En geometrisk rekke $\displaystyle \sum_{n=0}^{\infty} ar^{n}$ konvergerer hvis og bare hvis $|r| < 1$.

Her er $r = -x$, så konvergensbetingelsen blir:
\begin{align*}
	 |r| &< 1\\
	 |-x| &< 1 \\
	 |x| &< 1
\end{align*}

Dette gir:
\[
-1 < x < 1
\]

\textbf{Svar:} Konvergensområdet til rekka er $x \in (-1, 1)$ eller $|x| < 1$.

\vspace{2cm}

\qs{5b}{Finn summen $s(x)$ av rekka.}

\sol

Fra oppgave 5a er rekka en geometrisk rekke med:
\begin{itemize}
    \item Første ledd: $a = 1$
	\item Kvotient:  $r = -x$
	\item Konvergensområde:  $|x| < 1$
\end{itemize}

En geometrisk rekke $\displaystyle \sum_{n=0}^{\infty} ar^{n}$ har summen:
\[
	S = \frac{a}{1-r}
\]
når $|r| < 1$.

Setter inn $a = 1$ og $r = -x$:
\begin{align*}
s(x) &= \frac{a}{1-r} \\
&= \frac{1}{1-(-x)} \\
&= \frac{1}{1+x}
\end{align*}

\textbf{Verifisering:}

For eksempel med $x = \frac{1}{2}$ (som ligger i konvergensområdet):

\begin{align*}
	\text{Rekka: } &1 - \frac{1}{2} + \frac{1}{4} - \frac{1}{8} + \dots \\
	\text{Formel: } &s \left( \frac{1}{2} \right)  = \frac{1}{1 + \frac{1}{2}} = \frac{1}{\frac{3}{2}} = \frac{2}{3}
\end{align*}

Dette stemmer med at rekka konvergerer mot $\frac{2}{3}$.

\textbf{Svar:} Summen av rekka er $\displaystyle s(x) = \frac{1}{1+x}$ for $|x| < 1$.

\vspace{2cm}

\qs{5c}{Løs likningene.
\begin{enumerate}
\item $s(x) = 2$
\item $s(x) = \dfrac{1}{3}$
\end{enumerate}
}

\sol

Fra oppgave 5b er $s(x) = \frac{1}{1+x}$ for $|x| < 1$.

\textbf{1) Løs $s(x) = 2$:}

Setter inn $s(x) = \frac{1}{1+x}$:

\begin{align*}
	\frac{1}{1+x} &= 2\\
	1 &= 2(1+x) \\
	1 &= 2 + 2x \\
	-1 &= 2x \\
	x &= - \frac{1}{2}
\end{align*}

\textbf{Sjekk konvergensområdet:}

Løsningen må ligge i konvergensområdet $|x| < 1$:
\[
 \left| - \frac{1}{2} \right|  = \frac{1}{2} < 1 \quad \checkmark
\]

Løsningen er gyldig.

\textbf{Svar:} $x = -\frac{1}{2}$

\textbf{2) Løs $s(x) = \frac{1}{3}$:}

Setter inn $s(x) = \frac{1}{1+x}$:
\begin{align*}
	\frac{1}{1+x} &= \frac{1}{3} \\
	3 &= 1+x \\
	x &= 2
\end{align*}

\textbf{Sjekk konvergensområdet:}

Løsningen må ligge i konvergensområdet $|x| < 1$:
\[
|2| = 2 > 1 \quad \times
\]

Løsningen $x = 2$ ligger utenfor konvergensområdet til rekka. Dette betyr at rekka ikke konvergerer for $x = 2$, og derfor er $s(2) = \frac{1}{3}$ ikke definert i konteksten av denne rekka.

\textbf{Svar:} Ingen løsning (løsningen $x = 2$ ligger utenfor konvergensområdet)

\vspace{2cm}

Vi kan vise at når vi integrerer ledd for ledd en rekke med summen $s(x)$, får vi ei ny rekke med samme konvergensområde og der summen er en antiderivert til $s(x)$.

\qs{5d}{Bruk dette til å finne en rekke som har summen $f(x) = \ln(1+x)$.}

\sol

Fra oppgave 5b:
\[
s(x) = \frac{1}{1+x} = 1 - x + x^2 - x^3 + x^{4} - \ldots = \sum_{n=0}^{\infty} (-1)^{n} x^{n}
\]

for $|x| < 1$.

Deriverer $f(x) = \ln(1+x)$:
\[
f'(x) = \frac{1}{1 + x} = s(x)
\]

Dette betyr at $\ln(1+x)$ er en antiderivert til $s(x)$.

Siden summen av den integrerte rekka skal være en antiderivert til $s(x)$, integreres rekka ledd for ledd:

\begin{align*}
	 \int s(x) \, dx &= \int \sum_{n=0}^{\infty} (-1)^{n} x^{n} \, dx \\
	 &= \sum_{n=0}^{\infty} \int (-1)^{n} x^{n} \, dx \\
	 &= \sum_{n=0}^{\infty} (-1)^{n} \frac{x^{n+1}}{n+1} + C
\end{align*}

Dette gir:
\[
\sum_{n=0}^{\infty} (-1)^{n} \frac{x^{n+1}}{n+1} = \frac{x^{1}}{1} - \frac{x^2}{2} + \frac{x^3}{3} - \frac{x^{4}}{4} + \ldots
\]

Ved å sette $m = n+1$ (så starter summen fra $m=1$):
\[
\sum_{n=0}^{\infty} (-1)^{n} \frac{x^{n + 1}}{n+1} = \sum_{m=1}^{\infty} (-1)^{m-1} \frac{x^{m}}{m}
\]

\textbf{Bestemme integrasjonskonstanten:}

Når $x = 0$:
\begin{align*}
	 \ln(1+0) &= \sum_{m=1}^{\infty} (-1)^{m-1} \frac{0^{m}}{m} + C \\
	 0 &= 0 + C \\
	 C &= 0
\end{align*}

Rekka som har summen $f(x) = \ln(1+x)$ er:
\[
\ln(1+x) = \sum_{n=1}^{\infty} (-1)^{n-1} \frac{x^{n}}{n} = x - \frac{x^2}{2} + \frac{x^3}{3} - \frac{x^{4}}{4} + \ldots
\]

for $|x| < 1$.

\textbf{Svar:}  $\displaystyle \ln(1+x) = \sum_{n=1}^{\infty} \frac{(-1)^{n-1}}{n} x^{n} = x - \frac{x^2}{2} + \frac{x^3}{3} - \frac{x^{4}}{4} + \ldots$ for $|x| < 1$.

\vspace{2cm}

\qs{5e}{La $s_k(x)$ være summen av de fem første leddene i den rekka du fant i oppgave d). Tegn grafen til $f$ og grafen til $s_k$ i et koordinatsystem. Hva ser du?}

\sol

Fra oppgave 5d:
$$\ln(1+x) = x - \frac{x^2}{2} + \frac{x^3}{3} - \frac{x^4}{4} + \frac{x^5}{5} - \ldots$$

\textbf{De fem første leddene:}

Summen av de fem første leddene er:
$$s_5(x) = x - \frac{x^2}{2} + \frac{x^3}{3} - \frac{x^4}{4} + \frac{x^5}{5}$$

\textbf{Tegning av grafene:}

Grafene til:
\begin{itemize}
    \item $f(x) = \ln(1+x)$
    \item $s_5(x) = x - \frac{x^2}{2} + \frac{x^3}{3} - \frac{x^4}{4} + \frac{x^5}{5}$
\end{itemize}

tegnes i samme koordinatsystem for intervallet $[-0.8, 0.8]$ (som ligger innenfor konvergensområdet $|x| < 1$).

\textbf{Observasjoner:}

Ved å se på grafene:

\begin{enumerate}
    \item \textbf{Nær $x = 0$:} Grafene til $f(x) = \ln(1+x)$ og $s_5(x)$ ligger svært nært hverandre. Dette viser at Taylor-polynomet gir god tilnærming nær utviklingspunktet $x = 0$.

    \item \textbf{Lenger fra $x = 0$:} Jo lenger man beveger seg fra $x = 0$, desto større blir forskjellen mellom $f(x)$ og $s_5(x)$. Tilnærmingen blir dårligere.

    \item \textbf{Ved kantene av konvergensområdet:} Når $x$ nærmer seg $-1$ eller $1$ (kantene av konvergensområdet), avviker $s_5(x)$ mer og mer fra $f(x)$.

    \item Dette viser at en Taylor-rekke (potensrekke) konvergerer mot funksjonen innenfor konvergensområdet, men at flere ledd trengs for å få god tilnærming lenger fra utviklingspunktet. Med bare fem ledd får man god tilnærming nær $x = 0$, men tilnærmingen blir dårligere når man beveger seg vekk fra dette punktet.
\end{enumerate}

\textbf{Svar:} Grafene viser at $s_5(x)$ tilnærmer $\ln(1+x)$ godt nær $x = 0$, men avviker mer jo lenger man kommer fra origo. Dette demonstrerer hvordan Taylor-polynomet gir bedre tilnærming nær utviklingspunktet.

\end{document}
