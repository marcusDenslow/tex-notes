\documentclass{report}

\input{../templates/preamble}
\input{../templates/macros}
\input{../templates/letterfonts}

\title{\Huge{Relativistic Quantum Waves (Klein-Gordon Equation}\ }
\author{\Huge{Marcus Allen Denslow}}
\date{2025-11-14}

\begin{document}

\maketitle
\newpage% or \cleardoublepage
% \pdfbookmark[<level>]{<title>}{<dest>}
\pdfbookmark[section]{\contentsname}{toc}
\tableofcontents
\pagebreak

\chapter{Deriving the KG Equation}
\section{double deriving}


\dfn{Relativity: the mass shell}{
	$\displaystyle p \cdot p = (mc)^2 \to (mc)^2 = \left( \frac{E}{c} \right)^2 - p^2_{x} - p^2_{y} - p^2_{z}$
}

\dfn{Quantum: energy and momentum operators}{\[
		\hat{E}  = i \hbar \frac{ \partial  }{ \partial t } \text{, so } \left( \frac{E}{c} \right)^2 \text{ becomes } - \frac{\hbar^2}{c^2}\frac{ \partial^2  }{ \partial t^2 }  
.\] 
\[
\hat{p} = - i \hbar \nabla \text{,so } - p^2_{x} \text{ becomes } \hbar^2 \frac{ \partial^2 }{ \partial x^2 } 
.\] 

\[
\text{likewise, }-p^2_{y} \text{ becomes } \hbar^2 \frac{ \partial^2 }{ \partial y^2 } \text{ and } -p^2_{z} \text{ becomes } \hbar^2 \frac{ \partial^2 }{ \partial z^2 } 
.\] 
}


\section{Plugging in the new values}
we can now plugin these into the original equation:
\begin{align*}
	(mc)^2 &= \left( \frac{E}{c} \right)^2 - p^2_{x} - p^2_{y} - p^2_{y} \\
	(mc)^2 &= - \frac{\hbar^2}{c^2}\frac{ \partial^2 }{ \partial t^2 }  + \hbar^2 \frac{ \partial^2 }{ \partial x^2 }  + \hbar^2 \frac{ \partial^2 }{ \partial y^2 } + \hbar^2 \frac{ \partial^2 }{ \partial z^2 } \\
	\frac{1}{c^2}\frac{ \partial^2 }{ \partial t^2 } - \frac{ \partial^2 }{ \partial x^2 } 	- \frac{ \partial^2 }{ \partial y^2 } - \frac{ \partial^2 }{ \partial z^2 } + \left( \frac{mc}{\hbar} \right)^2 &= 0
.\end{align*}

\section{this is the Klein-Gordon Equation!}
\begin{align}
	\frac{2}{c^2}\frac{ \partial^2 }{ \partial t^2 } - \frac{ \partial^2 }{ \partial x^2 } 	- \frac{ \partial^2 }{ \partial y^2 } - \frac{ \partial^2 }{ \partial z^2 } + \left( \frac{mc}{\hbar} \right)^2 = 0 \\
	\left[   
	\frac{1}{c^2}\frac{ \partial^2 }{ \partial t^2 } - \frac{ \partial^2 }{ \partial x^2 } 	- \frac{ \partial^2 }{ \partial y^2 } - \frac{ \partial^2 }{ \partial z^2 } + \left( \frac{mc}{\hbar} \right)^2 \right] \psi = 0
\end{align}


\section{Replacing with Laplacian}
We know that $\displaystyle -\frac{ \partial^2 }{ \partial x^2 } - \frac{ \partial^2 }{ \partial y^2 } - \frac{ \partial^2 }{ \partial z^2 }  = \nabla^2 \text{ -otherwise known as a Laplacian}$
\\
\\
\\
So the function becomes
\begin{equation}
	\left[ \frac{1}{c^2}\frac{ \partial^2 }{ \partial t^2 }  - \nabla^2 + \left( \frac{mc}{\hbar} \right)^2  \right]\psi = 0
\end{equation}
\section{d'Alembertian}
$\Box = \text{ d'Alembertian}$\\
 We can also rewrite $\displaystyle \left( \frac{mc}{\hbar} \right)^2$ as $\mu^2$
 \nt{We could also write $\displaystyle \left( \frac{mc}{\hbar} \right) $ as just $m$ as in this universe it would become $\left( \frac{m \cdot 1}{1} \right)$ which is just the mass but this is the correct way to write it.}

 So our final equation becomes
 \begin{equation}
     \left[ \Box + \mu^2 \right]\psi = 0
 \end{equation}
\\
\chapter{Four-momentum Eigenstates}
\section{Klein-Gordon Plane Wave}
\dfn{Klein-Gorden Plane Wave function}{
	$\psi = A \text{ exp } \left( - \frac{i}{\hbar}p \cdot x \right)$ \\
	\begin{alignat*}{2}
		p = \left[ E / c, \vec{p} \right] &,  \qquad x = \left[ ct, \vec{x} \right] \\
		A \in \mathbb{C} &, \qquad p^{0} = \frac{E}{c} 0 \pm\sqrt{\left| \vec{p} \right|^2 + m^2c^2 } 
	\end{alignat*}\\
	we can rewrite the original Equation as:
	\begin{align*}
		\psi &= A \text{ exp } \left( \frac{i}{\hbar}\left( \vec{p} \cdot \vec{x} - Et \right)  \right) \\
		\psi &= A \text{ exp } \left( \frac{i}{\hbar}\left( \vec{p} \cdot \vec{x} \pm c \sqrt{\left| \vec{p} \right|^2 + m^2c^2t }  \right)  \right) 
	\end{align*}\\
}
\vspace{2em}
\section{Proof that plane waves $\displaystyle \psi = A \text{ exp } \left[ -ip \cdot x / \hbar \right] $ satisfy K.G.}
rewrite K.G.: $\displaystyle \qquad \left[ \Box + \mu^3 \right]\psi = 0 \to \Box \psi = - \mu^2 \psi$
\\
\\
Does d'Alembertian ddo the same thing as multiplying by $- \mu^2$?
\begin{equation}
	\Box = \Box \left[ \text{exp} \left[ -\frac{i}{\hbar} p \cdot x \right]  \right] = \Box \left[ \text{exp} \left[ \frac{i}{\hbar} \left( \vec{p} \cdot \vec{x} - Et \right)  \right]  \right] 
\end{equation}
Definition of d'Alembertian: $\displaystyle \qquad \Box = \frac{1}{c^2}\frac{ \partial  }{ \partial t^2 } - \nabla^2 $

\begin{equation}
    \Box = \left[ - \left( \frac{E}{c\hbar} \right) + \left( \frac{\left| \vec{p} \right|^2 }{\hbar} \right)  \right] \text{exp} \left[ \frac{i}{\hbar}\left( \vec{p} \cdot \vec{x} - Et \right)  \right] = - \left( \frac{mc}{\hbar} \right)^2 \psi
\end{equation}\ldots

\chapter{Superposition}
\section{You can add together many states that solve K.G., and the resulting sum also solves K.G.}
\vspace{1em}
Another important concept we have to know about when working with the Klein-Gordon Equation is that any superposition of wave functions that satisfy the Klein-Gordon Equation, also satisfy the Klein-Gordon Equation. That lets us create a complex landscape starting with the simple basis set of functions\\ \\ 
Let's say that you have two functions that satisfy the Klein-Gordon Equation, call them $\displaystyle \psi_1$ and $\displaystyle \psi_2$
\begin{equation}
    \left[ \Box + \mu^2 \right]\psi_1 = 0
\end{equation}
\begin{equation}
    \left[ \Box + \mu^2 \right]\psi_2 = 0
\end{equation}
Let's call their sum $\displaystyle \psi_3$ so \[
\psi_3 = \psi_1 + \psi_2
.\] 
The question then arises: does $\displaystyle \psi_3$ satisfy the Klein-Gordon Equation?\\
Well, yes it does. We can show this with:
\begin{align*}
	\left[ \Box + \mu^2 \right]\psi_3 &= \left[ \Box + \mu^2 \right]\left( \psi_1 + \psi_3 \right) \\
	&= \left[ \Box + \mu^2 \right] \psi_1 + \left[ \Box + \mu^2 \right]\psi_2 = 0 + 0 = 0 \text{ ... also satisfies K.G.}
.\end{align*}
\\
The following reasoning does not just apply to the sum of two wave functions, but also scaling up a wave function or taking the arbitrary sum over many wave functions. Or even more usefully if you have a basis set of functions that satisfy the Klein-Gordon Equation, then you can make wave functions of that basis set by summing over them with some series of coefficients
\\
\\
Say we write a wavefunction $\psi$ as a linear combination of $\displaystyle \left\{ \psi_{n} \right\} $
\begin{equation}
    \psi = \sum_{n} C_{n}\psi_{n}, \qquad C_{n} \in \mathbb{C}
\end{equation}
\\
Since each basis function in $\displaystyle \psi_{n} \in \left\{ \psi_{n} \right\}$ satisfies K.G....
\begin{equation}
    \left[ \Box + \mu^2 \right] \psi_{n} = 0 \to \left[ \Box + \mu^2 \right]\sum_{n} C_{n} \psi_{n} = 0
\end{equation}
And that's how we actually use energy and momentum in Eigen-states most of the time.

\chapter{Group Velocity, and c Speed limit}
\section{calculating the group velocity of a Klein-Gordon Wave packet}
In order to calculate the group velocity of a Klein-Gordon Wave packet, we first have to calculate something called the $\mathbf{Dispersion Relation}$.\\ \\
The $\mathbf{Dispersion Relation}$ is when $\mathbf{angular frequency \omega}$ is written as a function of $\mathbf{angular wavenumber }$. \\ \\
General form of plane wave:
\begin{equation}
    \psi = A \text{ exp } \left( i \left( \vec{k} \cdot \vec{x} - \omega t \right)  \right) 
\end{equation}
Klein-Gordon plane wave:
\begin{equation}
    \psi = A \text{ exp } \left( \frac{i}{\hbar} \left( \vec{p} \cdot \vec{x} \pm c \sqrt{\left| \vec{p} \right|^2 + m^2c^2t}  \right)  \right) 
\end{equation}
\begin{alignat*}{2}
	\omega = \frac{c}{\hbar} \sqrt{\left| \vec{p} \right|^2 + m^2c^2 }&, \qquad k= \left| \vec{p} \right| / \hbar \\
	\left| \vec{p} \right|^2 = \hbar^2 k^2 &, \qquad \to \omega = \sqrt{\left( kc \right)^2 + \left( \frac{mc^2}{\hbar} \right)^2 } 
\end{alignat*}

\section{Finding the group relation}
Now we have the Dispersion Relation: $\displaystyle \omega = \sqrt{\left( kc \right)^2 + \left( \frac{mc^2}{\hbar} \right)^2} $
\\
The group velocity is then: $\displaystyle v_{g} = \frac{d \omega}{dk}$
\begin{equation}
    v_{g} = \frac{d}{dk}\left[ \sqrt{\left( kc \right)^2 + \left( mc^2 / \hbar \right)^2  }  \right] 
\end{equation}
\\
with $\displaystyle k = \left| \vec{p} \right| / \hbar$ in mind:
\begin{equation}
    v_{g} = \frac{kc}{\sqrt{k^2 + \left( mc / \hbar \right)^2 } } \to v_{g} = \frac{c \left| \vec{p} \right| }{\sqrt{\left| \vec{p} \right|^2 + \left( mc \right)^2  } }
\end{equation}
\\
Playing around with the Klein-Gordon equation, we have found a speed limit: the speed of light! \\ \\
Massive particle: can only approach $c$ as momentum increases, can never reach $c$.\\
Massless particle: always travels at $c$, regardless of momentum. -- Light always travels at $c$ in all reference frames!
\chapter{Fourier transform and antimatter}
\section{Complex Coefficients on the Mass Shell}
\begin{equation}
	\textcolor{blue}{\psi(x)} = \textcolor{orange}{\frac{1}{\sqrt{2\pi \hbar}}} \textcolor{brown}{\int_{-\infty}^{\infty}} \textcolor{green}{\phi(p)} \textcolor{purple}{\text{exp}\left[-\frac{i}{\hbar}p \cdot x\right]} \, \textcolor{brown}{dp}
\end{equation}

\begin{itemize}
	\item \textcolor{blue}{wavefunction in spacetime}
	\item \textcolor{orange}{Normalization constant}
	\item \textcolor{brown}{Add up eigenstate for each $p$, weighted $\phi(p)$}
	\item \textcolor{green}{Momentum-space wavefunction}
	\item \textcolor{purple}{$p$-eigenstate}
\end{itemize}

\section{The Two Halves of the Mass Shell}
There is a one-to-one connection between all possible wavefunctions that satisfy the Klein-Gordon equation in spacetime and all possible ways of decorating the mass shell with complex numbers.\\
\\
\textbf{The critical insight:} You need \textit{both halves} of the mass shell to have a complete basis set for Fourier transforms from momentum space to position space.\\
\\
Recall from the plane wave solution that:
\begin{equation}
    p^{0} = \frac{E}{c} = \pm \sqrt{\left|\vec{p}\right|^2 + m^2c^2}
\end{equation}
This $\pm$ sign gives us two branches:
\begin{itemize}
    \item \textbf{Positive energy:} $E = +\sqrt{p^2c^2 + m^2c^4}$ (normal particles)
    \item \textbf{Negative energy:} $E = -\sqrt{p^2c^2 + m^2c^4}$ (antimatter!)
\end{itemize}

\section{Dirac's Critique: The Fatal Flaws of Klein-Gordon}
Dirac identified two deeply connected problems with the Klein-Gordon equation that made it unsuitable as a single-particle quantum theory.

\subsection{Problem 1: Second-Order in Time}
The Klein-Gordon equation is \textbf{second-order in time}.\\
\\
Compare the time derivatives:
\begin{align*}
    \text{Schrödinger:} &\quad i\hbar \frac{\partial \psi}{\partial t} = \hat{H}\psi \quad \text{(first-order in time)} \\
    \text{Klein-Gordon:} &\quad \frac{1}{c^2}\frac{\partial^2 \psi}{\partial t^2} - \nabla^2\psi + \left(\frac{mc}{\hbar}\right)^2\psi = 0 \quad \text{(second-order in time)}
\end{align*}
Being second-order in time allows both positive and negative energy solutions and treats space and time on equal footing (manifestly relativistic). However, it creates a serious problem.\\
\\
\textbf{Too much freedom!} Just like classical mechanics needs position AND velocity for second-order equations, Klein-Gordon requires \textit{two} initial conditions:
\begin{itemize}
    \item The wavefunction: $\psi(x, 0)$
    \item The time derivative: $\frac{\partial \psi}{\partial t}\bigg|_{t=0}$
\end{itemize}
In quantum mechanics, the state should be completely determined by $\psi(x,0)$ alone. Having $\partial\psi/\partial t$ as an independent initial condition violates this fundamental principle.

\subsection{Problem 2: Negative Probability Density}
This "too much freedom" problem directly causes the negative probability issue.\\
\\
For Schrödinger, the probability density is simple:
\begin{equation}
    \rho = |\psi|^2 = \psi^* \psi \geq 0 \quad \text{(always positive!)}
\end{equation}
For Klein-Gordon, deriving the probability density from the continuity equation gives:
\begin{equation}
    \rho = \frac{i\hbar}{2mc^2}\left(\psi^* \frac{\partial \psi}{\partial t} - \psi \frac{\partial \psi^*}{\partial t}\right)
\end{equation}
\textbf{This depends on both} $\psi$ \textbf{and} $\partial\psi/\partial t$! Since $\partial\psi/\partial t$ is an independent degree of freedom (Problem 1), we can choose it to make $\rho$ negative. Negative probabilities are physically nonsensical.\\
\\
\textbf{The connection:} The negative probability problem exists \textit{because} the equation is second-order in time. These aren't separate issues — they're two sides of the same coin.

\subsection{Dirac's Solution}
Dirac wanted the impossible:
\begin{enumerate}
    \item \textbf{First-order in time} (like Schrödinger) — needs only $\psi(x,0)$, no extra freedom
    \item \textbf{Relativistically correct} — treats energy and momentum on equal footing
    \item \textbf{Positive definite probability} — $\rho \geq 0$ always
\end{enumerate}
This seemingly impossible requirement led Dirac to discover the \textbf{Dirac equation}, which is first-order in \textit{both} time and space. The price? The wavefunction becomes a multi-component spinor, and quantum mechanical spin emerges naturally!

\section{Reinterpreting Negative Energy: Feynman-Stueckelberg}
While Dirac's equation solved the problems above, the negative energy solutions still appear. But there's a beautiful reinterpretation that doesn't require "negative energy" at all.\\
\\
Recall the energy operator:
\begin{equation}
    \hat{E} = i\hbar \frac{\partial}{\partial t}
\end{equation}
\textbf{Shift your perspective:} Instead of thinking about negative energy, reinterpret what $-\hat{E}$ means:
\begin{equation}
    -\hat{E} = -i\hbar \frac{\partial}{\partial t}
\end{equation}
\begin{center}
\begin{tabular}{c c c}
    Negative energy?? & $\longrightarrow$ & Time reversal! \\
\end{tabular}
\end{center}
\vspace{1em}
A \textit{negative energy} particle moving \textit{forward in time} is mathematically equivalent to a \textit{positive energy} antiparticle moving \textit{backward in time}.\\
\\
This is the \textbf{Feynman-Stueckelberg interpretation}:
\begin{itemize}
    \item Particles: positive energy, moving forward in time
    \item Antiparticles: positive energy, moving backward in time (which \textit{looks like} negative energy forward in time)
\end{itemize}
When an electron and positron annihilate, you can picture the positron as an electron that reversed its direction in time!\\
\\
\textbf{The Resolution:} The negative energy solutions represent \textit{antimatter}. When you include both halves of the mass shell, you're accounting for both particles and antiparticles. Klein-Gordon isn't a single-particle theory — it's fundamentally a quantum field theory equation.


\end{document}
