\documentclass{report}

\input{../templates/preamble}
\input{../templates/macros}
\input{../templates/letterfonts}

\title{\Huge{The Problems with Classical Physics}}
\author{\Huge{Marcus Allen Denslow}}
\date{2026-01-18}

\begin{document}

\maketitle
\newpage% or \cleardoublepage
% \pdfbookmark[<level>]{<title>}{<dest>}
\pdfbookmark[section]{\contentsname}{toc}
\tableofcontents
\pagebreak

\chapter{The Problems with Classical Physics}
\section{idk yet}
By the late nineteenth century the laws of physics were based on Mechanics and the law of Gravitation from Newton, Maxwell's equations describing Electricity and magnetism, and on Statistical Mechanics describing the state of large collection of matter. These laws of physics described nature very well under most conditions, however, some measurements of the late 19th and early 20th century could not be understood. The problems with classical physics led to the development of Quantum Mechanics and Special Relativity.
\\
\\
\begin{itemize}
    \item Black Body Radiation: Classical physics predicted that hot objects would instantly radiate away all their heat into electromagnetic waves. The calculation, which was based on Maxwell's equations and Statistical Mechanics, showed that the radiation rate went to infinity as the EM wavelength went to zero, "The Ultraviolet Catastrophe". Plank solved the problem by postulating that EM energy was emitted in quanta with $\displaystyle E = h \nu$
\end{itemize}




\end{document}
