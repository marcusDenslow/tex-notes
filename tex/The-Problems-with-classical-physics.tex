\documentclass{report}

\input{../templates/preamble}
\input{../templates/macros}
\input{../templates/letterfonts}

\title{\Huge{The Problems with Classical Physics}}
\author{\Huge{Marcus Allen Denslow}}
\date{2026-01-18}

\begin{document}

\maketitle
\newpage% or \cleardoublepage
% \pdfbookmark[<level>]{<title>}{<dest>}
\pdfbookmark[section]{\contentsname}{toc}
\tableofcontents
\pagebreak

\chapter{The Problems with Classical Physics}
\section{idk yet}
By the late nineteenth century the laws of physics were based on Mechanics and the law of Gravitation from Newton, Maxwell's equations describing Electricity and magnetism, and on Statistical Mechanics describing the state of large collection of matter. These laws of physics described nature very well under most conditions, however, some measurements of the late 19th and early 20th century could not be understood. The problems with classical physics led to the development of Quantum Mechanics and Special Relativity.
\\
\\
\begin{itemize}
    \item Black Body Radiation: Classical physics predicted that hot objects would instantly radiate away all their heat into electromagnetic waves. The calculation, which was based on Maxwell's equations and Statistical Mechanics, showed that the radiation rate went to infinity as the EM wavelength went to zero, "The Ultraviolet Catastrophe". Plank solved the problem by postulating that EM energy was emitted in quanta with $\displaystyle E = h \nu$.
	\item The Photoelectric Effect: When light was used to knock electrons out of solids, the results were completely different than exptected from Maxwell's equations. The measurements were easy to explain (for Einstein) if light is made up of particled with the energies Plank postulated.
	\item Atoms: After Rutherford found that the positive charge in atoms was concentrated in a very tiny nucleus, classical physics predicted that the atomic electrons orbitting the nucleus would radiate their energy away and spiral into the nucleus. This clearly did not happen. The energy radiated by atoms also came out in quantized amounts in contradiction to the predictions of classical physics. The Bohr Atom postulated an angular momentum quantization rule, $\displaystyle L = n \hbar \text{ for } n = 1, 2, 3, \dots$, that gave the right result for hydrogen, but turned out to be wrong since the ground state of hydrogen has zero angular momentum. It took a full understanding of Quantum Mechanics to explain the atomic energy spectra.
	\item Compton Scattering: When light was scattered off electrons, it behaved just like a particle but changes wave length in the scattering; more evidence for the particle nature of light and Plank's postulate.
	\item Waves and Particles: In diffraction experiments, light was shown to behave like a wave while in experiments like Photoelectric effect, light behaved like a particle. More difficult diffraction experiments showed that electrons (as well as the other particles) also behaved like a wave, yet we can only detect an integer number of electrons (or photons).
\end{itemize}
Quantum Mechanics incorporates a wave-particle duality and explains all of the above phenomena. In doing so, Quantum Mechanics changes our understanding of nature in fundamental ways. While the classical laws of physics are deterministic, QM is probabilistic. We can only predict the probability that a particle will be found in some region of space.
\\
\\
Electromagnetic waves like light are made up of particles we call photons. Einstein, based on Plank's formula, hypothesized that the particles of light had energy proportional to their frequency.
\begin{equation}
    E = h \nu
\end{equation}
The new idea of Quantum Mechanics is that every particle's probability (as a function of position and time) is equal to the square of a probability amplitude function and that these probability amplitudes obey a wave equation. This is much like the case in electromagnetism where the energy density goes like the square of the field and hence the photon probability density goes to the square of the field, yet the field is made up of waves. So probability amplitudes are like the fields we know from electromagnetism in many ways.
\\
\\
DeBrogile assumed $\displaystyle E = h \nu$ for photons and other particles and used Lorents invariance (from special relativity) to derive the wavelength for particles like electrons.
\\
\\
\begin{equation}
    \lambda = \frac{h}{p}
\end{equation}
\\
\\
The rest of the wave mechanics was built around these ideas, giving a complete picture that could explain the above measurements and could be tested to very high accuracy, particularly in the hydrogen atom. We will spend several chapters exploring these ideas.

\section{Black Body Radiation}
A black body is one that absorbs all the EM radiation (light\ldots) that strikes it. To stay in thermal equilibrium, it must emit radiation at the same rate as it absorbs it so a black body also radiates will. (That's why stoves are black). Radiation from a hot object is familiar to us. Objects around room temperature radiate mainly in the infrared as seen in the graph below.
%TODO! add graph here
If we heat an object up to about 1500 degrees we will begin to see a dull red glow and we say the object is red hot. If we heat something up to about 5000 degrees, near the temperature of the sun's surface, it radiates well throughout the visible spectrum and we say it is white hot.
\\
\\
By considering plates in thermal equilibrium it can be shown that the emissive power over the absorption coefficient must be the same as a function of wavelength, even for plates of different materials.
\\
\\
\begin{equation}
    \frac{E_1 \left( \lambda, T \right) }{A_1 \left( \lambda \right) } = \frac{E_2 \left( \lambda, T \right) }{A_2 \left( \lambda \right) }
\end{equation}
\\
\\
If there were differences, there could be a net energy flow from one plate to the other, violating the equilibrium condition. 
%TODO!add diagram here
A black body is one that absorbs all radiation incident upon it.
\\
\\
\begin{equation}
    A_{BB} = 1
\end{equation}
\\
\\
Thus, the black body Emissive power, $\displaystyle E \left( \nu, T \right) $, is universal and can be derived from first principles.
\\
\\
A good example of a black body is a cavity with a small hole in it. Any light incident upon the hole goes into the cavity and is essentially never reflected out since it would have to undergo a very large number of reflections off walls of the cavity. If we make the walls absorptive (perhaps by painting them black), the cavity makes a perfect black body.%TODO! add picture here
There is a simple relation between the energy density in a cavity, $\displaystyle u \left( \nu, T \right) $, and the black body emissive power of a black body which simply comes from an analysis of how much radiation, traveling at the speed of light, will flow out of a hole in the cavity in one second.
\\
\\
\begin{equation}
    u \left( \nu, T \right)  = \frac{8 \pi \nu^2}{c^{3}}kT
\end{equation}
\\
\\
%TODO! Add picture of Plank here
Plank found a formula that fit the data well at both long and short wavelength
\\
\\
\begin{equation}
    u \left( \nu, T \right)  = \frac{8\pi \nu^2}{c^{3}} \frac{h \nu}{e^{h \nu / kT} - 1}
\end{equation}
\\
\\
His formula fir the data so well that he tried to find a way to derive it. In a few months he was able to do this, by postulating that energy was emitted in quanta with $\displaystyle E = h \nu$. Even though there are a very large number of cavity modes at high frequency, the probability to emit such high energy quanta vanishes exponentially to the Boltzmann distribution. Plank thus suppressed high frequency radiation in the calculation and brought it into agreement with experiment. Note that Plank's Black Body formula is the same in the limit that $\displaystyle h \nu \ll kT$ but goes to zero at very large $\displaystyle \nu$ while the Rayleigh formula goes to infinity.
\\
\\
It is interesting to note that classical EM waves would suck all the thermal energy out of matter, making the universe a very cold place for us. The figure below compares the two calculations to some data at $\displaystyle T = 1600$ degrees. (It is also surprising that the start of the Quantum revolution came from Black Body radiation.)
%TODO add picture here
So the emissive power per unit area is
\\
\\
\begin{equation}
    E \left( \nu, T \right) = \frac{2\pi \nu^2}{c^2} \frac{h \nu}{e^{h \nu / kT} - 1}
\end{equation}
\\
\\
We can integrate this over frequency to get the total power radiated per unit area.
\\
\\
\begin{equation}
    R \left( T \right) = \frac{p^2 c}{60 \left( \hbar c \right)^3}k^{4}T^{4} = \left( 5.67 \times 10^{-8} W / m^2 / K^{4} \right) T^{4}
\end{equation}




\end{document}
