\documentclass{report}

\input{../templates/preamble}
\input{../templates/macros}
\input{../templates/letterfonts}

\title{\Huge{The Problems with Classical Physics}}
\author{\Huge{Marcus Allen Denslow}}
\date{2026-01-18}

\begin{document}

\maketitle
\newpage% or \cleardoublepage
% \pdfbookmark[<level>]{<title>}{<dest>}
\pdfbookmark[section]{\contentsname}{toc}
\tableofcontents
\pagebreak

\chapter{The Problems with Classical Physics}
\section{idk yet}
By the late nineteenth century the laws of physics were based on Mechanics and the law of Gravitation from Newton, Maxwell's equations describing Electricity and magnetism, and on Statistical Mechanics describing the state of large collection of matter. These laws of physics described nature very well under most conditions, however, some measurements of the late 19th and early 20th century could not be understood. The problems with classical physics led to the development of Quantum Mechanics and Special Relativity.
\\
\\
\begin{itemize}
    \item Black Body Radiation: Classical physics predicted that hot objects would instantly radiate away all their heat into electromagnetic waves. The calculation, which was based on Maxwell's equations and Statistical Mechanics, showed that the radiation rate went to infinity as the EM wavelength went to zero, "The Ultraviolet Catastrophe". Plank solved the problem by postulating that EM energy was emitted in quanta with $\displaystyle E = h \nu$.
	\item The Photoelectric Effect: When light was used to knock electrons out of solids, the results were completely different than exptected from Maxwell's equations. The measurements were easy to explain (for Einstein) if light is made up of particled with the energies Plank postulated.
	\item Atoms: After Rutherford found that the positive charge in atoms was concentrated in a very tiny nucleus, classical physics predicted that the atomic electrons orbitting the nucleus would radiate their energy away and spiral into the nucleus. This clearly did not happen. The energy radiated by atoms also came out in quantized amounts in contradiction to the predictions of classical physics. The Bohr Atom postulated an angular momentum quantization rule, $\displaystyle L = n \hbar \text{ for } n = 1, 2, 3, \dots$, that gave the right result for hydrogen, but turned out to be wrong since the ground state of hydrogen has zero angular momentum. It took a full understanding of Quantum Mechanics to explain the atomic energy spectra.
	\item Compton Scattering: When light was scattered off electrons, it behaved just like a particle but changes wave length in the scattering; more evidence
\end{itemize}




\end{document}
