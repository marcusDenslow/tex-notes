
\documentclass{report}

\input{preamble}
\input{macros}
\input{letterfonts}

\title{\Huge{tids-dialatasjon}}
\author{\huge{Marcus Denslow}}
\date{}

\begin{document}

\maketitle
\newpage% or \cleardoublepage
% \pdfbookmark[<level>]{<title>}{<dest>}
\pdfbookmark[section]{\contentsname}{toc}
\tableofcontents
\pagebreak

\chapter{}
\section{oppgaver}
\qs{}{hva vil det si at to hendelser skjer samtidig?}
\sol{det vil si at det skjer i samme tidsperspektiv\\
	det vil si at observtørene er i samme treghtssystem \\
	det vil si at de har synkronisert klokkene sine og måler samme hendelser på samme klokkeslett \\
	de er i samme treghetssystem
}

\qs{}{romulus vil observere at lampene lyser samtidig, jorid vil observer at lyset bak lyser først}
\qs{}{venke vil se lyssignalet fra vestfjell først og etter østhorn. dette er fordi hun står nærmere vestfjell\\
		venke kan måle tidsrommet mellom lyset pg øyden av lynet. dette ettersom begge disse har konstant hastighet\\
		nei han går mot kilden og vekk fra den andre kilden, dette betyr at han ser østhorn først
}

\end{document}
